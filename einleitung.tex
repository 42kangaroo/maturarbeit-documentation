%! suppress = MissingImport
\chapter{Introduction}\label{ch:intro}

In our daily lives, we are in contact with various kinds of algorithms at all times.
From searching on Google to sending texts, over asking questions to AI chatbots and searching for the fastest route with a navy, all of these consume resources in energy, storage space and time.
According to a report from 2021, 3.7 \% of global carbon emissions came from the IT domain, with an upward tendency.
This is similar to that of the airline industry~\cite{webFootprint}.
At this scale, it is thus vitally important to understand and find out if the actual resource consumption can be reduced.

To be able to understand how we can improve, we first need to understand how computers work, what we can compute, and why some problems are more difficult than others to solve.
The field of study that looks into this is called Complexity Theory.
This is done by abstraction of computational models and of the problems themselves.
It is then often possible to find classes of similar problems which allow for generalisations.
However, it has proven to be very hard to find proofs of either optimality of algorithms, separations of complexity classes and general faster algorithms.
One of the emblematic such open problems is the P versus NP question, of which we will speak more in~\cref{subsec:pnp}.
Nevertheless, some important results concerning complexity of problems on average in the real world and most importantly in cryptography have been made.

One of the subfields of Complexity Theory is descriptive complexity.
It relates mathematical logic to different complexity classes.
This allows us to get new insights into the underlying structure of problems of certain classes.
Formal languages are an abstraction of computational problems which allow for multiple equivalent definitions and can be manipulated more easily then a general case.
The famous linguist Noam Chomsky introduced the ``Chomsky Hierarch'', a hierarchy of multiple classes of formal languages.

Towards the end of the 20$^{\text{th}}$ century, many equivalences where proven between fragments and extensions of various logics to complexity classes, including all the Chomsky Hierarchy.
A great summary of all the results is found in Neil Immermans paper ``Languages that capture complexity classes''~\cite{Immerman1987}.

This work consists of two big parts:

Both \cref{ch:formal-languages} and \cref{ch:descriptive-complexity} present the most important theory of formal languages and of descriptive complexity, respectively.
Additionally, in \cref{sec:results-concerning-the-chomsky-hierarchy} whole proofs concerning equivalences in the chomsky hierarchy are presented.
These proofs are written in a way trying to show the motivation behind certain crucial steps.
Further, they are not completely formally correct, some details have been omitted for the sake of simplicity and length of this work.

After this introduction to the material, we go on in \cref{ch:personal-contribution} with a study of context-sensitive languages and formalisms in first-order logic.
Then, the relevant results are summarised in \cref{ch:results}.
Finally, in \cref{ch:conclusion-and-direction} we discuss further possibilities of research.
% TODO more exactly when done

For the relevant mathematical background, \cref{ch:mathematical-background} contains the basic definitions of Set Theory, first and second order logic and finally turing machines.

This whole work is written in english because all mathematical literature is in english, and thus terms need not be translated.