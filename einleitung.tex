\chapter{Einleitung}\label{sec:einleitung}
\section{Übersicht}
Dieses \LaTeX-Dokument soll eine Hilfe bieten, um die Maturaarbeit mit
\LaTeX{} zu verfassen. Das Dokument ist in keiner Weise
vollständig. Sollten Dinge fehlen, inkorrekt sein 
oder haben Sie Vervollständigungsvorschläge (am liebsten als \LaTeX-Code)
melden Sie sich
bitte bei mir (ivo . bloechliger @ ksbg . ch).

In \autoref{sec:struktur} wird die inhaltliche Struktur eines Berichts
erklärt. In \autoref{ch:literatur} wird auf die Verwaltung der
Literaturhinweise eingegangen. 

In den Kapiteln \ref{sec:grafiken} und \ref{sec:compilieren} wird auf das
Einbinden von Grafiken und die Kompilierung dieses Dokuments eingegangen.
In den Kapiteln \ref{sec:mathezeugs}, \ref{sec:massangaben}, \ref{sec:chemie}
und \ref{sec:code} wird das Setzen von Mathematik, Massangaben, Chemische
Formeln und Computercode präsentiert.

Das letzte \autoref{sec:praesentation} beinhaltet Hinweise zur mündlichen
Präsentation der Maturaarbeit.

\section{Struktur eines Berichts}\label{sec:struktur}

Die Gliederung\index{Gliederung} kann je nach Arbeit etwas anders ausfallen, Kapitel in
mehrere aufgeteilt oder zusammengelegt werden. Grundsätzlich sollte
folgende Struktur vorliegen:

\begin{description}
  \item[Vorwort] Ohne Kapitelnummer. Platz für Persönliches. Hier darf das Pronomen
    <<ich>> noch stehen.
\item[Einleitung] \quad\\
  \begin{itemize}
  \item Grobe Beschreibung des Untersuchungsgegenstandes
  \item Einbetten in grösserem Kontext (warum ist es überhaupt
    interessant, sich damit auseinander zu setzen?)
  \item Was haben andere schon dazu geschrieben/geforscht?
  \item Übersicht über den Inhalt des Berichts.
  \end{itemize}
\item[Problemstellung] Präzise Definition des Problems und Ausgangslage.
\item[Lösungen oder Forschungsbericht] \quad \\
  \begin{itemize}
  \item Wie wurde das Problem analysiert?
  \item Welche Lösungsvorschläge wurden gemacht? Begründung der Wahl.
  \item Verwendete Komponenten werden genau dokumentiert und mit Verweisen auf
	  deren genauen Dokumentation/Erklärung versehen (z.B. Datenblätter,
		  API-Dokumentation, etc.)
  \item Was hat nicht funktioniert und warum?
  \end{itemize}
\item[Resultate] \quad \\
  \begin{itemize}
    \item Wie wurden die Resultate getestet?
    \item Was wurde nicht getestet?
    \item Illustrierte Testresultate.
    \item Interpretation der Resultate.
  \end{itemize}
\item[Zusammenfassung] \quad \\
  \begin{itemize}
  \item Was wurde erreicht?
  \item Warum ist das Erreichte bedeutungsvoll?
  \item Wie sind die Resultate mit existierenden Resultaten vergleichbar?
  \item In welche Richtung könnte daran noch weiter gearbietet werden?
  \end{itemize}
\item[Danksagung] Ohne Kapitelnumer. Platz für Persönliches.


test of citing \cite{vorlage} and \cite[12 - 13]{arity-hierarchy}
\end{description}

