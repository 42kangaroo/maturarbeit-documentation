%! suppress = UnresolvedReference
\chapter{Mathematical Background}\label{ch:mathematical-background}

The definitions are taken from the lectures Discrete Mathematics in Computer Science~\cite{discrete-maths} and Theory of Computer Science~\cite{theory-cs}, as well as from the book Descriptive Complexity~\cite{descriptive-complexity}.


\section{Set Theory}\label{sec:set-theory}
\begin{description}
    \item[Set] An unordered collection of distinct elements, written with curly braces $\{\}$
    \item[Tuple] An ordered collection of elements, written with pointed braces $\langle  \rangle$.
    \item[Set operations] There are multiple ways to form new sets from already existing sets:
    \begin{description}
        \item[Union] denoted as $\cup$.
        An element is in $A \cup B$ if and only if it is in $A$ or $B$
        \item[Intersection] denoted as $\cap$.
        An element is in $A \cap B$ if and only if it is in $A$ and $B$
        \item[Cartesian product] denoted as $\times$. $A \times B$ is the set of 2-tuples $\langle a, b\rangle$ with an element $a$ of $A$ and an element $b$ of $B$
        \item[Cartesian power] $A^k$ denotes the Cartesian product of $A$ with itself repeated $k$ times
        \item[Power set] denoted as $\mathcal{P}(A)$.
        Contains all subsets of $A$
    \end{description}
\end{description}


\section{First-Order Logic}\label{sec:first-order-logic}
We abbreviate first-order logic as FO\@. % TODO check acronyms
\begin{description}
    \item[Variable] A variable is an element that can have a value from a set.
    Tuples of variables are denoted by $\overline{x} = \langle x_1, \dots, x_n \rangle$
    \item[Universe] The set over which variables and constants can range
    \item[Relation] A relation of arity $k$, $R(x_1, \dots, x_k)$ can be either true or false for any $k$-tuple of variables.
    In this document, we always consider an equality relation $=$, an ordering relation $\leq$, and BIT$(x, 1y)$, which means that the $y^{th}$ bit of $x$ is $1$ in binary notation, to be present
    \item[Vocabulary] A tuple $\tau = \langle R_1^{a_1}, \dots, R_r^{a_r}, c_1, \dots, c_s \rangle$ of relations $R_i$ with arity $a_i$ and constants $c_j$\footnote{We omit functions, which are included in most textbook definitions, as they can be simulated by a relation in our case}
    \item[Structure] A tuple $\mathcal{A} = \langle |\mathcal{A}|, R_1^{\mathcal{A}}, \dots, R_r^{\mathcal{A}}, c_1^{\mathcal{A}}, \dots, c_s^{\mathcal{A}} \rangle$ where $|\mathcal{A}|$ is the universe, the constants are assigned a value from $|\mathcal{A}|$, and each relation is assigned a truth value for each $a_i$-tuple in $|\mathcal{A}|^{a_i}$.
    The set of all structures for a given universe is denoted as STRUCT$[\tau]$
    \item[First-order formula] A first-order formula is inductively defined as follows:
    \begin{description}
        \item[Atoms] Any formula of the form $R(x_1, \dots, x_k)$ for some relation of arity $k$ is called an atomic formula
        \item[Conjunction] If $\varphi$ and $\psi$ are formulas, $(\varphi \land \psi)$ is a formula
        \item[Disjunction] If $\varphi$ and $\psi$ are formulas, $(\varphi \lor \psi)$ is a formula
        \item[Negation] If $\varphi$ is a formula, $\lnot \varphi$ is a formula
        \item[Existential quantification] If $\varphi$ is a formula, $\exists x \varphi$ is a formula
        \item[Universal quantification] If $\varphi$ is a formula, $\forall x \varphi$ is a formula
    \end{description}
    \item[Free variables] A variable in a formula which occurs at least once without being bound by a quantifier whose scope surrounds it
    \item[Semantics] For any structure, we can assign a truth value to any formula over the corresponding vocabulary.
    If the formula contains free variables, these need to be assigned a value from the universe first.
    We say $\mathcal{A}$ satisfies $\phi$, denoted as $\mathcal{A} \models \phi$, if and only if $\phi$ is true under the interpretation of the constant and relations in $\mathcal{A}$.
    This truth value is inductively assigned to all formulas as follows:
    \begin{description}
        \item[Atoms] For a formula $\phi$ of the form $R(x_1, \dots, x_k)$, we have $\mathcal{A} \models \phi$ if and only if the interpretation of the relation $R^{\mathcal{A}}$ maps $\langle x_1, \dots, x_k \rangle$ to true
        \item[Conjunction] We have $\mathcal{A} \models (\varphi \land \psi)$ if and only if $\mathcal{A} \models \varphi$ and $\mathcal{A} \models \psi$
        \item[Disjunction]  We have $\mathcal{A} \models (\varphi \lor \psi)$ if and only if $\mathcal{A} \models \varphi$ or $\mathcal{A} \models \psi$
        \item[Negation] We have $\mathcal{A} \models \lnot \varphi$ if and only if $\mathcal{A} \not\models \varphi$
        \item[Existential quantification] We have $\mathcal{A} \models \exists x\varphi$ if and only if there exists a $y \in |\mathcal{A}|$ such that $\mathcal{A} \models \varphi(y / x)$, where $\varphi(y / x)$ denotes $\varphi$ with any occurrence of $x$ replaced by the element $y$
        \item[Universal quantification] We have $\mathcal{A} \models \forall x\varphi$ if and only if for all $y \in |\mathcal{A}|$ we have $\mathcal{A} \models \varphi(y / x)$
    \end{description}
    \item[Logical operator] A logical operator can create a new formula from one or more existing formulas.
    One example is the conjunction, which combines two existing formulas.
    \item[First-order queries] A first-order query is a map from structures over one vocabulary $\sigma$ to structures over another vocabulary $\tau$.
    The mapping is done in such a way that first-order formulas define the universe, which is a subset of $|\mathcal{A}|^k$ for some $k$, the relation symbols, and all the constants.
    For a more formally thorough definition see~\cite{descriptive-complexity}
    \item[Isomorphism] An isomorphism is a map $I: |\mathcal{A}| \to |\mathcal{B}|$ with $\mathcal{A}, \mathcal{B}$ over the same vocabulary which satisfies the following properties:
    \begin{itemize}
        \setlength\itemsep{0.15em}
        \item $I$ is bijective
        \item for every available relation $R_i$ of arity $a_i$ and every $a_i$-tuple $\overline{e} = \langle e_1, \dots, e_{a_i} \rangle$ in $|\mathcal{A}|^{a_i}$, we have \[R_i^{\mathcal{A}}(e_1, \cdot, e_{a_i}) \Leftrightarrow R_i^{\mathcal{B}}(I(e_1), \cdot, I(e_{a_i}))\]
        \item for every constant symbol $c_j$, we have $I(c_j^{\mathcal{A}}) = c_j^{\mathcal{B}}$
    \end{itemize}
    If such an $I$ exists for two structures $\mathcal{A}$ and $\mathcal{B}$, we write $\mathcal{A} \cong \mathcal{B}$
\end{description}


\section{Second-Order Logic}\label{sec:second-order-logic}
In second-order logic, we extend the capabilities of first-order logic with the ability to quantify over relations.
We thus also need to extend our definitions.
We abbreviate second-order logic as SO\@.
\begin{description}
    \item[Second-order variables] A relation that is not given in the vocabulary and can be substituted with a specific interpretation
    \item[Second-order formula] In addition to the inductive rules of the FO formulas, we can quantify over second-order formulas
    \begin{description}
        \item[Second-order Existential quantification] If $\varphi$ is a formula, then $\exists V\varphi$ is a formula
        \item[Second-order Universal quantification] If $\varphi$ is a formula, then $\forall V\varphi$ is a formula
    \end{description}
    \item[Second-order Semantics] We also need to extend the first-order semantics
    \begin{description}
        \item[Second-order Existential quantification]  We have $\mathcal{A} \models \exists V\varphi$ if and only if there exists a relation $U$ over $|\mathcal{A}|$ such that $\mathcal{A} \models \varphi(U / V)$, where $\varphi(U / V)$ denotes $\varphi$ with any occurrence of $V$ replaced by $U$
        \item[Second-order Universal quantification] We have $\mathcal{A} \models \forall V\varphi$ if and only if for all relations $U$ over $|\mathcal{A}|$ we have $\mathcal{A} \models \varphi(U / V)$
    \end{description}
\end{description}


\section{Turing Machines}\label{sec:turing-machines}
Turing machines are the most common model of computation.
\begin{description}
    \item[Informal definition] A Turing machine is an automaton with a finite number of states and an infinite tape.
    Using a read/write head, which can read one symbol on the tape, modify one symbol on the tape and move left and right, a Turing Machine can compute functions
    \item[Formal definition] Formally, a Turing machine is a 7-tuple $M = \langle  Q, \Sigma, \Gamma, \delta, q_0, q_{accept}, q_{reject}\rangle$, where
    \begin{description}
        \item[$Q$] is the set of states
        \item[$\Sigma$] is the alphabet of the input word
        \item[$\Gamma$] is the set of symbols which can be written or read on the tape, which we call the tape alphabet
        \item[$\delta$] is the transition function, with $\delta : \Gamma \times Q \to \Gamma \times Q \times \{L, R\}$.
        This means that when a Turing machine is in state $n$ and reads $a$ on the tape, $\delta$ tells us to which state we should transition, which symbol we should write and which direction we should move the read/write head
        \item[$q_0$] the start state
        \item[$q_{accept}$] the accepting state
        \item[$q_{reject}$] the rejecting state
    \end{description}
    \item[Turing computation] In the beginning, the Turing machine is in the start state, the input word is written on a consecutive part of the tape and the read/write head is on the first character of the input word.
    In the following steps, the Turing machine state changes according to the transition function.
    If at some point the Turing machine enters the accepting or the rejecting state, the computation halts, and the Turing machine is said to have accepted / rejected the input.
    It can happen that the Turing machine continues indefinitely or loops.
    In that case, we also say that it has rejected its input.
    In this document, we will ignore the tape content after the computation and focus on decision problems.
    \item[Decidability] If a Turing machine halts on all inputs, we say that it decides a problem, as we can always be sure that the machine will accept or reject an input in finite time.
    \item[Nondeterministic Turing machine] We can extend the transition function $\delta$ to allow multiple transitions from a given state.
    Formally, we then have $\delta : \Gamma \times Q \to \mathcal{P}(\Gamma \times Q \times \{L, R\})$.
    If at some point there are multiple transitions that are possible from the current state, we can take any of them.
    If there exists any computational path which leads to an accepting state, the nondeterministic Turing machine accepts.
    This is not analogous to how real sequential computers work, but allows interesting results, and is as powerful as a normal deterministic Turing machine.
    \item[Space/Time-Constructible functions] A function $f(n)$ is time constructible if there exists a Turing machine which on input $1^{n}$ writes $f(n)$ in binary on its tape in time $f(n)$.
    Space-constructible functions are defined analogously.
    \item[Church-Turing Thesis] The Church-Turing Thesis states that anything that can be done on a real-world computer can be done using a Turing machine.
\end{description}
