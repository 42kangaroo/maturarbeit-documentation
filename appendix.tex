\chapter{Mathematical Background}\label{ch:mathematical-background}

The definitions are taken from the lectures Discrete Mathematics in Computer Science~\cite{discrete-maths} and Theory of Computer Science~\cite{theory-cs} and also from the book Descriptive Complexity~\cite{descriptive-complexity}.


\section{Set Theory}\label{sec:set-theory}
\begin{description}
    \item[Set] An unordered collection of distinct elements, written with curly braces $\{\}$
    \item[Tuple] An ordered collection of elements written with pointed braces $\langle  \rangle$
    \item[Set operations] There are multiple ways to form new sets from already existing sets:
    \begin{description}
        \item[Union] denoted as $\cup$, an element is in $A \cup B$ if and only if it is in $A$ or $B$
        \item[Intersection] denoted as $\cap$, an element is in $A \cap B$ if and only if it is in $A$ and $B$
        \item[Cartesian product] denoted as $\times$, $A \times B$ is the set of tuples with an element of $A$ and an element of $B$
        \item[Cartesian power] $A^k$ denotes the cartesian product of $A$ with itself repeated $k$ times
    \end{description}
\end{description}


\section{First Order Logic}\label{sec:first-order-logic}
We abbreviate first order logic as FO.
\begin{description}
    \item[Variable] A variable is an element that can have a value from a set.
    \item[Universe] The set over which variables and constants can range
    \item[Relation] A relation of arity $k$, $R(x_1, \dots, x_k)$ can be either true or false for any $k$-tuple of variables. In this work we always consider equality($=$), an ordering relation $\leq$, and $BIT(x, y)$, which means that the $y^{th}$ bit of $x$ is set in binary notation, to exist.
    \item[Vocabulary] A tuple $\tau = \langle R_1^{a_1}, \dots, R_r^{a_r}, c_1, \dots, c_s \rangle$ of relations $R_i$ with arity $a_i$ and constants $c_j$ (We omit functions as they can be simulated by a relation in our case)
    \item[Structure] A tuple $\mathcal{A} = \langle |\mathcal{A}|, R_1^{\mathcal{A}}, \dots, R_r^{\mathcal{A}}, c_1^{\mathcal{A}}, \dots, c_s^{\mathcal{A}} \rangle$ where $|\mathcal{A}|$ is the universe, the constants are assigned a value from $|\mathcal{A}|$ and the truth of the relations have a truth value for each $a_i$-tuple from $|\mathcal{A}|^{a_i}$
    \item[First Order Formula] A first order formula is inductively defined as follows:
    \begin{description}
        \item[Atoms] Any formula of the form $R(x_1, \dots, x_k)$ for some relation of arity $k$ is called an atomic formula
        \item[conjuction] If $\varphi$ and $\psi$ are formulas, $(\varphi \land \psi)$ is a formula
        \item[disjuction] If $\varphi$ and $\psi$ are formulas, $(\varphi \lor \psi)$ is a formula
        \item[negation] If $\varphi$ is a formula, $\lnot \varphi$ is a formula
        \item[Existencial Quantification] If $\varphi$ is a formula, $\exists x \varphi$ is a formula
        \item[Universal Quantification] If $\varphi$ is a formula, $\forall x \varphi$ is a formula
    \end{description}
    \item[Semantics] For any structure, we can assign a truth value to any formula (by assigning values from the universe to free variables if they exist in the formula). We say $\mathcal{A}$ satisfies $\phi$ (where $\phi$ is taken over the vocabulary of $\mathcal{A}$), denoted $\mathcal{A} \models \phi$ if and only if $\phi$ is true under the interpretation of the constant and relations of $\mathcal{A}$. This is inductively defined as follow:
    \begin{description}
        \item[Atoms] For a formula $\phi$ of the form $R(x_1, \dots, x_k)$, we have $\mathcal{A} \models \phi$ if and only if the interpretation of the relation maps $\langle x_1, \dots, x_k \rangle$ to true
        \item[conjuction] We have $\mathcal{A} \models (\varphi \land \psi)$ if and only if $\mathcal{A} \models \varphi$ and $\mathcal{A} \models \psi$
        \item[disjuction]  We have $\mathcal{A} \models (\varphi \lor \psi)$ if and only if $\mathcal{A} \models \varphi$ or $\mathcal{A} \models \psi$
        \item[negation] We have $\mathcal{A} \models \lnot \varphi$ if and only if $\mathcal{A} \not\models \varphi$
        \item[Existencial Quantification] We have $\mathcal{A} \models \exists x\varphi$ if and only if there exists a $y \in |\mathcal{A}|$ such that $\mathcal{A} \models \varphi(y)$ (where $\varphi(y)$ denotes $\varphi$ with any occurrence of $x$ replaced with the element $y$)
        \item[Universal Quantification] We have $\mathcal{A} \models \forall x\varphi$ if and only if for all $y \in |\mathcal{A}|$ we have $\mathcal{A} \models \varphi(y)$
    \end{description}
\end{description}


\section{Second Order Logic}\label{sec:second-order-logic}
In second order logic, we extend the capabilities of first order logic with the ability to quantify over relations. We thus also need to extend our definitions. We abreviate second order logic as SO.
\begin{description}
    \item[SO variables] A relation that is not given in the vocabulary and can be substituted with a specific interpretation
    \item[SO formula] In addition to the inductive rules from the FO formulas, we can quantify over second order formulas
    \begin{description}
        \item[SO Existencial Quantification] If $\varphi$ is a formula, then $\exists V\varphi$ is a formula
        \item[SO Universal Quantification] If $\varphi$ is a formula, then $\forall V\varphi$ is a formula
    \end{description}
    \item[SO Semantics] Here we also need to extend the FO semantics
    \begin{description}
        \item[SO Existencial Quantification]  We have $\mathcal{A} \models \exists V\varphi$ if and only if there exists a relation $U$ over $|\mathcal{A}|$ such that $\mathcal{A} \models \varphi(U)$ (where $\varphi(U)$ denotes $\varphi$ with any occurrence of $V$ replaced with $U$)
        \item[SO Universal Quantification] We have $\mathcal{A} \models \forall V\varphi$ if and only if for all relations $U$ over $|\mathcal{A}|$ we have $\mathcal{A} \models \varphi(U)$
    \end{description}
\end{description}


\section{Turing Machines}\label{sec:turing-machines}
Turing machines are the most common model of computation. We abreviate Turing Machines as TM
\begin{description}
    \item[Informal definition] A turing machine is a automaton with a finite number of states and an infinite tape. Using a read/write head, which can read one symbol on the tape, modify one symbol on the tape and move left and right, a Turing Machine can compute functions
    \item[Formal definition] Formally, a Turing machine is a 7-tuple $M = \langle  Q, \Sigma, \Gamma, \delta, q_0, q_{accept}, q_{reject}\rangle$, where
    \begin{description}
        \item[$Q$] is the set of states
        \item[$\Sigma$] is the set representing the symbols of which the input word on the tape can consist
        \item[$\Gamma$] is the set of symbols which can be written or read on the tape
        \item[$\delta$] is the transition function, with $\delta : \Gamma \times Q \to \Gamma \times Q \times \{L, R\}$. So when a TM is in state $n$ and reads $a$ on the tape, $\delta$ tells us to which state we should transition, which symbol we should write and which direction we should move the read/write head
        \item[$q_0$] the start state
        \item[$q_{accept}$] the accept state
        \item[$q_{reject}$] the reject state
    \end{description}
    \item[Turing computation] At the beginning, the TM is in the start state, the input is written in a consecutive way on the tape and the read/write head is on the first character of the input word. In consecutive steps, the machine state then changes according to the transition function. If at some point the machine enters the accept or the reject state, the computation halts, and the TM is said to have accepted / rejected the input. In this work we will ignore the tape content after the computation and focus on decision problems.
    \item[Decidability] If a TM halts on all inputs, we say that it decides a problem, as we can always be sure that the machine will accept or reject an input in finite time.
    \item[Nondetermenistic TM (NTM)] We can extend the transition function $\delta$ to allow multiple transitions from a given state. If there exists any computational path which leads to an accept state, the NTM accepts. This is not analog to how real sequential computers work, but allows interesting results, and is as powerfull as a normal deterministic TM.
    \item[Church-Turing Thesis] According to the Church-Turing Thesis, this formalism is equivalent to what any computer can compute.
\end{description}
