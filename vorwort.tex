\chapter*{Forword}

Some years ago, I started to get interested in informatics and programming by the Lego Mindstorms.
By attending some courses at the Ph\ae novum in Lörrach, I was able to learn how to program using Java, my first not block-based programming language.
At that time I was planning to work at Boston Dynamics in the future, as I loved being able to physically see what I had achieved.
But then came the RoboCup robot~\cite{roboCup}.
It was meant to be a rolling robot for playing football on a miniature playing field.
After two years, we wer still unable to follow the ball because the Corona pandemic prevented us from working on site together, but also, and to a greater extent, because the hardware never did what it was meant to, and we passed interminable hours just trying to make it roll forward.
As may have become apparent, I got tired of it and decided to move on to something which didn't include too much hardware and was more abstract.

So I decided to participate in the Swiss Olympiad in Informatics, which organises national programming contests and selections various teams for international olympiads.
Here, I get quite a lot of interesting problems which I love to solve, found like-minded friends and am able to participate at some international competitions.
Over the years, I began to notice that I have much more fun solving tasks theoretically then implementing them.
That is also something that gets reflected in my competition scores, where I often come out knowing I solved a lot in theory, but failed to get the points.
Some internships in informatics firms confirmed that actual programing is still too concrete.

Now let's get more abstract.
Thanks to the ``Schülerstudium'', I attended a course on the mathematical background of computer science~\cite{discrete-maths} and on the theory of computer science~\cite{theory-cs}.
There, I learned more about complexity theory, computability, logic and the Chomsky Hierarchy
The way in which proofs could be made to hold for every problem with certain properties has fascinated me ever since.
Further, logic is a tool which captures mathematical reasoning, and is can thus be seen as a formalisation of every ``logical'' thought we have.
Also, I don't have to bother with implementation anymore.

Using books I began to inform myself more, and found the domain of descriptive complexity.
This domain relates different kinds of logic to different classes of problems in computer science.
So I knew I wanted to do my Matura project in this domain, but had difficulties finding an open question which seamed approachable.
While asking a friend at the informatics olympiad, she asked \emph{ChatGPT} which told me I could study the connection between the chomsky hierarchy and descriptive complexity.
Refining this proposition a bit, I came up with (almost) the current question, relating logic and context-sensitive languages.
Then, I found out there exists a characterisation using second-order logic, so I added a ``first-order logic'' to the question.
