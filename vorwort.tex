%! suppress = UnresolvedReference
\chapter*{Foreword}

Some years ago, the Lego Mindstorms sparked my interest in informatics and programming.
By attending some courses at the Phænovum in Lörrach, I was able to learn how to program using Java, my first text-based programming language.
At that time, I was planning to work at Boston Dynamics, as I loved being able to physically see what I had achieved.
But then came the RoboCup robot~\cite{roboCup}.
It was meant to be a rolling robot for playing football on a miniature playing field.
After two years, we were still unable to follow the ball because the Corona pandemic prevented us from working on site together.
Also, and to a greater extent, it didn't work because the hardware never did what it was meant to, and we passed interminable hours just trying to make it roll forward.
As may have become apparent, I got tired of it and decided to move on to something which didn't include too much hardware and was more abstract.

So I decided to participate in the Swiss Olympiad in Informatics, which organizes national programming contests and selects various teams for international Olympiads.
There, I got and still get quite a lot of interesting problems which I love to solve, found like-minded friends and was able to participate in some international competitions.
Over the years, I began to notice that I enjoy solving tasks theoretically way more than implementing them.
That is also something that got reflected in my competition scores, where I often came out knowing I solved a lot in theory, but failed to get the points.
Some internships at informatics firms confirmed that actual programming is still too concrete for me.

Now let's get more abstract.
Thanks to the ``Schülerstudium'', I attended a course on the mathematical background of computer science~\cite{discrete-maths} and one on the theory of computer science~\cite{theory-cs}.
There, I learned more about Complexity theory, computability, logic and the Chomsky Hierarchy.
The way in which proofs could be made to hold for every problem with certain properties has fascinated me ever since.
Further, logic is a tool which captures mathematical reasoning, and can thus be seen as a formalization of every ``logical'' thought we have.
Also, I don't have to bother with implementation any more.

Using books, I began to inform myself more, and found out about the domain of Descriptive Complexity.
This domain relates different kinds of logic to different classes of problems in computer science.
So I knew I wanted to do my Matura project in this domain, but had difficulties finding an open question which seemed approachable.
I asked many people if they had a more exact idea what I could do.
In the end, a friend at the informatics Olympiad asked \emph{ChatGPT}, which told me I could study the connection between the Chomsky Hierarchy and Descriptive Complexity.
Refining this proposition a bit, I came up with (almost) the current question of relating logic and context-sensitive languages.
Then, I found out that a characterization using second-order logic already exists, so I added ``first-order logic'' to the question.
