%! suppress = UnresolvedReference
%! suppress = MissingImport
\chapter{Personal Contribution}\label{ch:personal-contribution}

In this chapter, results of searches for new restrictions of first-order logic and other related concepts are presented.
I did not find any fundamentally new result about the characterization of context-sensitive languages using first-order logic but managed to prove that various approaches could not lead to an equivalence.
Further, I also lowered some upper bounds for simulating alternating Turing machines using iterative logic.
Using a restricted form of iterative logic, I could lower the bounds for an iterative simulation of a nondeterministic Turing Machine.


\section{Direct Transformation}\label{sec:direct-transformation}

Similar to Immerman in~\cite{descriptive-complexity}, we use extended variables to model second-order variables in first-order logic.
This then directly gives us a characterization of \acs{NSPACE}$[s(n)]$ without requiring any new insights, as we can just use the same technique as for second-order logic.

\begin{define}[Extended Variable]
    The logic \acs{FO-VAR}$[1, s(n)]$ has two types of variables.
    Domain variables are the first type, which we denote by lowercase characters, ranging from $0$ to $n - 1$.
    Extended variables are the other type, which we denote by uppercase characters, ranging from $0$ to $2^{s(n)\log(n)} - 1$, and thus having $s(n)\log(n)$ bits.
    The extended variables are not allowed to appear as an input to any relation apart from \acs{BIT}\@.
    Thus, we can only query if a specific bit in the binary representation of an extended variable is on.
    For extended variables with more than $n$ bits, we can extend \acs{BIT} to accept a tuple of domain variables encoding the position we want to query.
    This makes extended variables with a polynomial number of bits possible.
\end{define}

The extended variables in \acs{FO-VAR}$[1, n^k / \log(n)]$ have exactly the same capabilities as second-order variables of arity $k$ in second-order logic.
This is true as for every second-order variable $X$ we can have an extended variable $X'$ such that bit $\overline{x}$ of $X'$ is set if and only if $X(\overline{x})$ is true.
By generalizing the proof of \cref{thm:contextsensitveMSOTC}, we have \acs{NSPACE}$[n^k] = $ \acs{SO}(\acs{TC}, arity $k$).
We now prove that \acs{SO}(\acs{TC}, arity $k$) = \acs{FO-VAR}$[1, n^k/\log(n)]$(\acs{TC}), thus also capturing \acs{NSPACE}$[n^k]$ with an extension of first-order logic.

\begin{proof}
    We show by induction on the structures of the formulas that each formula in \acs{SO}(\acs{TC}, arity $k$) has an equivalent formula in \acs{FO-VAR}$[1, n^k/\log(n)]$(\acs{TC}) and vice versa.

    Any atomic formula in \acs{SO}(\acs{TC}, arity $k$) which does not include any second-order variable is trivially writable in \acs{FO-VAR}$[1, n^k/\log(n)]$(\acs{TC}).
    The same holds for formulas in \acs{FO-VAR}$[1, n^k/\log(n)]$(\acs{TC}) without extended variables.
    An atomic formula with a second-order variable $Y(\overline{x})$ can be represented as querying the bit $\overline{x}$ of an extended variable $Y'$.
    We can do the same transformation backwards for an atomic formula of the form \acs{BIT}$(Y', \overline{x})$ and get a formula of the form $Y(\overline{x})$ with a second-order variable $Y$.

    Using this, we can induct on the structure of the formulas.
    Taking the conjunction, disjunction, or negation of equivalent formulas generates new formulas which are also equivalent.
    When a formula quantifies over a second-order or extended variable, we can directly exchange this with a quantification over the respective other type of variable.
    By induction, the sub-formulas without the quantification are equivalent.
    Binding the new second-order or extended variable makes the new formulas equivalent.

    A formula that takes the transitive closure can also be reformulated by replacing all variables of the current type with variables of the other type.
    Again using induction on the sub-formulas, we get that the new formulas are equivalent.

    By induction, we then have that every formula in either logic has an equivalent formula in the other logic.
\end{proof}

Using this equivalence and the proof in \cref{subsec:des-context-sensitive-languages}, we get that \acs{FO-VAR}$[1, n/\log(n)]$(\acs{TC}) describes exactly the context-sensitive languages.
This gives us our first characterization of context-sensitive languages in first-order logic.


\section{Analogues to Proof for \acs{DSPACE}}\label{sec:analogues-to-proof-for-dspace}

For \acs{DSPACE}$[s(n)]$, there are multiple other logical characterizations which do not use the transitive closure operator.

First, the logics \acs{FO}$[t(n)]$, which formalizes iterative definitions, and \acs{VAR}$[k]$, which restricts the number of variables but allows for unbounded first-order iterations, are defined.

\begin{define}[{\acs{FO}$[t(n)]$}]
    Let $Q_1, \dots, Q_n$ be a series of quantifiers, $s_1, \dots, s_n$ a series of variables, and $M_1 \dots, M_n$ a series of quantifier-free formulas.
    A quantifier block has the form \[\acs{QB} = (Q_{1}s_{1}.M_{n})\dots(Q_{n}s_{n}.M_{n})\]
    For a universal quantifier, $(\forall s.M)\varphi \equiv \forall s (M \to \varphi)$.
    For an existential quantifier, $(\exists s.M)\varphi \equiv \exists s(M \land \varphi)$.
    Both of these equivalences effectively mean that the quantifiers are restricted to range only over the elements that satisfy $M$.
    Then, a formula of \acs{FO}$[t(n)]$ is of the form
    \[
        \left([\acs{QB}]^{t(n)}M_{0}\right)(\overline{c} / \overline{s})
    \]
    where $M_0$ is a quantifier-free formula, $\overline{c} = \langle c_1, \dots, c_n \rangle$ is a tuple of constants, and $\overline{s} = \langle s_1, \dots, s_n \rangle$ are the variables occurring in the quantifier block.
    The notation $(\overline{c} / \overline{s})$ means that in the beginning, we define $s_1 \coloneqq c_1, \dots, s_n \coloneqq c_n$.
    $[\acs{QB}]^{t(n)}$ stands for $\acs{QB}$ literally repeated $t(n)$ times.
    The truth values of these formulas for a specific structure are defined by evaluating the formula obtained by iterating the quantifier block $t(\lVert \mathcal{A}\rVert)$ times.
\end{define}
This gives us a formalism for iterative procedures.

If we restrict the number of variables such that there are at most $k$ distinct variables, but allow some additional boolean variables\footnote{variables which have only two possible values}, we get \acs{FO-VAR}$[t(n), k]$.
In particular, we can reuse the same variable multiple times in the same quantifier block.
Additionally, we define
\[
    \text{\acs{VAR}}[k] = \bigcup_{c = 1}^{\infty}\text{\acs{FO-VAR}}[2^{cn^k}, k]
\]
This is the same as saying that we allow unbounded iterations, as after at most $2^{cn^k}$ iterations, the truth value of the formula will loop or stay the same.
The constant $c$ depends only on the number of boolean variables in the formula.

For \acs{DSPACE}, we have
\[
    \text{\acs{DSPACE}}[n^k] = \text{\acs{VAR}}[k + 1]
\]
A proof of this can be found in~\cite{descriptive-complexity}.
In the main part of the proof, a construction is made to simulate a \acs{DSPACE}$[n^k]$ Turing machine using \acs{VAR}$[k + 1]$.
There, the relation $C_{t}(\overline{x}, \overline{b})$ is inductively defined to mean that at time $t$, the character on the tape at position $\overline{x}$ is the one encoded by $\overline{b}$, where $\overline{b}$ is a tuple of boolean variables.
The characters are defined in the same way as in the proof of \cref{thm:contextsensitveMSOTC} to include the machine state as well as the tape symbol.
As the Turing machines under consideration are deterministic, the character at position $\overline{x}$ is uniquely determined.
Further, it only depends on the characters at positions $\overline{x} - 1, \overline{x}$, and $\overline{x} + 1$ at time $t - 1$.
Because it can be assumed without loss of generality that the Turing machine returns to the starting tape cell and clears its tape after accepting, finding out the value of $C_{2^{cn^k}}(\overline{0}, \overline{b})$ is sufficient to determine if the machine accepts.

If we want to extend this to nondeterministic Turing machines using $C_{t}(\overline{x}, \overline{b})$ to mean that the character at position $\overline{x}$ \emph{can} be $\overline{b}$ at time $t$, we run into problems.
As we do not have the guarantee that the computation is deterministic, we can not say any more that the new state only depend on the possible characters on the tape at time $t - 1$.
Doing this would mean that impossible states could be reached.
This would happen because the formula does not remember which combinations of characters are possible, which is required as the characters of each tape cell can not be chosen independently.
One way to fix this is by remembering the nondeterministic choices made in the previous steps.
This would make the values deterministic again, fixing the issue.
In the worst case, this would take $\mathcal{O}(2^{cn^k})$ additional bits.
But we already know by Savitch's theorem (\cref{subsec:nspacesubsetdspacesquared}) and the equivalence for \acs{DSPACE} that we only need $\log(n) \cdot (k + 2)$ bits to represent a \acs{NSPACE}$[n^k]$ computation.
This is much less than the $2^{cn^k}$ bits needed using this approach.

Let \acs{VAR}$[r(n)]$ be the generalization of \acs{VAR}$[k]$ which means that a total of $r(n) + \mathcal{O}(1)$ bits of variables are allowed.
Any proof which would show that \acs{NSPACE}$[n^k]$ can be expressed in \acs{VAR}$[r(n)]$ with $\log(n) \cdot (k + 1) \leq r(n) < \log(n) \cdot (k + 2)$ would be an improvement on Savitch's theorem.
That is because the proof for \acs{DSPACE}$[n^k]$ can easily be extended to more general polynomial functions, as Immerman showed in~\cite{Immerman1999}.
If Savitch's theorem can be improved, we can also describe \acs{NSPACE} with less than $k + 2$ variables and unbounded iterations.


\section{Mixing Iterations and Transitive Closure}\label{sec:mixing-iterations-and-transitive-closure}

\sloppy One question that came up during my research was: Can we find a logic that is more restricted than $\text{\acs{FO-VAR}}[s(n), s(n)/\log(n)]$ but still captures \acs{NSPACE}$[s(n)]$?
This led to the idea of mixing up iterative procedures with the transitive closure operator.
We now show that a success of this approach would mean that there exists a restriction using only iterative procedures that also contains \acs{NSPACE}$[s(n)]$.
As formulas mixing up multiple operators and procedures are more difficult to analyse, it makes this approach unpractical.

There are two ways of combining the transitive closure with iterated formulas.
The first is surrounding a formula in \acs{FO-VAR}$[t(n), r(n)/\log(n)]$ with a \acs{TC} operator using variables of size $r(n)$.
For this to contain \acs{NSPACE}$[s(n)]$, for any formula of the form $(\text{\acs{TC}}_{a, b}\varphi)(c, d)$ we need a quantifier block, a formula $\phi$ and some constants $\overline{e}, f, g$ such that
\[
    \left(\text{\acs{TC}}_{a, b}\left([\acs{QB}]^{t(n)}\phi(\overline{e}/\overline{s})\right)\right)(f, g) \equiv (\text{\acs{TC}}_{a, b}\varphi)(c, d)
\]
We know that we can simulate any \acs{TC} formula with variables of size $r(n)$ using $r(n)$ iterations of a quantifier block.
This is done by the technique of halving paths used in Savitch's theorem.
An exact formula achieving this is presented in \cref{sec:restricting-universal-quantification}.
Like this, we find an equivalent formula:
\[
    \left(\text{\acs{TC}}_{a, b}\left([\acs{QB}]^{t(n)}\phi(\overline{e}/\overline{s})\right)\right)(f, g) \equiv [\acs{QB}_1]^{r(n)}[\acs{QB}_2]^{t(n)}\phi(\overline{e}/\overline{s}, f/a, g/b)
\]
The two quantifier blocks can then be merged into one by additionally maintaining a counter variable of size $\log(r(n))$ such that the quantifier block acts like $[\acs{QB}_1]$ during the first $r(n)$ round and then acts like $[\acs{QB}_2]$.

The other case in which we write the \acs{TC} formula after the quantifier block can be treated similarly.
\[
    \left([\acs{QB}]^{t(n)}\left(\text{\acs{TC}}_{a, b}\phi\right)(f, g)\right)(\overline{e}/\overline{s}) \equiv (\text{\acs{TC}}_{a, b}\varphi)(c, d)
\]
If the condition $c \cdot r(n) > \log(t(n))$ is satisfied for some fixed $c$, we can again merge the quantifier blocks by using the counting trick.
As any $t(n) > 2^{cr(n)}$ can be seen as having an unbounded amount of iterations, we can assume without loss of generality that the condition holds.

In both cases, we have a formula in \acs{FO-VAR}$[r(n) + t(n), r(n)/\log(n)]$.
Now, there are multiple cases
\begin{description}
    \setlength\itemsep{0.15em}
    \item[$r(n) \leq t(n)$:]  Then, \acs{FO-VAR}$[r(n) + t(n), r(n)/\log(n)] =$ \acs{FO-VAR}$[t(n), r(n)/\log(n)]$.
    This is only interesting in the case where $r(n) < s(n)$
    \item[$r(n) > t(n)$:] Then, \acs{FO-VAR}$[r(n) + t(n), r(n)/\log(n)] =$ \acs{FO-VAR}$[r(n), r(n)/\log(n)]$.
    This is again only interesting when $r(n) < s(n)$.
\end{description}

We thus get that any improvement on our simulation results happens only when we have less than $s(n)$ variable bits.
Using Savitch's theorem once more, we also get that $r(n) > \sqrt{n}$.
Another result is that if $t(n)r(n) < s(n)^2$, Savitch's theorem can be improved as
\[
    \text{\acs{FO-VAR}}[t(n), r(n)/\log(n)] \subseteq \text{\acs{DSPACE}}[t(n)r(n)]
\]
This same method also tells us that $t(n)r(n) \geq s(n)$ because otherwise, we could simulate a \acs{NSPACE}$[s(n)]$ Turing machine deterministically using less than $s(n)$ space, which is a contradiction to the space hierarchy theorem described in \cref{subsec:space-hierarchy-theorem}.


\section{Restricting Universal Quantification}\label{sec:restricting-universal-quantification}

Another approach that I attempted was restricting universal quantification in the \acs{FO-VAR} formulas.
I called the resulting logic \acs{FOEx-VAR}\@.
In \acs{FOEx-VAR}, universal quantification is only allowed over boolean variables.
We denote boolean variables by $b_j$.
Further, the requirement of $M_i$ to be quantifier-free is dropped.
This step was motivated by the fact that nondeterministic Turing machines in essence capture existential quantification.

We use a similar idea to the one used in Savitch's theorem: guessing the middle of a path and checking if both shorter paths are connected.
A formula in \acs{FOEx-VAR}$[s(n), s(n)/\log(n)]$ which is equivalent to a formula in \acs{FO-VAR}$[1, s(n)/\log(n)]$(\acs{TC}) of the form $\left( TC_{\overline{X}, \overline{Y}}\varphi \right)(\overline{C}, \overline{D})$ is
\[
    \begin{aligned}
        \acs{QB} \equiv~& (\forall b_{1}.M_{1})(\exists\overline{Z}) (\forall b_{2})(\exists \overline{A}, \overline{B}.M_{2})(\exists\overline{X}, \overline{Y}.M_{3}) \\
        M_{1} \equiv~& \neg(\forall \overline{z} (\text{\acs{BIT}}(\overline{X}, \overline{z}) \leftrightarrow \text{\acs{BIT}}(\overline{Y}, \overline{z})) \lor \varphi(\overline{X}, \overline{Y})) \\
        M_{2} \equiv~&(b_{2} \land \forall \overline{z} (\text{\acs{BIT}}(\overline{X}, \overline{z}) \leftrightarrow \text{\acs{BIT}}(\overline{A}, \overline{z}))\land \forall \overline{z} (\text{\acs{BIT}}(\overline{B}, \overline{z}) \leftrightarrow \text{\acs{BIT}}(\overline{Z}, \overline{z}))) \lor \\
        &(\neg b_{2} \land \forall \overline{z} (\text{\acs{BIT}}(\overline{Z}, \overline{z}) \leftrightarrow \text{\acs{BIT}}(\overline{A}, \overline{z}))\land \forall \overline{z} (\text{\acs{BIT}}(\overline{B}, \overline{z}) \leftrightarrow \text{\acs{BIT}}(\overline{Y}, \overline{z}))) \\
        M_{3} \equiv~&\forall \overline{z} (\text{\acs{BIT}}(\overline{X}, \overline{z}) \leftrightarrow \text{\acs{BIT}}(\overline{A}, \overline{z}))\land \forall \overline{z} (\text{\acs{BIT}}(\overline{B}, \overline{z}) \leftrightarrow \text{\acs{BIT}}(\overline{Y}, \overline{z}))
    \end{aligned}
\]
and finally
\[
    [\acs{QB}]^{cs(n)}(false)(\overline{C} / \overline{X}, \overline{D} / \overline{Y})
\]
for some constant $c$.

In $\acs{QB}$, the first quantification and $M_1$ mean that the formula breaks with the actual branch being true whenever $X$ and $Y$ are connected.
When this happens, quantification is done universally over the empty set, which is defined as true.
If no connection exists yet, a middle configuration $\overline{Z}$ is guessed, and both sides are checked by universally choosing $b_2$.
In $M_2$, the formula then checks that $\overline{A}$ and $\overline{B}$ are the endpoints of the path determined by $b_2, \overline{X}, \overline{Z} \text{ and }\overline{Y}$.
After this, $\overline{A}$ and $\overline{B}$ are copied to $\overline{Y}$ and $\overline{X}$ in $M_3$.
When iterating this, a path from $\overline{C}$ to $\overline{D}$ is guessed, and every connection between adjacent states is checked.

We define \acs{FOExUn-VAR}$[t(n), f(n), s(n)]$ to contain all formulas with $t(n)$ iterations of a quantifier block, extended variables of $f(n)$ bits which can be universally quantified, and extended variables of $s(n)\log(n)$ bits which can be used in existential quantification.
Using this, we can make the more general statement that for any function $r(n) \leq 2^{cs(n)}$ for some constant $c$, we can define a formula in \[\text{\acs{FOExUn-VAR}$[s(n)/\log(r(n)), \log(r(n)), s(n)r(n)/\log(n)]$}\] which simulates a nondeterministic Turing machine using $s(n)$ space.
These formulas are very similar to the one defined for \acs{FOEx-VAR}\@.
The only difference lies in how the paths are split.
Instead of splitting the paths into two parts, they are split into $r(n)$ parts.
This is done by guessing $r(n) - 1$ middle configurations at once in an existential extended variable and choosing which part to check using the universal extended variable.
Thus, for a path of length $2^{cs(n)}$, the formula needs at most \[\log_{r(n)}\left(2^{cs(n)}\right) = \frac{\log\left(2^{cs(n)}\right)}{\log(r(n))} = cs(n)/\log(r(n))\] iterations.
As for simulating this formula with a Turing machine, the product of the number of iterations and the size of the extended variables are important, we do not gain any tighter results from this.


\section{Alternating Bounds}\label{sec:alternating-bounds}

In the proof of Savitch's theorem, alternating Turing machines were used as an intermediate step for proving that \acs{NSPACE}$[s(n)]$ is a subset of \acs{DSPACE}$[s(n)^2]$.
We can also add another intermediate step using \acs{FO-VAR}$[s(n), s(n)/\log(n)]$ with the method shown in \cref{sec:restricting-universal-quantification}.
It is also quite easy to simulate any formula in \acs{FO-VAR}$[t(n), s(n)/\log(n)]$ with an alternating Turing machine in \acs{ATSR}$[t(n)s(n), s(n), t(n)]$, where in this new class \acs{ATSR}, we specify time, space and reversals.
By reversals, we mean the number of times we switch from universal states to existential states and back.
A Turing machine in \acs{ATSR}$[t(n)s(n), s(n), t(n)]$ does this by writing the current values of all variables on the tape while iterating and evaluating the quantifier-free first-order formulas on the way, which can be done efficiently in terms of space.

We now investigate two ways of simulating an \acs{ATSR}$[t(n), s(n), r(n)]$ Turing machine using \acs{FO-VAR}\@.

The first one is a straightforward simulation of each step of the computation.
For this, our formula encodes the whole state of the Turing machines in a tuple of $s(n)$ bit extended variables and iteratively guesses the next configuration.
We also assume without loss of generality that a predicate $existential(\overline{X})$ exists and that there exists exactly one accepting state, denoted by $\overline{E}$.
\[
    \begin{aligned}
        \acs{QB} \equiv~&(\forall b.M_{1})(\exists\overline{A}.\varphi(\overline{X}, \overline{A}))(\forall \overline{B}.M_{2})(\exists \overline{X}. \forall \overline{z}(\text{\acs{BIT}}(\overline{X}, \overline{z}) \leftrightarrow \text{\acs{BIT}}(\overline{B}, \overline{z}))) \\
        M_{1} \equiv~&\neg\forall \overline{z}(\text{\acs{BIT}}(\overline{X}, \overline{z}) \leftrightarrow \text{\acs{BIT}}(\overline{E}, \overline{z})) \\
        M_{2} \equiv~& (existential(\overline{X}) \to \forall \overline{z}(\text{\acs{BIT}}(\overline{A}, \overline{z}) \leftrightarrow \text{\acs{BIT}}(\overline{B}, \overline{z}))) \land \varphi(\overline{X}, \overline{B})
    \end{aligned}
\]
with \[
         \left([\acs{QB}]^{t(n)}(false)\right)(\overline{S} / \overline{X})
\]
This gives us a formula in \acs{FO-VAR}$[t(n), s(n)/\log(n)]$ which simulates a computation of an alternating Turing machine.
Some of these $M_i$ are not quantifier-free.
The quantifiers could be moved out of the $M_i$, but would make the formula less readable.
This formula works as follows:
The formula $\acs{QB}$ starts by checking if the accepting configuration has already been reached using $M_1$.
Next, if the simulation is not done yet, $\acs{QB}$ existentially guesses the next configuration, restricting this quantification to only those configurations which are directly connected to the actual one.
In $M_2$, either this existentially guessed next state is retained if the simulated Turing machine is in an existential state (thus effectively doing nothing) or the existentially quantified state is ignored and a universal quantification over all states which are connected to the current one is done.
The last thing $\acs{QB}$ does is copying $\overline{B}$ to $\overline{X}$, making it the new start of the path.

By combining the above techniques for switching between universal and existential quantification with the path-halving method used in Savitch's theorem, we get an even stronger result.

From now on, we refer to configurations as vertices to emphasize that we are searching for paths on a graph.
On a high level, our plan is to do the following:
\begin{enumerate}
    \setlength\itemsep{0.15em}
    \item From the actual starting vertex, find all reachable vertices using only vertices which have the same type (existential or universal) as the starting vertex.
    \begin{enumerate}
        \item For existential states, we can use the normal technique of halving paths naively.
        \item For universal states, we need to quantify over all ending vertices and then consider two cases:
        \begin{itemize}
            \item We claim that there is no path going to this vertex that uses only universal states.
            Then, we need to show that for each middle vertex, one of the two new paths still does not exist.
            \item We claim that a path exists.
            In this case, we can try to find a path normally.
        \end{itemize}
    \end{enumerate}
    \item Repeat with all the reached vertices.
\end{enumerate}

Before presenting the formula, we first think of how many iterations are needed.
Because of the halving trick,  $\log(a)$ iterations are used for a path of length $a$.
By summing over all path segments with only one quantification type, we can bound the total number of iterations.
As we are simulating an alternating Turing machine which makes at most $t(n)$ steps, the total of all path segments can not exceed $t(n)$.
To get an upper bound on this, we can use Jensen's inequality (\cite{inequalities-math-oly}, p.28).
We get
\[
    \begin{aligned}
        \frac{\sum_{j = 0}^{r(n)}\log(a_{j})}{r(n)} &\leq \log\left(\frac{\sum_{j= 0}^{r(n)}a_{j}}{r(n)} \right)  \\
        &\leq \log\left(\frac{t(n)}{r(n)} \right)
    \end{aligned}
\]
because of the condition that $\sum_{j= 0}^{r(n)}a_{j} \leq t(n)$.
Multiplying this with $r(n)$, we get that a path can require a maximum of $r(n)\log\left(\frac{t(n)}{r(n)} \right)$ iterations.
Thus, the formula is in ${\text{\acs{FO-VAR}}\left[r(n)\log\left(t(n)/r(n) \right), s(n)/\log(n) \right]}$.


The following formula simulates an \acs{ATSR}$[t(n), s(n), r(n)]$ Turing machine:
\[
    \begin{aligned}
        \acs{QB} \equiv&~(\forall b_{0}.M_{1})(\exists b_{0}.N_{1})(\exists b_{1}.M_{2})(\exists Z_{e}.M_{3})(\forall Z.N_{2})\\
        &~(\exists b_{2e})(\forall b_{2}.N_{3})(\exists A, B.M_{4})(\forall S, E.M_{5})(\exists A, B.M_{6})(\exists I, X.M_{7})(\exists b_{0}.N_{4})(\exists b_{path}.N_{5}) \\
        M_{1} \equiv&~b_{path} \to \neg(I = Y \land (S = E \lor \varphi(S, E))) \\
        N_{1} \equiv&~\neg b_{path} \to \neg(S = E \lor \varphi(S, E)) \\
        M_{2} \equiv&~\neg b_{1} \leftrightarrow (S = E \lor \varphi(S, E)) \\
        M_{3} \equiv&~existential(Z_{e}) \leftrightarrow existential(X)\\
        N_{2} \equiv&~(b_{path} \to Z = Z_{e}) \land (existential(Z) \leftrightarrow existential(X))\\
        N_{3} \equiv&~ \neg b_{path} \to b_{2} = b_{2e}\\
        M_{4} \equiv&~(b_{1} \land ((\neg b_{2} \land A = S  \land B = Z) \lor (b_{2} \land A = Z \land B = E))) \lor \\
        &~(\neg b_{1} \land  A = I \land (B = Y \lor (existential(B) \leftrightarrow \neg existential(I)))) \\
        M_{5} \equiv&~S = A \land ((( b_{1} \lor existential(S)) \land E = B) \lor \\
        &~(\neg( b_{1} \lor existential(S)) \land (E = Y \lor (existential(E)\leftrightarrow \neg existential(S))))) \\
        M_{6} \equiv&~(\neg b_{1}  \land S = A \land E = B) \lor (b_{1} \land X = A \land I = B)\\
        M_{7} \equiv&~X = A \land I = B \\
        N_{4} \equiv&~b_{0} = b_{path} \\
        N_{5} \equiv&~(b_{1} \land b_{path} = b_{0}) \lor (\neg b_{1} \land (existential(X) \to b_{path}))
    \end{aligned}
\]
and $\left([\acs{QB}]^{r(n)\log(t(n)/r(n))}(\neg b_{path})\right)(\text{true} / b_{path}, AS / S, AS / X, AS /I, AS /S, AS / E, AE / Y)$ being the final formula.

The meanings of the variables are explained below:
\begin{description}
    \setlength\itemsep{0.15em}
    \item[$Y$:] the accepting vertex
    \item[$X$:] the starting vertex in the actual component
    \item[$I$:] the vertex for which whether a path from $X$ exists is being checked
    \item[$S$:] the start of the path segment which is currently checked
    \item[$E$:] the end of the path segment  which is currently checked
    \item[$Z, Z_{e}$:] the guessed middle vertex of a path segment from $S$ to $E$
    \item[$A, B, b_{0}$:] variables used for copying or as dummies
    \item[$b_{path}$:] whether the formula claims that there is a path or that there is no path from $S$ to $E$
    \item[$b_{1}$:] remembers whether $S$ and $E$ are connected
    \item[$b_{2}, b_{2e}$:] which half of the $S \to Z \to E$ path is currently checked
\end{description}
As this formula is quite intricate, it is proposed to read it four times, each time focusing on a different stage of the search.
\begin{enumerate}
    \item First, we look at the formula in the easiest case: It is trying to find a path from $X$ to $I$ using $S$ and $E$, which are not directly connected.
    In this case, both $b_{path}$ and $b_1$ are true and $\acs{QB}$ works like this:
    In $M_1$, the formula checks and sees that it is not done, as $S$ and $E$ are not connected.
    As $b_{path}$ is true, nothing is done in $N_1$.
    In $M_2$, $b_1$ is set to true.
    In $M_3$, a middle vertex between $S$ and $E$ is guessed which has the same type as $X$.
    $N_2$ copies $Z_e$ into $Z$.
    Next, a side of the path is chosen.
    As $b_{path}$ is true, in $N_3$ the choice made for $b_{2e}$ is ignored and the side is universally chosen.
    $M_4$ copies the relevant new starting and ending vertices to $A$ and $B$.
    Because $b_1$ is true, $A$ and $B$ are copied to $S$ and $E$ in $M_5$.
    In $M_6$ and $M_7$ combined  $X$ and $I$ are copied to $A$ and $B$ and back again.
    The same is done with $b_{path}$ using $N_4$, $N_5$ and $b_0$.
    In the end, the formula gets a new configuration with the guessed middle vertex replacing either the former ending or starting vertex.
    If no path exists, this continues until in the formula $\neg b_{path}$ and this branch is marked as false.

    \item The second case we look at is the case where the formula claims that no path exists.
    Here, $b_{path}$ is false.
    Because of this, in $M_1$ nothing is done.
    In $N_1$, the formula checks if it found a connection, and if it did, it sets this logical branch to false because of the existential quantification over the empty set.
    If $N_1$ is true, the formula goes on to $M_2$, where it establishes $b_1$.
    As $M_3$ is ignored in $N_2$, it has no effect.
    In $N_2$, a universal quantification over all middle vertices $Z$ which could lie on a path between $S$ and $E$ is done.
    The next two quantifiers quantify existentially which half of $S \to Z \to E$ is not connected.
    This half must exist, as otherwise, a path from $S$ to $E$ exists, which would contradict $\neg b_{path}$.
    Because $b_1$ is again true, the relevant part of the path is copied to $A$ and $B$ in $M_4$ and back to $S$ and $E$ in $M_5$.
    In $M_6$, $M_7$, $N_4$ and $N_5$ the formula does the same as in the first case, which was not changing $X$, $I$ and $b_{path}$.
    If no path exists, this continues until the formula $\neg b_{path}$ is hit.
    At this point, this branch of the formula is marked as true, as effectively, no path from $X$ to $I$ exists by making these choices.

    \item The third possibility we need to consider is that the formula is currently trying to find a path from $X$ to $I$ in a universal component and that it succeed in finding a connection between $S$ and $E$.
    In this case, $b_{path}$ is true and $b_1$ is false.
    In $M_1$, if $I = Y$, this logical branch is done as the simulation reached the accepting vertex.
    Otherwise, $N_1$ can be disregarded, and in $M_2$, $b_1$ is set to false.
    The quantification for $Z$ and $b_2$ can be ignored as these variables are not used in this iteration of the formula.
    In $M_4$, $A$ and $B$ are guessed such that $A$ is the new starting vertex corresponding to the old ending vertex $I$, and $B$ is either the accepting vertex $E$ or a universal vertex.
    Then, $A$ is copied to $S$ in $M_5$, marking it as the starting vertex.
    Further, because the quantification type changed, a valid $S$ must now be existential.
    Thus, $B$ is also copied to $E$ directly.
    After that, in $M_6$ and $M_7$, the formula copies $S$ to $X$ and $E$ to $I$ via $A$ and $B$.
    The last change happens in $N_{5}$, where $b_{path}$ is set to true.
    Thus, the variables are set for a new block of existential states.

    \item The last case is similar to the third one, with the difference that the current vertex $X$ is existential.
    In this case, $b_{path}$ is true and $b_1$ is false.
    In $M_1$, if $I = Y$, this logical branch is done as the simulation reached the accepting vertex.
    Otherwise, $N_1$ can be disregarded, and in $M_2$, $b_1$ is set to false.
    The quantification for $Z$ and $b_2$ can be ignored as these variables are not used in this iteration of the formula.
    In $M_4$, $A$ and $B$ are guessed such that $A$ is the new starting vertex corresponding to the old ending vertex $I$, and $B$ is either the accepting vertex $E$ or an existential vertex.
    Then, $A$ is copied to $S$ in $M_5$, marking it as the starting vertex.
    Further, because we changed the quantification type, a valid $S$ must now be universal.
    Thus, all vertices which are either the accepting vertex or are existential are universally quantified.
    After that, in $M_6$ and $M_7$, the formula copies $S$ to $X$ and $E$ to $I$ via $A$ and $B$.
    The last change happens in $N_{5}$, where the formula quantifies existentially if it claims that a path from $X$ to $I$ exists or that no such path does.
    Thus, the variables are set for a new block of  universal states.
\end{enumerate}
The complete formula sets the starting values of the variables.
By setting $b_{path} = \text{true}$ and all of $X, S, E$ and $I$ to the staring configuration $AS$, either case three or four is triggered.
This is what we want as after this, the component including $AS$ will be searched.
The accepting vertex $Y$ is set to the desired ending $AE$.

These results allow us to enclose an \acs{ATSR} class between two \acs{FO-VAR} classes with only logarithmic overhead.
Thus:
\[
    \text{\acs{FO-VAR}}\left[a(n), s(n)/\log n \right] \subseteq \text{\acs{ATSR}}[a(n)s(n), s(n), a(n)] \subseteq \text{\acs{FO-VAR}}\left[a(n)\log(s(n)), s(n)/\log n \right]
\]