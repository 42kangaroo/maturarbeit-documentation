\chapter{Communications}\label{sec:communications}


\section{Email Dr.~Dennis Komm}\label{email-dr.-dennis-komm}

Sehr geehrter Dr.~Komm,

Ich heisse Yaël Arn und bin Gymnasiast am Gymnasium Bäumlihof in Basel.
Auf Empfehlung meiner Maturarbeitsbetreuungslehrperson schreibe ich
Ihnen.

In meiner Maturarbeit beschäftige ich mich mit Finite Model Theory,
insbesondere im Bezug auf deren Verbindung mit Formal Languages. Somit
will ich insbesondere untersuchen, ob ich eine korrespondierende Logik
zu den Context-Sensitive Languages finde, da ich keinen Artikel gefunden
habe, der eine solche erwähnt.

Die Grundlagen der Theoretischen Informatik konnte ich mir im Rahmen des
Schülerstudiums an der Uni Basel aneignen. Auch interessiere ich mich
stark für Logik, Complexity Theory und Finite Model Theory und habe
darüber Bücher gelesen.

Damit Sie meine Fähigkeiten einschätzen können: Auch bei der Schweizer
Informatikolympiade, der Schweizer Mathematikolympiade sowie bei
mehreren internationalen Olympiaden (MEMO, CEOI, BOI, WEOI) habe ich
teilgenommen.

Jetzt zur eigentlichen Frage: Haben sie Tipps, wie ich vorgehen sollte
und in welche Richtung ich forschen sollte? Und halten sie mein
Unterfangen überhaupt für möglich?

Vielen Dank im Voraus für Ihre Zeit und Hilfe.

Viele Grüsse,

Yaël Arn


\section{Response Dr.~Dennis Komm}\label{response-dr.-dennis-komm}

Lieber Herr Arn

Bitte nehmen Sie es mir nicht übel, dass ich Ihnen noch nicht
geantwortet habe.~ Ich habe Ihre Mail einem sehr guten Kollegen
weitergeleitet, der sich sicherlich bei Ihnen melden wird.~ Momentan
fehlen mir leider die zeitlichen Ressourcen, Ihnen persönlich
detaillierter zu schreiben, was ich sehr bedauere.

Liebe Grüsse

Dennis Komm

\section{Email Dr.~Gabriel
Röger}\label{email-dr.-gabriel-ruxf6ger}

Sehr geehrte Dr.~Röger,

Ich heisse Yaël Arn (Nickname 42kangaroo, falls Sie sich daran erinnern)
und bin Gymnasiast am Gymnasium Bäumlihof in Basel. Da ich durch Ihre
Vorlesung ``Theory of computer science'' auf mein Maturarbeitsthema
gestossen bin schreibe ich Ihnen jetzt.

In meiner Maturarbeit beschäftige ich mich mit Finite Model Theory,
insbesondere in Bezug auf deren Verbindung mit Formal Languages. Somit
will ich insbesondere untersuchen, ob ich eine korrespondierende Logik
zu den Context-Sensitive Languages finde, da ich keinen Artikel gefunden
habe, der eine solche erwähnt.

Neben Ihren Vorlesungen habe ich auch durch diverse Bücher über Logik,
Complexity Theory und Finite Model Theory Einblicke in das Thema gehabt.

Damit Sie meine Fähigkeiten einschätzen können: Auch bei der Schweizer
Informatikolympiade, der Schweizer Mathematikolympiade sowie bei
mehreren internationalen Olympiaden (MEMO, CEOI, BOI, WEOI) habe ich
teilgenommen.

Jetzt zur eigentlichen Frage: Haben sie Tipps, wie ich vorgehen sollte
und in welche Richtung ich forschen sollte? Und halten sie mein
Unterfangen überhaupt für möglich?

Vielen Dank im Voraus für Ihre Zeit und Hilfe.

Viele Grüsse,

Yaël Arn

\section{Email Dr.~Angelika
Steger}\label{email-dr.-angelika-steger}

Sehr geehrte Dr.~Steger,

Ich heisse Yaël Arn und bin Gymnasiast am Gymnasium Bäumlihof in Basel.
Auf Empfehlung von Charlotte Knierim schreibe ich Ihnen.

In meiner Maturarbeit beschäftige ich mich mit Finite Model Theory,
insbesondere im Bezug auf deren Verbindung mit Formal Languages. Somit
will ich insbesondere untersuchen, ob ich eine korrespondierende Logik
zu den Context-Sensitive Languages finde, da ich keinen Artikel gefunden
habe, der eine solche erwähnt.

Die Grundlagen der Theoretischen Informatik konnte ich mir im Rahmen des
Schülerstudiums an der Uni Basel aneignen. Auch interessiere ich mich
stark für Logik, Complexity Theory und Finite Model Theory und habe
darüber Bücher gelesen.

Damit Sie meine Fähigkeiten einschätzen können: Auch bei der Schweizer
Informatikolympiade, der Schweizer Mathematikolympiade sowie bei
mehreren internationalen Olympiaden (MEMO, CEOI, BOI, WEOI) habe ich
teilgenommen.

Jetzt zur eigentlichen Frage: Haben sie Tipps, wie ich vorgehen sollte
und in welche Richtung ich forschen sollte? Und halten sie mein
Unterfangen überhaupt für möglich?

Ein volles Mentoring suche ich aktuell nicht, eine Antwort in der Form
einer Mail, eines Telefonates oder eines kurzen Treffens würde mich aber
freuen.

Vielen Dank im Voraus für Ihre Zeit und Hilfe.

Viele Grüsse,

Yaël Arn

\section{Response
Dr.~Královi\v{c}}\label{response-dr.-kruxe1loviux10d}

Dear Mr.~Arn,

Dr.~Komm asked me to write you some tips about your planned thesis; I am
working in his group as senior~scientific assistant. I hope it is ok if
I write in English.

As you are likely aware, context-sensitive languages are exactly the
languages that can be recognized by nondeterministic~Turing machines in
linear space. This class is often denoted as NSPACE(n) =
nondeterministic space O(n).

I am not really an expert in the field of descriptive complexity theory,
but I am not aware of logic-based characterization of this class. There
are, however, some related results that are known. For example,
characterization of DSPACE(n\^{}k) is known (see
e.g.~\url{https://people.cs.umass.edu/~immerman/descriptiveComplexity.html}),
where DSPACE(f(n)) denotes the class of languages accepted by
deterministic Turing machines using space O(f(n)). NSPACE(n) is
trivially a superset of DSPACE(n), and it is also a subset of
DSPACE(n\^{}2) (due to Savitch's theorem). This immediately gives two
logic-based characterizations that bound NSPACE(n) from below and from
above. This observation is already mentioned in
\url{https://people.cs.umass.edu/~immerman/pub/ams_notices.pdf}~as
Theorem 2.

To get an exact logic-based characterization of NSPACE(n) may be not
easy,~most likely,~it~is a very ambitious goal. You could try to adapt
the results for DSPACE(n\^{}k) to NSPACE; even if you do not succeed
there, explaining the proof ideas and showing why they are not so easy
to transfer to NSPACE would make a valuable thesis. Augmenting such
results with explanations of the upper and lower bounds of NSPACE(n)
would, in my opinion, form a very solid ``Maturaarbeit''.

If you have more questions, feel free to ask. I wish you good luck in
writing your thesis and I hope you will also enjoy it.

Best regards,

~ ~ Richard Kralovic

\section{Further Communication with
Dr.~Královi\v{c}}\label{further-communication-with-dr.-kruxe1loviux10d}

Dear Dr.~Královi\v{c},

First of all, thanks a lot for your response and your time!

English is ok for me, in fact I'm writing my Matura project in English.

I effectively wasn't aware of the characterisation of DSPACE(n\^{}k), so
this is very valuable advice into which I will look. I had found upper /
lower bounds using other formal languages, notably
\url{https://link.springer.com/chapter/10.1007/BFb0022257} (existential
SO restricted to binary matching predicates) for context-free languages
and FO($\exists$N) for enumerable languages (As described in the book
``Descriptive Complexity'' by Neil Immermann), but they had the
disadvantage of being less tight.

If (or rather when) I have more questions, I will write you another
e-mail, thanks again for giving me this possibility!

Best,\\
Yaël Arn

\section{Question for writing MA in
English}\label{question-for-writing-ma-in-english}


Sehr geehrter Herr Ziegler, sehr geehrter Herr Leuthardt,

\sloppy Gerne würde ich meine Maturarbeit über ``Theoretische Informatik und
Logig'' auf Englisch schreiben, da die mathematische Literatur zum
Grossteil auf Englisch ist und ich Mathematik lieber in dieser Sprache
löse. Ist dies möglich?

Vielen Dank und viele Grüsse,

Yaël Arn \\


\noindent Sehr geehrter Herr Ziegler, sehr geehrter Herr Leuthardt,

Da ich gerne bald mit der schriftlichen Arbeit anfangen würde, wäre ich
froh um eine Rückmeldung.

Viele Grüsse,

Yaël Arn \\


\noindent Lieber Yael

Die Schulleitung hat Ihren Antrag genehmigt. Sie dürfen die Maturaarbeit
auf Englisch schreiben.

@Aline Steiner: Auflage der Schulleitung: Koreferent*in muss über
entsprechende Englischkenntnisse verfügen, um die Arbeit berwerten zu
können.

Liebe Grüsse

Andreas Leuthardt
