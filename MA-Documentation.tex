\documentclass[a4paper,11pt]{report}
% !BIB TS-program = biber
% !BIB program = biber
% Die beiden obigen Zeilen helfen TeXShop auf Mac mit biber
% (Verarbeitung der Bibliographiedatenbank in literatur.bib)


% ======== START VON PAKETE LADEN =========
\usepackage[english]{babel}        % Deutschsprachige Beschriftungen
\usepackage[utf8]{inputenc}       % Utf8 Zeichensatz
\usepackage[T1]{fontenc}          % Schriftenkodierung
\usepackage[lighttt]{lmodern}     % Schriftart
\usepackage{amsmath}              % Mathematische Formeln
\usepackage[normalem]{ulem}       % Durchgestrichener Text
\usepackage{xcolor}               % Farbiger Text
\usepackage{verbatim}             % Text ohne Formatierung
\usepackage{listings}             % Code mit Formatierung
\usepackage{csquotes}             % Kontextsensitive Zitatanlage
\usepackage{caption}              % Erweiterte Beschriftungen
\usepackage{subcaption}           % Unterbeschriftungen
\usepackage{geometry}             % Seitenränder
\usepackage{setspace}             % Zeilenabstand
\usepackage{fancyhdr}             % Header und Footer
\usepackage{graphicx}             % Grafiken
\usepackage{wrapfig}
\usepackage{svg}                  % SVG Grafiken
\usepackage{booktabs}             % Tabellen
\usepackage{tabularx}             % Breite von Boxen
\usepackage[style=alphabetic]{biblatex} % Referenzen
\usepackage{hyperref}             % Hyperlinks im PDF-Dokument
\usepackage{hyperxmp}             % Metadaten im PDF-Dokument
\usepackage{makeidx}              % Optional, für Glossar (Index)
\usepackage[version=4]{mhchem}    % Für chemische Formeln
\usepackage{siunitx}              % Angaben mit Masseinheiten
\makeindex
% ========= ENDE VON PAKETE LADEN =========


% ==== START VON DOKUMENTEIGENSCHAFTEN ==== (Dokument)
\title % Titel festlegen
{Theoretical Informatics: Formal languages and finite model thoery}
\date{\today}   % Datum festlegen
\author{Yaël Arn}  % Autor festlegen
\makeatletter            % Erlaubt das Auslesen der obigen Eigenschaften
\let\papertitle\@title% Titel in \papertitle speichern
\let\paperdate\@date     % Datum in \paperdate speichern
\let\paperauthor\@author % Autor in \paperauthor speichern
\makeatother
\def\papersubtitle
{A study of the connection of first order logic and context-sensitive languages}
\def\paperinstitution    % Bildungseinrichtung in \paperinstitution speichern
{Gymnasium Bäumlihof}
\def\papertype           % Art dieser Arbeit in \papertype speichern
{Maturaarbeit}
\def\papersupervisor     % Betreuungsperson in \papersupervisor speichern
{Aline Sprunger}
\def\papercoreferent{Bernhard Pfammatter}
\def\paperauthorclass
{4A}
\def\paperplace{4058 Basel}
% Achtung: Schlüsselwörter sind weiter unten definiert!
% ==== ENDE VON DOKUMENTEIGENSCHAFTEN =====


% =========================================


% === START VON SCHRIFTPROBLEME BEHEBEN === (Schriftart)
% Die Schreibmaschinen-Schrift lädt nur den normalen Stil, nicht den fetten oder den kursiven
\ttfamily
\DeclareFontShape{T1}{lmtt}{m}{it}{<->sub*lmtt/m/sl}{}
% === ENDE VON SCHRIFTPROBLEME BEHEBEN ====

% Unterstützung von Sonderzeichen
\lstset{literate=
  {á}{{\'a}}1 {é}{{\'e}}1 {í}{{\'i}}1 {ó}{{\'o}}1 {ú}{{\'u}}1 {Á}{{\'A}}1 {É}{{\'E}}1 
  {Í}{{\'I}}1 {Ó}{{\'O}}1 {Ú}{{\'U}}1 {à}{{\`a}}1 {è}{{\`e}}1 {ì}{{\`i}}1 {ò}{{\`o}}1
  {ù}{{\`u}}1 {À}{{\`A}}1 {È}{{\'E}}1 {Ì}{{\`I}}1 {Ò}{{\`O}}1 {Ù}{{\`U}}1 {ä}{{\"a}}1
  {ë}{{\"e}}1 {ï}{{\"i}}1 {ö}{{\"o}}1 {ü}{{\"u}}1 {Ä}{{\"A}}1 {Ë}{{\"E}}1 {Ï}{{\"I}}1
  {Ö}{{\"O}}1 {Ü}{{\"U}}1 {â}{{\^a}}1 {ê}{{\^e}}1 {î}{{\^i}}1 {ô}{{\^o}}1 {û}{{\^u}}1
  {Â}{{\^A}}1 {Ê}{{\^E}}1 {Î}{{\^I}}1 {Ô}{{\^O}}1 {Û}{{\^U}}1 {œ}{{\oe}}1 {Œ}{{\OE}}1
  {æ}{{\ae}}1 {Æ}{{\AE}}1 {ß}{{\ss}}1 {ű}{{\H{u}}}1 {Ű}{{\H{U}}}1 {ő}{{\H{o}}}1 {Ő}{{\H{O}}}1
  {ç}{{\c c}}1 {Ç}{{\c C}}1 {ø}{{\o}}1 {å}{{\r a}}1 {Å}{{\r A}}1 {€}{{\euro}}1 {£}{{\pounds}}1 {«}{{\guillemotleft}}1 {»}{{\guillemotright}}1 {ñ}{{\~n}}1 {Ñ}{{\~N}}1 {¿}{{?`}}1
}


% ======== START VON SEITENRÄNDER ========= (Seitenränder)
\geometry{
  a4paper,
  total={150mm,237mm},
  left=20mm,
  top=30mm,
  right=20mm,
  bottom=30mm
}
\setlength{\headheight}{14pt}
% ========= ENDE VON SEITENRÄNDER =========



% ====== START VON QUELLBIBLIOTHEKEN ====== (Referenzen)
\addbibresource{literatur.bib}
% ====== ENDE VON QUELLBIBLIOTHEKEN =======



% ========= START VON METADATEN ===========
\hypersetup{
pdftoolbar=true,           % Toolbar anzeigen?
pdfmenubar=true,           % Menüleiste anzeigen?
pdffitwindow=false,        % Fenstergrösse anpassen?
pdfstartview={FitH},       % Breite dem Fenster anpassen
pdftitle={\papertitle},    % Titel
pdfauthor={\paperauthor},  % Autor
pdfsubject={\papertype},   % Thema
pdfcreator={pdfLaTeX},     % PDF-Ersteller
pdfproducer={\paperauthor},% Dokument-Ersteller
pdfkeywords={\paperinstitution} {\papersupervisor} % Schlüsselwörter
    {LaTex} {Maturaarbeit} {Finite Model Theory} {Formal Languages},
pdfnewwindow=true,         % Links in neuem Fenster?
colorlinks=false,          % Farbige Links: true, Link-Boxen: false
linkcolor=red,             % Farbe interner Links
citecolor=green,           % Farbe der Zitat-Links
filecolor=magenta,         % Farbe der Datei-Links
urlcolor=cyan              % Farbe externer Links
bookmarks=true             % Lesezeichen erstellen?
bookmarksdepth=section     % Lesezeichen bis zu Abschnitten
bookmarksopen=false        % Lesezeichen öffnen?
bookmarksnumbered=true     % Kapitelnummern in Lesezeichen?
}
\pdfinfo{
/Title  (\papertitle)
/Author (\paperauthor)
/Subject (\papertype)
/Keywords (\paperinstitution;\papersupervisor;LaTex;Maturaarbeit;Finite Model Theory;Formal Languages)
}
% ========== ENDE VON METADATEN ===========



% ========== START VON BILDPFADE ==========
\graphicspath{{img/pdf/}{img/png/}}
\svgpath{{img/svg/}}
% =========== ENDE VON BILDPFADE ==========



% =============================================================
\begin{document} % START VOM DOKUMENT =========================
% =============================================================

% === TITELSEITE ===
% Titelseite (https://de.wikibooks.org/wiki/LaTeX/_Eine_Titelseite_erstellen)
\begin{titlepage}
  \centering
  {\scshape\paperinstitution\par}
  \vspace{1cm}
  {\scshape\Large\papertype\par}
  \vspace{1.5cm}
  {\huge\bfseries\papertitle\par}
  \vspace{.5cm}
  {\LARGE\bfseries\papersubtitle\par}
  \vspace{2cm}
  {\Large Written by:\par\paperauthor, \paperauthorclass}
  \vfill
  Supervisor:\par
  {\sc\papersupervisor\par}
  \vspace{.5cm}
  Second Examiner:\par
  {\sc\papercoreferent\par}
  \vspace{1cm}
  {\large\paperdate, \paperplace\par}
\end{titlepage}
 %     \
% ------------------

% === INHALTSVERZEICHNIS ===
\tableofcontents %          \
\newpage %                  \
% --------------------------



% =========== START VON STYLING ===========
\onehalfspacing % Zeilenabstand 1.5
\renewcommand{\chaptermark}[1]{\markboth{\MakeUppercase{\thechapter.\ #1}}{}} % Kapitel
\renewcommand{\sectionmark}[1]{\markright{\thesection.\ #1}{}} % Abschnitt
\fancyhead[R]{\rightmark}
\fancyhead[L]{\leftmark}
\fancyhead[C]{\thepage{}} % Seitennummer
\fancyhead[L]{\leftmark}
\fancyhead[R]{\rightmark}
\fancyfoot{}
\pagestyle{fancy}
% =========== ENDE VON STYLING ============

% === KAPITEL DER ARBEIT ===
%! suppress = UnresolvedReference
\chapter*{Foreword}

Some years ago, the Lego Mindstorms sparkled my interest in informatics and programming.
By attending some courses at the Ph\ae novum in Lörrach, I was able to learn how to program using Java, my first text-based programming language.
At that time, I was planning to work at Boston Dynamics, as I loved being able to physically see what I had achieved.
But then came the RoboCup robot~\cite{roboCup}.
It was meant to be a rolling robot for playing football on a miniature playing field.
After two years, we were still unable to follow the ball because the Corona pandemic prevented us from working on site together.
Also, and to a greater extent, it didn't work because the hardware never did what it was meant to, and we passed interminable hours just trying to make it roll forward.
As may have become apparent, I got tired of it and decided to move on to something which didn't include too much hardware and was more abstract.

So I decided to participate in the Swiss Olympiad in Informatics, which organizes national programming contests and selects various teams for international Olympiads.
There, I got and still get quite a lot of interesting problems which I love to solve, found like-minded friends and was able to participate at some international competitions.
Over the years, I began to notice that I enjoy solving tasks theoretically way more than implementing them.
That is also something that got reflected in my competition scores, where I often came out knowing I solved a lot in theory, but failed to get the points.
Some internships at informatics firms confirmed that actual programming is still too concrete for me.

Now let's get more abstract.
Thanks to the ``Schülerstudium'', I attended a course on the mathematical background of computer science~\cite{discrete-maths} and one on the theory of computer science~\cite{theory-cs}.
There, I learned more about Complexity Theory, computability, logic and the Chomsky Hierarchy.
The way in which proofs could be made to hold for every problem with certain properties has fascinated me ever since.
Further, logic is a tool which captures mathematical reasoning, and can thus be seen as a formalization of every ``logical'' thought we have.
Also, I don't have to bother with implementation any more.

Using books, I began to inform myself more, and found out about the domain of Descriptive Complexity.
This domain relates different kinds of logic to different classes of problems in computer science.
So I knew I wanted to do my Matura project in this domain, but had difficulties finding an open question which seamed approachable.
I asked multiple people if they had a more exact idea what I could do.
In the end, a friend at the informatics Olympiad asked \emph{ChatGPT}, which told me I could study the connection between the Chomsky Hierarchy and Descriptive Complexity.
Refining this proposition a bit, I came up with (almost) the current question of relating logic and context-sensitive languages.
Then, I found out there exists a characterization using second-order logic, so I added ``first-order logic'' to the question.

%! suppress = UnresolvedReference
%! suppress = MissingImport
\chapter{Introduction}\label{ch:intro}

In our daily lives, we are in contact with various kinds of algorithms at all times.
Searching on Google, sending texts, asking questions to AI chatbots, searching for the fastest route with a GPS; all of these consume resources in energy, storage space and time.
According to a report from 2021, 3.7 \% of global carbon emissions come from the IT domain, with an upward tendency.
This is similar to that of the airline industry~\cite{webFootprint}.
At this scale, it is thus vitally important to understand and find out if the resource consumption can be reduced.

To find out how we can improve, we first need to understand how computers work, what we can compute, and why solving some problems is more difficult than solving others.
The field of study that investigates this is called Complexity Theory.
This is done by abstraction of computational models and of the problems themselves.
It is then often possible to find classes of similar problems which allow for generalizations.
However, it has proven to be very hard to find proofs of either optimality of algorithms or the separations of complexity classes.
One of the emblematic open problems is the P versus NP question, which we will go further into in~\cref{subsubsec:pnp}.
Nevertheless, some significant results concerning the complexity of problems on average in the real world and most importantly in cryptography have been made.

One of the subfields of Complexity Theory is Descriptive Complexity.
It relates mathematical logic to different complexity classes.
This allows us to get new insights into the underlying structure of problems of certain classes.
The other main field present in this work, formal languages, is concerned with the abstraction of computational problems as sets, which is often helpful for proofs.
This then allows us to define multiple formalisms which describe these sets.
Our formal language class of focus, context-sensitive languages, is part of the ``Chomsky Hierarchy'' introduced by the famous linguist Noam Chomsky.

Towards the end of the 20$^{\text{th}}$ century, many equivalences were proven between fragments and extensions of various logics to complexity classes, including all classes in the Chomsky hierarchy.
A great summary of all the results is found in Neil Immermans paper ``Languages that capture complexity classes''~\cite{Immerman1987}.

In this work, we will first introduce the relevant theory.
After this, a personal study of connections between first-order logic and context-sensitive languages is presented.

Both \cref{ch:formal-languages} and \cref{ch:descriptive-complexity} present the most important theory of formal languages and Descriptive Complexity, respectively.
Additionally, in \cref{sec:results-concerning-the-chomsky-hierarchy} full proofs concerning equivalence of context-sensitive languages with logics and automata are presented.
These proofs are written in a way trying to show the motivation behind certain crucial steps.
Further, some details have been omitted for the sake of the simplicity and length of this work.
For the same purpose, many proofs, explanation, and examples for other classes in the Chomsky hierarchy were moved to \cref{ch:mathematical-context-and-further-proofs}.

After this introduction to the material, we go on in \cref{ch:personal-contribution} with my personal studies about connections between first-order logic and context-sensitive language.
In this process, different paths that I tried are presented and investigated.
Towards the end of the chapter, the direct connection to context-sensitive languages fades and the objects of study becomes more related to Savitch's Theorem, which plays a great role in the theory of space-bounded computation.
The most complicated proof is given in \cref{sec:alternating-bounds} and concerns a simulation of alternating Turing machines using iterative logic.
This divergence from the original working question was due to the hope that researching other, more loosely connected problems would ultimately help finding a solution for the core problem.
Also, I tried to pursue ideas I had instead of trying paths where I did not make any progress.

Finally, in \cref{ch:conclusion-and-direction} we reflect on what was accomplished in this work.
A review about the working process is presented, and finally we discuss further possibilities of research.

For the relevant mathematical background, \cref{ch:mathematical-background} contains the basic definitions of Set Theory, first and second-order logic, and Turing machines.

A full collection of description, proofs, techniques, and context information about Descriptive Complexity and languages in the Chomsky Hierarchy is given in \cref{ch:mathematical-context-and-further-proofs}.

The entire work is written in English because all mathematical literature is in that language.       \
\chapter*{Dank}
Ich darf hier wieder <<ich>> schreiben, und ich möchte Luca Bosin ganz herzlich
dafür danken, dass ich diese Vorlage auf Basis seiner Maturaarbeit anpassen
durfte und dabei gleich noch einige Neuerungen entdecken durfte, insbesondere
die verbesserte Literaturverzeichniserstellung mit biber.



% --------------------------

% === Optional: INDEX ====
\cleardoublepage  %      \  
\phantomsection %        \
\addcontentsline{toc}{chapter}{\indexname} 
\printindex %            \
% ------------------------


% === ABBILDUNGSVERZEICHNIS ===
\cleardoublepage %             \
\phantomsection %              \
\addcontentsline{toc}{chapter}{\listfigurename}
\listoffigures %               \
% -----------------------------

% === CODEBLOCKVERZEICHNIS ===
\cleardoublepage %            \
\phantomsection %             \
\addcontentsline{toc}{chapter}{\lstlistlistingname}
\lstlistoflistings %          \
% ----------------------------

% === LITERATURVERZEICHNIS ===
\cleardoublepage %            \
\phantomsection %             \
\addcontentsline{toc}{chapter}{\bibname}
\printbibliography %          \
% ----------------------------



% =========== START VON ANHÄNGE ===========
\appendix

% === EIGENSTÄNDIGKEITSERKLÄRUNG ===
\chapter{Independence declaration (German)}\label{ch:appendix_independencedeclaration}
Ich, Yaël Arn, 4A \\ \\
bestätige mit meiner Unterschrift, dass die eingereichte Arbeit
selbstständig und ohne unerlaubte Hilfe Dritter verfasst wurde.
Die Auseinandersetzung mit dem Thema erfolgte ausschliesslich
durch meine persönliche Arbeit und Recherche. Es wurden
keine unerlaubten Hilfsmittel benutzt.
Ich bestätige, dass ich sämtliche verwendeten Quellen sowie
Informanten/-innen im Quellenverzeichnis bzw. an anderer dafür
vorgesehener Stelle vollständig aufgeführt habe. Alle Zitate
und Paraphrasen (indirekte Zitate) wurden gekennzeichnet und
belegt. Sofern ich Informationen von einem KI-System wie
bspw. ChatGPT verwendet habe, habe ich diese in meiner Maturaarbeit
gemäss den Vorgaben im Leitfaden zur Maturaarbeit
korrekt als solche gekennzeichnet, einschliesslich der Art und
Weise, wie und mit welchen Fragen die KI verwendet wurde.
Ich bestätige, dass das ausgedruckte Exemplar der Maturaarbeit
identisch mit der digitalen Version ist.
Ich bin mir bewusst, dass die ganze Arbeit oder Teile davon
mittels geeigneter Software zur Erkennung von Plagiaten oder
KI-Textstellen einer Kontrolle unterzogen werden können.

\vspace{3cm}

\noindent
\begin{tabular}{p{0.47\linewidth}p{0.47\linewidth}}
  Ort \&\ Datum & Unterschrift \\
  & \\[1cm]
  \hline
\end{tabular}

% ----------------------------------

% =========== ENDE VON ANHÄNGE ============

% =============================================================
\end{document} % ENDE VOM DOKUMENT ============================
% =============================================================
