\documentclass[a4paper,11pt]{report}
% !BIB TS-program = biber
% !BIB program = biber
% Die beiden obigen Zeilen helfen TeXShop auf Mac mit biber
% (Verarbeitung der Bibliographiedatenbank in literatur.bib)


% ======== START VON PAKETE LADEN =========
\usepackage[british]{babel}        % Deutschsprachige Beschriftungen
\usepackage[utf8]{inputenc}       % Utf8 Zeichensatz
\usepackage[T1]{fontenc}          % Schriftenkodierung
\usepackage[lighttt]{lmodern}     % Schriftart
\usepackage{amsmath}
\usepackage{amsthm}     % Mathematische Formeln
\usepackage{amssymb}
\usepackage{mathtools}
\usepackage[normalem]{ulem}       % Durchgestrichener Text
\usepackage{xcolor}               % Farbiger Text
\usepackage{verbatim}             % Text ohne Formatierung
\usepackage{listings}             % Code mit Formatierung
\usepackage{csquotes}             % Kontextsensitive Zitatanlage
\usepackage{caption}              % Erweiterte Beschriftungen
\usepackage{subcaption}           % Unterbeschriftungen
\usepackage{geometry}             % Seitenränder
\usepackage{setspace}             % Zeilenabstand
\usepackage{fancyhdr}             % Header und Footer
\usepackage{graphicx}             % Grafiken
\usepackage{wrapfig}
\usepackage{svg}                  % SVG Grafiken
\usepackage{booktabs}             % Tabellen
\usepackage{tabularx}             % Breite von Boxen
\usepackage[style=alphabetic]{biblatex} % Referenzen
\usepackage{hyperref}             % Hyperlinks im PDF-Dokument
\usepackage{hyperxmp}             % Metadaten im PDF-Dokument
\usepackage{makeidx}              % Optional, für Glossar (Index)
\usepackage[version=4]{mhchem}    % Für chemische Formeln
\usepackage{siunitx}
\usepackage{stackengine}
\usepackage{cleveref}
% ========= ENDE VON PAKETE LADEN =========


% ==== START VON DOKUMENTEIGENSCHAFTEN ==== (Dokument)
\title % Titel festlegen
{Theoretical Informatics: Formal languages and finite model theory}
\date{\today}   % Datum festlegen
\author{Yaël Arn}  % Autor festlegen
\makeatletter            % Erlaubt das Auslesen der obigen Eigenschaften
\let\papertitle\@title% Titel in \papertitle speichern
\let\paperdate\@date     % Datum in \paperdate speichern
\let\paperauthor\@author % Autor in \paperauthor speichern
\makeatother
\def\papersubtitle
{A study of the connection of first order logic and context-sensitive languages}
\def\paperinstitution    % Bildungseinrichtung in \paperinstitution speichern
{Gymnasium Bäumlihof}
\def\papertype           % Art dieser Arbeit in \papertype speichern
{Maturaarbeit}
\def\papersupervisor     % Betreuungsperson in \papersupervisor speichern
{Aline Sprunger}
\def\papercoreferent{Bernhard Pfammatter}
\def\paperauthorclass
{4A}
\def\paperplace{4058 Basel}
% Achtung: Schlüsselwörter sind weiter unten definiert!
% ==== ENDE VON DOKUMENTEIGENSCHAFTEN =====


% =========================================


% === START VON SCHRIFTPROBLEME BEHEBEN === (Schriftart)
% Die Schreibmaschinen-Schrift lädt nur den normalen Stil, nicht den fetten oder den kursiven
\ttfamily
\DeclareFontShape{T1}{lmtt}{m}{it}{<->sub*lmtt/m/sl}{}
% === ENDE VON SCHRIFTPROBLEME BEHEBEN ====

% Unterstützung von Sonderzeichen
\lstset{literate=
{á}{{\'a}}1 {é}{{\'e}}1 {í}{{\'i}}1 {ó}{{\'o}}1 {ú}{{\'u}}1 {Á}{{\'A}}1 {É}{{\'E}}1
{Í}{{\'I}}1 {Ó}{{\'O}}1 {Ú}{{\'U}}1 {à}{{\`a}}1 {è}{{\`e}}1 {ì}{{\`i}}1 {ò}{{\`o}}1
{ù}{{\`u}}1 {À}{{\`A}}1 {È}{{\'E}}1 {Ì}{{\`I}}1 {Ò}{{\`O}}1 {Ù}{{\`U}}1 {ä}{{\"a}}1
{ë}{{\"e}}1 {ï}{{\"i}}1 {ö}{{\"o}}1 {ü}{{\"u}}1 {Ä}{{\"A}}1 {Ë}{{\"E}}1 {Ï}{{\"I}}1
{Ö}{{\"O}}1 {Ü}{{\"U}}1 {â}{{\^a}}1 {ê}{{\^e}}1 {î}{{\^i}}1 {ô}{{\^o}}1 {û}{{\^u}}1
{Â}{{\^A}}1 {Ê}{{\^E}}1 {Î}{{\^I}}1 {Ô}{{\^O}}1 {Û}{{\^U}}1 {œ}{{\oe}}1 {Œ}{{\OE}}1
{æ}{{\ae}}1 {Æ}{{\AE}}1 {ß}{{\ss}}1 {ű}{{\H{u}}}1 {Ű}{{\H{U}}}1 {ő}{{\H{o}}}1 {Ő}{{\H{O}}}1
{ç}{{\c c}}1 {Ç}{{\c C}}1 {ø}{{\o}}1 {å}{{\r a}}1 {Å}{{\r A}}1 {€}{{\euro}}1 {£}{{\pounds}}1 {«}{{\guillemotleft}}1 {»}{{\guillemotright}}1 {ñ}{{\~n}}1 {Ñ}{{\~N}}1 {¿}{{?`}}1
}


% ======== START VON SEITENRÄNDER ========= (Seitenränder)
\geometry{
a4paper,
total={150mm,237mm},
left=20mm,
top=30mm,
right=20mm,
bottom=30mm
}
\setlength{\headheight}{14pt}
% ========= ENDE VON SEITENRÄNDER =========


% ====== START VON QUELLBIBLIOTHEKEN ====== (Referenzen)
\addbibresource{literatur.bib}
% ====== ENDE VON QUELLBIBLIOTHEKEN =======


% ========= START VON METADATEN ===========
\hypersetup{
pdftoolbar=true,           % Toolbar anzeigen?
pdfmenubar=true,           % Menüleiste anzeigen?
pdffitwindow=false,        % Fenstergrösse anpassen?
pdfstartview={FitH},       % Breite dem Fenster anpassen
pdftitle={\papertitle},    % Titel
pdfauthor={\paperauthor},  % Autor
pdfsubject={\papertype},   % Thema
pdfcreator={pdfLaTeX},     % PDF-Ersteller
pdfproducer={\paperauthor},% Dokument-Ersteller
pdfkeywords={\paperinstitution} {\papersupervisor} % Schlüsselwörter
{LaTex} {Maturaarbeit} {Finite Model Theory} {Formal Languages},
pdfnewwindow=true,         % Links in neuem Fenster?
colorlinks=false,          % Farbige Links: true, Link-Boxen: false
linkcolor=red,             % Farbe interner Links
citecolor=green,           % Farbe der Zitat-Links
filecolor=magenta,         % Farbe der Datei-Links
urlcolor=cyan              % Farbe externer Links
bookmarks=true             % Lesezeichen erstellen?
bookmarksdepth=section     % Lesezeichen bis zu Abschnitten
bookmarksopen=false        % Lesezeichen öffnen?
bookmarksnumbered=true     % Kapitelnummern in Lesezeichen?
}
\pdfinfo{
/Title  (\papertitle)
/Author (\paperauthor)
/Subject (\papertype)
/Keywords (\paperinstitution;\papersupervisor;LaTex;Maturaarbeit;Finite Model Theory;Formal Languages)
}
\makeatletter
\def\@makechapterhead#1{%
    \vspace*{20\p@}%
    {\parindent \z@ \raggedright \normalfont
    \ifnum \c@secnumdepth >\m@ne
        \Huge\bfseries \thechapter.\space%
    \fi
    \interlinepenalty\@M
    \Huge \bfseries #1\par\nobreak
    \vskip 30\p@
}}
\makeatother
\newtheorem{theorem}{Theorem}[chapter]
\newtheorem{corollary}{Corollary}[theorem]
\newtheorem{lemma}[theorem]{Lemma}
\theoremstyle{definition}
\newtheorem{define}{Definition}[chapter]
\theoremstyle{definition}
\newtheorem{exmp}{Example}[chapter]
\addto\captionsbritish{
\renewcommand{\contentsname}{Table of Contents}
}
% ========== ENDE VON METADATEN ===========


% ========== START VON BILDPFADE ==========
\graphicspath{{img/pdf/}{img/png/}}
\svgpath{{img/svg/}}
% =========== ENDE VON BILDPFADE ==========


% =============================================================
\begin{document} % START VOM DOKUMENT =========================
% =============================================================

% === TITELSEITE ===
% Titelseite (https://de.wikibooks.org/wiki/LaTeX/_Eine_Titelseite_erstellen)
\begin{titlepage}
  \centering
  {\scshape\paperinstitution\par}
  \vspace{1cm}
  {\scshape\Large\papertype\par}
  \vspace{1.5cm}
  {\huge\bfseries\papertitle\par}
  \vspace{.5cm}
  {\LARGE\bfseries\papersubtitle\par}
  \vspace{2cm}
  {\Large Written by:\par\paperauthor, \paperauthorclass}
  \vfill
  Supervisor:\par
  {\sc\papersupervisor\par}
  \vspace{.5cm}
  Second Examiner:\par
  {\sc\papercoreferent\par}
  \vspace{1cm}
  {\large\paperdate, \paperplace\par}
\end{titlepage}
 %     \
% ------------------

\onehalfspacing % Zeilenabstand 1.5

\pagenumbering{gobble}
%! suppress = UnresolvedReference
\chapter*{Foreword}

Some years ago, the Lego Mindstorms sparkled my interest in informatics and programming.
By attending some courses at the Ph\ae novum in Lörrach, I was able to learn how to program using Java, my first text-based programming language.
At that time, I was planning to work at Boston Dynamics, as I loved being able to physically see what I had achieved.
But then came the RoboCup robot~\cite{roboCup}.
It was meant to be a rolling robot for playing football on a miniature playing field.
After two years, we were still unable to follow the ball because the Corona pandemic prevented us from working on site together.
Also, and to a greater extent, it didn't work because the hardware never did what it was meant to, and we passed interminable hours just trying to make it roll forward.
As may have become apparent, I got tired of it and decided to move on to something which didn't include too much hardware and was more abstract.

So I decided to participate in the Swiss Olympiad in Informatics, which organizes national programming contests and selects various teams for international Olympiads.
There, I got and still get quite a lot of interesting problems which I love to solve, found like-minded friends and was able to participate at some international competitions.
Over the years, I began to notice that I enjoy solving tasks theoretically way more than implementing them.
That is also something that got reflected in my competition scores, where I often came out knowing I solved a lot in theory, but failed to get the points.
Some internships at informatics firms confirmed that actual programming is still too concrete for me.

Now let's get more abstract.
Thanks to the ``Schülerstudium'', I attended a course on the mathematical background of computer science~\cite{discrete-maths} and one on the theory of computer science~\cite{theory-cs}.
There, I learned more about Complexity Theory, computability, logic and the Chomsky Hierarchy.
The way in which proofs could be made to hold for every problem with certain properties has fascinated me ever since.
Further, logic is a tool which captures mathematical reasoning, and can thus be seen as a formalization of every ``logical'' thought we have.
Also, I don't have to bother with implementation any more.

Using books, I began to inform myself more, and found out about the domain of Descriptive Complexity.
This domain relates different kinds of logic to different classes of problems in computer science.
So I knew I wanted to do my Matura project in this domain, but had difficulties finding an open question which seamed approachable.
I asked multiple people if they had a more exact idea what I could do.
In the end, a friend at the informatics Olympiad asked \emph{ChatGPT}, which told me I could study the connection between the Chomsky Hierarchy and Descriptive Complexity.
Refining this proposition a bit, I came up with (almost) the current question of relating logic and context-sensitive languages.
Then, I found out there exists a characterization using second-order logic, so I added ``first-order logic'' to the question.


% === INHALTSVERZEICHNIS ===
\newpage
\pagenumbering{roman}
\tableofcontents %          \
\newpage %                  \
% --------------------------

% =========== START VON STYLING ===========
\renewcommand{\chaptermark}[1]{\markboth{\MakeUppercase{\thechapter.\ #1}}{}} % Kapitel
\renewcommand{\sectionmark}[1]{\markright{\thesection.\ #1}{}} % Abschnitt
\fancyhead[R]{\rightmark}
\fancyhead[L]{\leftmark}
\fancyhead[L]{\leftmark}
\fancyhead[R]{\rightmark}
\pagestyle{fancy}
% =========== ENDE VON STYLING ============
\pagenumbering{arabic}
% === KAPITEL DER ARBEIT ===
%! suppress = UnresolvedReference
%! suppress = MissingImport
\chapter{Introduction}\label{ch:intro}

In our daily lives, we are in contact with various kinds of algorithms at all times.
Searching on Google, sending texts, asking questions to AI chatbots, searching for the fastest route with a GPS; all of these consume resources in energy, storage space and time.
According to a report from 2021, 3.7 \% of global carbon emissions come from the IT domain, with an upward tendency.
This is similar to that of the airline industry~\cite{webFootprint}.
At this scale, it is thus vitally important to understand and find out if the resource consumption can be reduced.

To find out how we can improve, we first need to understand how computers work, what we can compute, and why solving some problems is more difficult than solving others.
The field of study that investigates this is called Complexity Theory.
This is done by abstraction of computational models and of the problems themselves.
It is then often possible to find classes of similar problems which allow for generalizations.
However, it has proven to be very hard to find proofs of either optimality of algorithms or the separations of complexity classes.
One of the emblematic open problems is the P versus NP question, which we will go further into in~\cref{subsubsec:pnp}.
Nevertheless, some significant results concerning the complexity of problems on average in the real world and most importantly in cryptography have been made.

One of the subfields of Complexity Theory is Descriptive Complexity.
It relates mathematical logic to different complexity classes.
This allows us to get new insights into the underlying structure of problems of certain classes.
The other main field present in this work, formal languages, is concerned with the abstraction of computational problems as sets, which is often helpful for proofs.
This then allows us to define multiple formalisms which describe these sets.
Our formal language class of focus, context-sensitive languages, is part of the ``Chomsky Hierarchy'' introduced by the famous linguist Noam Chomsky.

Towards the end of the 20$^{\text{th}}$ century, many equivalences were proven between fragments and extensions of various logics to complexity classes, including all classes in the Chomsky hierarchy.
A great summary of all the results is found in Neil Immermans paper ``Languages that capture complexity classes''~\cite{Immerman1987}.

In this work, we will first introduce the relevant theory.
After this, a personal study of connections between first-order logic and context-sensitive languages is presented.

Both \cref{ch:formal-languages} and \cref{ch:descriptive-complexity} present the most important theory of formal languages and Descriptive Complexity, respectively.
Additionally, in \cref{sec:results-concerning-the-chomsky-hierarchy} full proofs concerning equivalence of context-sensitive languages with logics and automata are presented.
These proofs are written in a way trying to show the motivation behind certain crucial steps.
Further, some details have been omitted for the sake of the simplicity and length of this work.
For the same purpose, many proofs, explanation, and examples for other classes in the Chomsky hierarchy were moved to \cref{ch:mathematical-context-and-further-proofs}.

After this introduction to the material, we go on in \cref{ch:personal-contribution} with my personal studies about connections between first-order logic and context-sensitive language.
In this process, different paths that I tried are presented and investigated.
Towards the end of the chapter, the direct connection to context-sensitive languages fades and the objects of study becomes more related to Savitch's Theorem, which plays a great role in the theory of space-bounded computation.
The most complicated proof is given in \cref{sec:alternating-bounds} and concerns a simulation of alternating Turing machines using iterative logic.
This divergence from the original working question was due to the hope that researching other, more loosely connected problems would ultimately help finding a solution for the core problem.
Also, I tried to pursue ideas I had instead of trying paths where I did not make any progress.

Finally, in \cref{ch:conclusion-and-direction} we reflect on what was accomplished in this work.
A review about the working process is presented, and finally we discuss further possibilities of research.

For the relevant mathematical background, \cref{ch:mathematical-background} contains the basic definitions of Set Theory, first and second-order logic, and Turing machines.

A full collection of description, proofs, techniques, and context information about Descriptive Complexity and languages in the Chomsky Hierarchy is given in \cref{ch:mathematical-context-and-further-proofs}.

The entire work is written in English because all mathematical literature is in that language.
%! suppress = UnresolvedReference
%! suppress = MissingImport
\chapter{Formal languages}\label{ch:formal-languages}


\section{Definitions}\label{sec:definition}

In informatics, we often get an input as a string of characters and want to compute some function on it.
Complexity theory mostly focuses on decision problems which ask whether some input fulfils some given property.
To formalize this, there is the concept of formal languages.
The following definitions are taken from the lecture Theory of Computer Science~\cite{theory-cs}.
For the mathematical background, refer to \cref{ch:mathematical-background}.

\begin{define}[Alphabet]
    An alphabet $\Sigma$ is a finite set of symbols.
\end{define}

\begin{define}[Word]
    A word $w$ over some alphabet $\Sigma$ is a finite sequence of symbols from $\Sigma$.
    We denote $\varepsilon$ as the empty word, $\Sigma^*$ as the set of all words over $\Sigma$, $xy$ as the concatenation of the two words $x$ and $y$, $x^{n}$ as the concatenation of $x$ with itself $n$ times, and $|x|$ as the number of symbols in $x$.
\end{define}

\begin{define}[Formal language]
    A formal language is a set of words over some alphabet $\Sigma$, or equivalently a subset of $\Sigma^*$.
\end{define}

For any computational decision problem, we can reformulate it as the problem of deciding if the input word is contained in the formal language consisting of all words which have the required property.


\section{Chomsky Hierarchy}\label{sec:chromsky-hierarchy}

One of the multiple ways to categorize formal languages was invented by Avram Noam Chomsky, a modern linguist, in \cite{Chomsky1959}.
It is based on the complexity of defining the language using formal grammars, which are a finite representation of formal languages (which can be infinite in the general case).

\subsection{Grammars}\label{subsec:grammars}

A grammar can informally be seen as a set of rules telling us how to generate all words in a language.

\begin{define}[Grammar]
    A grammar is a 4-tuple $\langle V, \Sigma, R, S \rangle$ consisting of
    \begin{itemize}
        \item[$V$:] The set of non-terminal symbols.
        \item[$\Sigma$:] The alphabet of terminal symbols.
        All words generated by the grammar are in $\Sigma^{*}$.
        \item[$R$:] The set of rules of the form $a \to b$ with $a$ and $b$ being words over the alphabet consisting of the union of $V$ and $\Sigma$.
        Any rule in $R$ must have at least one symbol from $V$ on its left-hand side.
        \item[$S$:] The start symbol from the set $V$.
    \end{itemize}
\end{define}

The non-terminal symbols in $V$ are symbols that are not in $\Sigma$ and exist for the purpose of steering the process of word generation.
No non-terminal symbols appear in any of the words generated by the grammar.

To generate the words, there is the concept of derivations.
\begin{define}[Derivation]
    First, one derivation step is defined:

    We say that $u'$ can be derived directly from $u$ if
    \begin{itemize}
        \setlength\itemsep{0.15em}
        \item $u$ is of the form $xyz$ and $u'$ is of the form $xy'z$ for some words $x, y, y', z$ consisting of symbols in $\Sigma$ and $V$.
        \item there exists a rule $y \to y'$ in $R$.
    \end{itemize}

    We say that a word is in the \emph{generated language} of a grammar if it consists only of symbols in $\Sigma$ and can be derived in a finite number of steps from $S$.
\end{define}

\begin{exmp}
    Consider the grammar $\langle \{S\}, \{a, b\}, R, S \rangle$ with
    \[
        R = \begin{Bmatrix*}[l]
                S \to aSb,
                &S \to \varepsilon
        \end{Bmatrix*}
    \]
    The generated language of this grammar is $\{\varepsilon, ab, aabb, \dots\} = \{a^{n}b^{n} \mid n \in \mathbb{N}_0\}$.
\end{exmp}

Now that we have a tool to describe some infinite languages using a finite description, we can further differentiate the complexity of a language by the minimum required complexity of the rules in any formal grammar describing that language.
In the main section, only the context-sensitive languages are presented, as they are important for later chapters.
In \cref{sec:formal-languages-app}, explanations for regular languages (\cref{subsec:regular-languages}), context-free languages (\cref{subsec:context-free-languages}), and recursive languages (\cref{subsec:recursive-languages}) are provided.

\subsection{Context-Sensitive Languages}\label{subsec:context-sensitive-languages}

We can define the most significant class of languages for this document  by multiple equivalent restrictions on the grammars.

One restriction is that all rules have to be of the form $\alpha\beta\gamma \to \alpha\varphi\gamma$ with $\alpha, \gamma$ being words over the union of $\Sigma$ and $V$, $\beta \in V$ and $\varphi$ being a nonempty word over $\Sigma \cup V$.
This means that only the non-terminal symbol is allowed to change.
Additionally, if $S$ is the start symbol and never occurs on the right-hand side of any rule, we may include the exception $S \to \varepsilon$.

Equivalently, we can require that for any rule $u \to v$ we have $|u| \leq |v|$ as shown in \cite{Parkes2002}.
This means that applying any rule will make the result longer.
Again, we also allow the exception $S \to \varepsilon$ if $S$ does not occur on the right-hand side of any rule in $R$.
These grammars are called noncontracting.

\begin{exmp}
    Consider the grammar $\langle \{S, B\}, \{a, b, c\}, R, S \rangle$ with
    \[
        R = \begin{Bmatrix*}[l]
                S \to abc, &S \to aSBc, \\
                cB \to Bc, &bB \to bb
        \end{Bmatrix*}
    \]
    It generates the language $a^{n}b^{n}c^{n}$ for $n \in \mathbb{N}_{1}$ and is noncontracting.
\end{exmp}

The last restriction, the one that is most useful for proofs, is the Kuroda normal form presented in~\cite{Pettorossi2022}, where all rules have one of the following structures:
\begin{itemize}
    \setlength\itemsep{0.15em}
    \item $A \to BC$
    \item $AB \to CB$
    \item $A \to a$
    \item $S \to \varepsilon$ if $S$ is the start symbol and does not occur on any right-hand side of a rule.
\end{itemize}
where $A, B, C, S \in V$ and $a \in \Sigma$.

The corresponding formalism for these languages is the linearly bounded nondeterministic Turing machine, which can only write on the tape cells that contained the input word in the beginning.
This and an equivalent extension of second-order logic are proven in~\cref{subsec:des-context-sensitive-languages}.

%! suppress = MissingImport
\chapter{Descriptive Complexity}\label{ch:descriptive-complexity}


\section{Aims}\label{sec:aims}

In mathematics, abstraction is one of the most important tools as it enables us to make general statements and prove them for all the concrete instantiations of a concept.
Formal Logic takes this even further and makes it possible to abstract mathematical thought itself.
In Computer science, we are often interested in the amount of resources needed to compute a certain function or solve a certain problem, speaking in terms of time and storage space.
Different forms of logic have the power to describe different types of problems.
By focusing on decision problems\footnote{Any problem can be reduced to boolean queries, for example by having a boolean querry meaning "the i$^{th}$ bit of an encoding of the answer is 1"}, we can say a corresponding logical characterisation of a problem is a formula $\varphi$ which is true if and only if a structure satisfies the required properties.
By looking at the complexity of formulas which are needed to describe problems in terms of relations, operators, variables and other metrics, we can often find remarkably natural classes of logic corresponding to classes of problems.

Using these results, many insights into the underlying structure of real-world problems can be made which in turn can give us better ways to deal with them.
Further, descriptive complexity has applications in database theory and computer aided verification and proofs.


\section{Tools}\label{sec:tools}

As always, we first need to present some tools and techniques which will be used later in the proofs.
Definition are again taken and modified from~\cite{theory-cs} and~\cite{descriptive-complexity}.

\subsection{Complexity Theory}\label{subsec:complexity-theory}

Complexity theory is the study of the resources, measured mostly in time and space, needed to compute certain problems\footnote{The specific model of computation is not important as all give almost the same results. We will assume turing machines}.
Also, we do not really care about constants in the computation, and thus use a notation which omits these.

\begin{define}[Big-O notation]
    Let $f, g$ be functions $f, g: \mathbb{N} \to \mathbb{R}_+$.

    We say that $f \in \mathcal{O}(g)$ if there exists positive integers $n_0, c$ such that for all $n \geq n_0$ we have \[f(n) \leq c\cdot g(n)\]
\end{define}

Complexity classes can then be defined as all the problems which have a turing machine satisfying some bounds that can compute their solutions.
Now we will define some common and important complexity classes.

\begin{define}
    [{DTIME[$\mathcal{O}(t)$]}]
    We say that a decision problem is in DTIME[$\mathcal{O}(t)$] if there exists a deterministic turing machine that takes a maximum of $f(n)$ steps on any input of size $n$ and $f \in \mathcal{O}(t)$.
\end{define}

\begin{define}[P]
    We say that a decision problem is in P if there exists a polynomial $q$ such that the problem is in DTIME[$\mathcal{O}(q)$]
\end{define}

\begin{define}
    [{DSPACE[$\mathcal{O}(t)$]}]
    We say that a decision problem is in DTIME[$\mathcal{O}(t)$] if there exists a deterministic turing machine that visits a maximum of $f(n)$ tape cells on any input of size $n$ and $f \in \mathcal{O}(t)$.
\end{define}

\begin{define}[PSPACE]
    We say that a decision problem is in PSPACE if there exists a polynomial $q$ such that the problem is in DSPACE[$\mathcal{O}(q)$]
\end{define}

We can go do the same for nondeterministic turing machines, and get the corresponding complexity classes NTIME, NP, NSPACE, NPSPACE.
There, we always take the maximum of tape cells and steps over any computation branch.

The complexity class P has a special meaning for computer scientists as these are the problems which are deemed "feasible" on modern computers.

\subsection{Reduction and Completeness}\label{subsec:reduction}

A reduction can informally be seen as a method of using a problem we already solved to solve a new problem by converting this new problem into an instance of the old problem.
These reduction can be very useful to define complete problems for complexity classes, which in turn enable us to prove theorems for all problems of a specific complexity class.

\begin{define}[first-order reduction]
    Let $\mathcal{C}$ be a complexity class and $A$ and $B$ be two problems over vocabularies $\sigma$ and $\tau$.
    Now suppose that there is some first-order query $I: \text{STRUC}[\sigma] \to \text{STRUC}[\tau]$ for which we have the following property:
    \[
        \mathcal{A} \in A \Leftrightarrow I(\mathcal{A}) \in B
    \]
    Then $I$ is a first order reduction from $A$ to $B$, denoted as $A \leq_{fo} B$.
\end{define}

First order reductions can then be used to show that some problem is also a member in some complexity class, as in most complexity classes, we can compute the first order query, and then we are left with a problem that we already know is in the required class.
The converse can also be shown: for some problem $B$ which is not in some complexity class $\mathcal{C}$, if we have $B \leq_{fo} A$, then $A$ is also not in $\mathcal{C}$, as otherwise $B$ would also be in $\mathcal{C}$, which is a contradiction.

Using the reductions, we can define completeness.

\begin{define}[Completness via first-order reductions for Complexity Class $\mathcal{C}$]
    We say some problem $A$ is complete for $\mathcal{C}$ via $\leq_{fo}$ if and only if
    \begin{itemize}
        \setlength\itemsep{0.2em}
        \item $A \in \mathcal{C}$
        \item for all $B \in \mathcal{C}$, we have $B \leq_{fo} A$
    \end{itemize}
\end{define}

Informally, a complete problem captures the essence of the complexity class.
Further, they have an application in some proofs of equivalences between complexity classes $\mathcal{C}$ and logics $\mathcal{L}$.
These proofs follow the following steps as in~\cite{descriptive-complexity}:
\begin{enumerate}
    \item Show that $\mathcal{L} \subseteq \mathcal{C}$ by providing a way to convert any formula $\varphi \in \mathcal{L}$ into an algorithm in $\mathcal{C}$.
    \item Find a complete problem $T$ for $\mathcal{C}$ via first-order reductions.
    \item Show that $\mathcal{L}$ is closed under first-order reductions, that is that any formula can be extended by first-order quantifiers and boolean connectives and stay in $\mathcal{L}$.
    \item Find a formula for $T$ in $\mathcal{L}$, which shows $T \in \mathcal{L}$.
\end{enumerate}
The above steps work, as for any problem $B$ in $\mathcal{C}$, there is a first-order reduction $I$ to $T$, and both $\mathcal{L}$ and $\mathcal{C}$ are complete via these reductions, so we also have $B \in \mathcal{L} = \mathcal{C}$.

\subsection{Ehrenfeucht-Fraïssé Games}\label{subsec:ehrenfeucht-fraisse-games}

Ehrenfeucht-Fraïssé games are combinatorial games which are equivalent to first-order formulas and their extensions.
Using these games, it is often possible to show inexpressibility results for certain problems in some logic $\mathcal{L}$.

As a motivation, we can look at what it means for a formula to hold on some structure.
Assume the formula has the form $\forall x\varphi(x)$.
Then this can be seen as some opponent choosing some element $a \in |\mathcal{A}|$ and us now needing to show that $\varphi(a)$ holds.
The case where the formula has the form $\exists x\psi(x)$ can be treated similarly, but we can choose the element ourselves.

Now for the formal definition
\begin{define}[Ehrenfeucht-Fraïssé Game]
    The \emph{$k$-pebble Ehrenfeucht-Fraïssé Game $\mathcal{G}_k$} is played by two players: the Spoiler and the Duplicator on a pair of structures $\mathcal{A}$ and $\mathcal{B}$ using $k$ pairs of pebbles.
    In each move, the spoiler places one of the remaining pebbles on an element of one of the two structures.
    Then, the duplicator tries to match the move on the other structure by placing the corresponding pebble on an element.
    We say that the duplicator wins the $k$-pebble Ehrenfeucht-Fraïssé Game on $\mathcal{A}, \mathcal{B}$ if after the $k$ rounds, the map $i : |\mathcal{A}| \to |\mathcal{B}|$ defined as for all elements of $|\mathcal{A}|$ with a pebble and the constants as the element in $|\mathcal{B}|$ with the corresponding pebble or constant forms a partial isomorphism.
    A partial isomorphism is an isomorphism formed for some subset of the universe, with all relations restricted to that subset.
\end{define}

In this context, the spoiler wants to show that $\mathcal{A}$ and $\mathcal{B}$ are different, whereas the duplicator wants to show their equivalence.

As this is a zero-sum game of full information, one of the two players must have a winning strategy.
It can be proven that if the duplicator has a winning strategy for the $k$-pebble Ehrenfeucht-Fraïssé on $\mathcal{A}$ and $\mathcal{B}$ if and only if $\mathcal{A}$ and $\mathcal{B}$ agree on all formulas with less or equal to $k$ nested quantifiers.
We can use these facts to prove inexpressibility of some problems in first-order logic by exhibiting two structures $\mathcal{A}_k$ and $\mathcal{B}_k$ for each $k$ with one satisfying the problem constraints and the other not and a winning strategy for the duplicator on these two structures.
This methodology can be extended to other logics by adding new moves or restrictions to the game.

\section{Important Results}\label{sec:important-results}

\subsection{NSPACE[$s(n)$] $\subseteq$ DSPACE[$s(n)^2$]}\label{subsec:nspacesubsetdspacesquared}

This result is part of Savitch's Theorem, which introduces alternating turing machines in an intermediate step.
These TMs are a generalisation of nondeterministic TMs, which can be seen as machine taking the "or" of all it's computation paths.

\begin{define}[Alternating Turing Machine]
    An alternating Turing machine is a turing machine with two types of states: the existential and universal gates.
    Now, the acceptance conditions change compared to a NTM and depends on the state we are currently in.
    If we are in an existential state, we accept if and only if \emph{at least one} of the computations leading on from this configuration is accepting.
    If we are in an universal state, we accept if and only \emph{all} the computations leading on from this configuration are accepting.
\end{define}

These ATMs are now capable of also taking the "and" of the child states and maintain the ability to take the "or" of its children.

We define ATIME[$\mathcal{O}(t)$] and ASPACE[$\mathcal{O}(t)$] analogously to NTIME[$\mathcal{O}(t)$] and NSPACE[$\mathcal{O}(t)$].

Now we can proceed to the (quite technical) proof of Savitch's Theorem as presented in~\cite{descriptive-complexity}.

\begin{theorem}[Savitch's Theorem]
    For all space-constructible functions $t \geq \log n$ we have
    \[
        \text{NSPACE}[\mathcal{O}(t)] \subseteq \text{ATIME}[\mathcal{O}(t^2)] \subseteq \text{DSPACE}[\mathcal{O}(t^2)]
    \]
\end{theorem}

\begin{proof}
    We start with the first inclusion, $\text{NSPACE[$\mathcal{O}(t)$]} \subseteq \text{ATIME[$\mathcal{O}(t^2)$]}$.
    We now need to show that any NSPACE[$\mathcal{O}(t)$] turing machine can be simulated by a ATIME[$\mathcal{O}(t^2)$] alternating turing machine.
    Let $N$ be a NSPACE[$\mathcal{O}(t)$] turing machine.
    Without loss of generality, we assume that $N$ clears its tape after accepting and goes back to the first cell.

    Now consider $G_w$, the computation graph of $N$ on input $w$.
    We now see that $N$ accepts $w$ if and only if there is a path from the start configuration $s$ to the accepting configuration $t$.
    We now present a routine $P(d, x, y)$ which asserts that there is a path of length at most $2^{d}$ from vertex $x$ to $y$.
    Inductively, we can define $P$ as follows:
    \[
        P(d, x, y) = (\exists z)(P(d - 1, x, z) \land P(d - 1, z, y))
    \]
    This formula asserts that there exists a middle vertex $z$ for which there is a $2^{d - 1}$ path from $x$ to $z$ and from $z$ to $y$.
    Using an alternating turing machine, we can evaluate the formula using an existential state to find the middle vertex $z$, and then an universal state covering both shorter paths.

    Now for the runtime analysis, we see that we need $\mathcal{O}(t(n))$ time to write down the middle vertex $z$, as a configuration includes the tape, which has length $\mathcal{O}(t(n))$.
    Further, we then need to evaluate some $P(d - 1, z, y)$.
    By induction, we find that we need $\mathcal{O}(d\cdot t(n))$ time to compute $P(d, x, y)$.
    By the fact that there are only $2^{\mathcal{O}(t(n))}$ possible configurations, we get that the initial $d$ is also in $\mathcal{O}(t(n))$, and thus our total runtime is $\mathcal{O}(t(n)\cdot t(n)) = \mathcal{O}(t(n)^2)$.

    For the second inclusion, we need to simulate a ATIME[$\mathcal{O}(s(n))$] machine $A$ using a DSPACE[$\mathcal{O}(s(n))$] machine (here, we substituted $s(n)$ for $t(n)^2$).
    Again, we consider the computation graph of $A$ on input $w$.
    This graph has depth $\mathcal{O}(s(n))$ and size $2^{\mathcal{O}(s(n))}$.

    We can systematically search this computation graph to get our answer.
    This is done by keeping a string of choices $c_{1}c_{2}\dots c_r$ of length $\mathcal{O}(s(n))$ made until this point.
    Note that this uniquely determines which state we are in.

    Now, we can find the answer recursively.
    If we are in a halt state, we report this back to the previous state.
    In an existential state, we simulate its children, and if we get a positive result from one of them, we also return a positive result.
    In an universal state, we simulate its children, and if we get a positive result from all of them, we return a positive result.

    In total, we use only $\mathcal{O}(s(n))$ space, to simulate $A$.

    Thus, the second part of the theorem follows and by transitivity of $\subseteq$ we have $\text{NSPACE}[\mathcal{O}(t(n))] \subseteq \text{DSPACE}[\mathcal{O}(t(n)^2)]$.
\end{proof}

We do not know if the containment is strict or not for any of the inclusions of the theorem.
From this theorem, we also get the following interesting corollary.

\begin{corollary}
    We have $\text{PSPACE} = \text{NPSPACE}$.
\end{corollary}

\begin{proof}
    \begin{align*}
        \text{NPSPACE} &= \bigcup_{k \in \mathbb{N}}^{\infty}\text{NTIME}[\mathcal{O}(n^k)] \\
        &\subseteq \bigcup_{k\in \mathbb{N}}^{\infty}\text{DTIME}[\mathcal{O}(n^{2k})] \\
        &= \bigcup_{k\in \mathbb{N}}^{\infty}\text{DTIME}[\mathcal{O}(n^{k})] \\
        &= \text{PSPACE} \\
        &\subseteq \bigcup_{k\in \mathbb{N}}^{\infty}\text{NTIME}[\mathcal{O}(n^{k})] \\
        &= \text{NPSPACE}
    \end{align*}
\end{proof}

\subsection{SPACE Hierarchy theorem}\label{subsec:space-hierarchy-theorem}

The SPACE hierarchy theorem states that for both nondeterministic and deterministic space, we have problems that can be solved in some space $t(n)$, but not in less.
Formally, we have
\[
    \text{DSPACE}[o(t)] \subsetneq \text{DSPACE}[\mathcal{O}(t)]
\]
where $o(t)$ is the set of functions $f$ such that $f \in \mathcal{O}(t)$ but $t \not \in \mathcal{O}(f)$, that is all functions that grow more slowly than $t$.
This holds for all space-constructible $t \geq \log n$.
The same holds for NSPACE.

We will present a proof for deterministic space.
\begin{proof}
    The proof uses a diagonalization argument by presenting some machine $D$ that takes a turing machine $M$ and an input size in unary as input and does the opposite of $M$ if it halts.
    We want to show that for all $M$ which run in space $f(n) \in o(t(n))$, we have an input on which $D$ and $M$ do not agree.
    This would show that the language computed by $D$ is not in DSPACE[$o(t)$], and thus the strict containment.

    On input $\langle M, 1^{k} \rangle$ our machine $D$ marks of $t(|\langle M, 1^{k} \rangle|)$ tape cells, which are the cells that are allowed for the computation.
    Further we also maintain a counter with size $|M|\cdot 2^{t(|\langle M, 1^{k} \rangle|)}$, which is the maximum amount of different configurations a TM can pass before looping on a binary tape of size $t(|\langle M, 1^{k} \rangle|)$.
    Then, we simulate $M$ on input $\langle M, 1^{k} \rangle$.
    If we transcend any bound, we reject.
    For all $M$ in DSPACE[$o(t)$], there is a $k$ such that $f(n) \leq t(n)$ by definition.
    On this input, the simulation finishes, and we can invert the output.

    This directly gives us an input for which $M$ and $D$ differ, and thus proves our claim.
    Furthermore, $D$ runs in DSPACE[$\mathcal{O}(t)$] as by construction we assured that we do not run infinitely and that we stay within the space bound.
\end{proof}


\section{Results concerning the Chomsky hierarchy}\label{sec:results-concerning-the-chomsky-hierarchy}

Now that we have seen most of the required theory, we can start to apply it to the main theme of this work, the Chomsky hierarchy.
For this section, we define the vocabulary on strings to be $\sigma = \langle \{0, \dots, n - 1\}, Q_a, Q_b, \dots, Q_z, \leq , 0, 1, \text{max} \rangle$.
The universe consists of the numbers from $0$ to $n - 1$, we have a unary predicate for each character in $\Sigma$, a total ordering on the universe, and the constants $0, 1$ and $\text{max} = n - 1$.


\subsection{Regular Languages}\label{subsec:des-regular-languages}

Here, we will show that the regular languages are captured exactly by second-order logic where we restrict ourselves to quantify only over predicates of arity one and do not include $\leq$.
Further, we also are not allowed to use $\leq$, but have access to equality $x = y$ and the successor relation $x = y + 1$.
We call this class SOM[$+1$].

First we need to present a formal definition of deterministic finite automata.
\begin{define}[DFA]
    A deterministic finite automaton is a 5-tuple $M = \langle Q, \Sigma, \delta, q_0,  F \rangle$ where
    \begin{itemize}
        \item[$Q$] is the set of states
        \item[$\Sigma$] is the alphabet
        \item[$\delta$] is the transition function mapping a state and a symbol to the next state, so formally $\delta : Q\times \Sigma \to Q$.
        \item[$q_0$] the start state
        \item[$F$] a subset of $Q$ which are the accepting states.
    \end{itemize}
\end{define}
We say that a DFA $D$ accepts a word $w \in \Sigma^{*}$ if when starting at the start state, if we go through $w$ and always transition to the next state according to the actual symbol in $w$ and the actual state, we end up in an accepting state.

In~\cite{theory-cs} and~\cite{Straubing1994} there is a proof of the following fact we will use in our proof for SOM[$+1$]:
\begin{theorem}
    For any alphabet $\Sigma$, there is a DFA recognising language $L \subseteq \Sigma^{*}$ if and only if it is regular.
\end{theorem}

Now we can start to prove our main theorem for regular languages.
\begin{theorem}
    For any alphabet $\Sigma$, a language $L \subseteq \Sigma^{*}$ is expressible in SOM[$+1$] if and only if it is regular.
\end{theorem}

\begin{proof}
    First we show that any regular language can be expressed in SOM[$+1$].
    Let $L$ be regular, end $D_L$ be a DFA recognising the language.
    We assume $L$ does not contain the empty word, otherwise we can recognise the language $L \setminus \{\varepsilon\}$ and then add $\varphi \lor \forall x(x \neq x)$, which adds the empty string back.

    Now let $D_L$ have $k$ states.
    We can existentially quantify unary relations $X_1, \dots, X_k$ to have the meaning that $X_i(y)$ is true if and only if $D_L$ is in state $i$ after $y$ steps.
    Then, we need to make consistency checks.
    We present formulas for each of the consistency checks, and then can take the "and" of those to get our final formula $\exists X_1, \dots, X_k(\varphi_1 \land \varphi_2 \land \varphi_3)$.
    \begin{description}
        \item[The start state is $q_j$] we have \[\varphi_1 \coloneqq \bigwedge_{i = 1}^{k} (i = j \leftrightarrow X_j(0))\]
        \item[We end in an accepting state] Let $T_i$ be the set of all characters which lead from $q_i$ to an accepting state.
        Then we have
        \[
            \varphi_2 \coloneqq \bigwedge_{i = 1}^{k}\left(X_i(\text{max}) \to \bigvee_{a \in T_i} Q_a(max)\right)
        \]
        \item[We move according to the transition function] We have
        \begin{align*}
            \forall x\left( \forall y \left( y = x + 1 \to \left(\bigwedge_{i = 1}^{k} \bigwedge_{a \in \Sigma} \left(\left(X_i(x) \land Q_a(x)\right) \to X_{\delta(i, a)}(y)\right) \right. \right. \right. \\
            \left. \left. \left.\land \bigwedge_{i = 1}^{k}\bigwedge_{a \in \Sigma} \left(\left(X_i(y) \land Q_a(x)\right) \to \bigvee_{r = 1}^{k}\left(X_r(x) \land \delta(r, a) = j\right)\right)  \right)  \right) \right)
        \end{align*}
    \end{description}
    By induction we can show that always exactly one $i$ satisfies $X_i(x)$ for any $x$.
    Thus, if the created formula is satisfied, we know that $D_L$ accepts the word, and thus we have described $L$ in SOM[$+1$].

    For the other direction, we need to introduce two new concepts.

    One of them is the nondeterministic finite automaton, which is analogous to the nondeterministic turing machine as it can also have multiple transitions going from the same state.
    As with the NTM and the TM, both the DFA and the NFA have the same expressive power.

    The other concept is that of $(\mathcal{V}_1, \mathcal{V}_2)$-structures.
    These structures are generalisations of our former vocabulary $\sigma$ as they have characters in $A \times \mathcal{P}(\mathcal{V}_1)\times \mathcal{P}(\mathcal{V}_2)$.
    These structures are useful as we can make $\mathcal{V}_1$ to be the set of free first-order variables in a formula $\varphi$ and $\mathcal{V}_2$ be the set of free second-order variables in the formula.
    If at a position $i$ in our $(\mathcal{V}_1, \mathcal{V}_2)$-structures we have $x$ in the first-order component of its character, we see this as meaning that $x = i$.
    For the second-order variables in the third component, a $X$ at position $i$ means that $X(i)$ holds.

    Now, we can prove by induction that all formulas in SOM[$+1$] with free variables in $\mathcal{V}_1$ and $\mathcal{V}_2$ are regular.
    Sentences, the formulas without free variables are the special case where $\mathcal{V}_1, \mathcal{V}_2 = \emptyset$.

    First, we need to check that the $(\mathcal{V}_1, \mathcal{V}_2)$-structures are consistent, and no first-order variable $x$ appears more than once.
    This can be done by a NFA which has one state for each subset of variables, and extends its subset while going over the string.
    If a variable appears twice, we enter a state that always loops and rejects.

    Then, we see that the atomic formulas can be checked, as $x = y$, $x = y + 1$ and $Q_a(x)$ are easy to check, and checking $X(x)$ is equivalent to looking if the occurrence of $x$ has $X$ in the third component.
    We always need to take the intersection with the NFA which checks if the structure is valid.

    All boolean connective are also valid as regular languages are closed under complement, intersection and union as seen in~\cite{theory-cs}.

    The most difficult case is a formula of the form $\exists x \varphi$ (as $\forall x \varphi \equiv \neg \exists x \neg \varphi)$).
    If $\exists x \varphi$ is over $(\mathcal{V}_1, \mathcal{V}_2)$-structures, then $\varphi$ is over $(\mathcal{V}_1 \cup \{x\}, \mathcal{V}_2)$-structures.
    By induction, we know that $\varphi$ defines a regular language and thus there is a NFA $N$ which recognises it.
    For the new automaton, we duplicate our states, with the meanings "used $x$" and "not used $x$".
    If we are in a state where $x$ was used, we can not take any transition with $x$ in the second set.
    If we are in a state where $x$ was not used, we can take a transition with $x$ in the second set and go to the corresponding state with $x$ used or take a transition where $x$ is not used and go to the corresponding state where $x$ was not used.

    The remaining case with second-order variables is treated analogously, without the restriction on the number of times the variable is used, so we do not need to duplicate our states.

    By induction, we have thus showed the other direction, and we see that SOM[$+ 1$] and the regular languages are equivalent.
\end{proof}

\subsection{Context-Free Languages}\label{subsec:des-context-free-languages}

\subsection{Context-Sensitive Languages}\label{subsec:des-context-sensitive-languages}

\subsection{Recursive Languages}\label{subsec:des-recursive-languages}


\section{Open questions}\label{sec:open-questions}

The domain of descriptive complexity is full of open questions as the proofs of lower bounds seems to be very difficult in most cases.
Further, even separation between complexity classes wich seem to take an exponential amount of resources compared to another one in practice can not be shown to be different.

\subsection{P$\overset{?}{=}$NP}\label{subsec:pnp}
The P vs. NP question is the most emblematic question in descriptive complexity theory.
In practice, for any NP-complete problem, only exponential worst-case algorithms are known.
This leads to the widely believed conjuncture that P $\neq$ NP.
The problem is one of the seven Millennium Problems and a solution of equality or inequality is worth 1 Million US dollars.

The consequences of a solution stating that P $=$ NP could have many practical advantages if it was constructive and had a low constant, as many important problems in research and logistics could be solved quickly.
It would also mean the breakdown of most of modern cryptography, which relies on problem being intractable.
On a conceptual level, it would mean that finding a proof to a problem is not harder than verifying its correctness, which would greatly impact the work of mathematicians.
If a proof of the contrary would be known, this would focus the research more on the average case complexity of NP problems, but because of the continued lack of success on the question, this shift has already widely taken place.

\subsection{NSPACE[$O(n)$]$\overset{?}{=}$DSPACE[$O(n)$]}\label{subsec:nspacedspace}

This problem is known under the name first Linear bounded automaton problem since its proposal by Kuroda in~\cite{Kuroda1964}, and asks if nondeterminism adds power in the context of bounded space.
This comes from the fact that a NSPACE[$\mathcal{O}(n)$] turing machine can be seen as a TM with a linear bound on its space usage.
This theorem is of interest as we know that NSPACE[$\mathcal{O}(n)$] is equivalent to the context-sensitive languages by~\cref{subsec:des-context-sensitive-languages}.

Since the proposal, there were two advances.
One is the proof that NSPACE is closed under complement.
The contrary would have implied $\text{NSPACE}[\mathcal{O}(n)] \neq \text{DSPACE}[\mathcal{O}(n)]$ as DSPACE is closed under complement.
The second advance is Savitch's Theorem in~\cref{subsec:nspacesubsetdspacesquared} which already gives a bound for simulating NSPACE using DSPACE machines.
It is not known if this theorem is optimal, that is whether the blowup by a power of 2 is optimal or if we can do better.

An equality would imply that the context-sensitive languages can be recognised by a deterministic linear bounded automaton, which could make recognising words in context-sensitive langauges easier and faster.
%! suppress = MissingImport
\chapter{Personal Contribution}\label{ch:personal-contribution}

In this chapter, multiple tries at searches for new first order logics and other related concepts are presented.
The range of success as well as the range of the concrete themes are broad.
I did not find a fundamentally new result about the characterisation of context-sensitive languages using first order logic, but managed to prove that various approaches could not work.
Further, I also lowered some upper bounds for simulating alternating Turing machines using iterative logic.
Using a restricted form of iterative logic, I could lower the bounds for an iterative simulation of a nondeterministic Turing Machine.

\section{Direct Transformation}

Similar to Immerman in~\cite{descriptive-complexity}, we will use extended variables to model second order variables in first order logic.
This will then directly give us a characterisation of NSPACE$[s(n)]$ without requiring any new insights, as we can just use the same technique as for second order logic.

\begin{define}[Extended Variable]
    The logic FO-VAR$[1, s(n)]$ has two types of variables.
    One are the normal domain variables, which we will denote by lowercase characters, ranging from $0$ to $n - 1$.
    Further, a formula can also include extended variables, which we will denote by uppercase characters, ranging from $0$ to $2^{s(n)\log(n)} - 1$, and thus having $s(n)\log(n)$ bits.
    The extended variables can not appear as an input to any predicate variables and can only be used in quantification and as an argument to the BIT relation.
    Thus, we can query if a specific bit in the binary representation is on.
    For extended variables with more than $n$ bits, we can extend BIT to accept a tuple of domain variables encoding the position.
    This makes polynomial extended variables possible.
\end{define}

The extended variables in FO-VAR$[1, s(n)]$ have exactly the same capabilities as second order variables in second order logic, as we can query it at a position, but can not do anything else.
As we want to capture NSPACE$[s(n)]$, we will now prove that SO(TC, arity $k$) = FO-VAR$[1, n^k/\log(n)]$(TC).

\begin{proof}
    We show by induction on the structures that every formula has a very similar equivalent 
    For both, we show by induction that we can write an equivalent formula using the other logic.

    Any atomic formula of SO(TC, arity $k$) which does not include any second-order variable is trivially writable in FO-VAR$[1, n^k/\log(n)]$(TC).
    An atom of the form $Y(\overline{x})$ can be written as BIT$(Y', \overline{x})$ for $Y'$ being an extended variable.
    We then understand this as $Y$ being true with the variables $\overline{x}$ exactly when BIT$(Y', \overline{x})$ is true.
    This always works as we have relations of at most arity $k$, and thus $\overline{x}$ will never represent any higher number.

    Using this, we then can induct.
    Conjunction, disjunction and negation do not change anything.
    When we quantify over a second-order variable, we can directly exchange this with quantifying over an extended variable, as by induction the subformulas without the quantification is equivalent and plugging in the new second order / extended variable will not change this.
    Taking a transitive closure can also be done by replacing all second order variables with extended variables.

    By induction, we then have that all formulas in either of the logics has an equivalent formula in the other logic.
\end{proof}

Using this equivalence and the proof in \cref{subsec:des-context-sensitive-languages}, we get that FO-VAR$[1, n^k/\log(n)]$(TC) describes exactly the context-sensitive languages, and thus have our first characterisation in first order logic.

\section{Analogues to Proof for DSPACE}

For DSPACE$[s(n)]$, there are multiple other characterisations in logic without using the transitive closure operator.

To understand the following, we first need to define the logic FO$[t(n)]$, which formalises iterative definitions, and then VAR$[k]$ which restricts the number of variable but allows for unbounded FO iterations.

\begin{define}[{FO$[t(n)]$}]
    Let $Q_1, \dots, Q_n$ be a series of quantifiers, $s_1, \dots, s_n$ variables and $M_1 \dots, M_n$ be quantifier-free formulas.
    Then a quantifier block is $QB = (Q_{1}s_{1}.M_{n})\dots(Q_{n}s_{n}.M_{n})$.
    Here, for a universal quantifier $(\forall s.M)\varphi \equiv \forall s (M \to \varphi)$.
    Also, for an existential quantifier $(\exists s.M)\varphi \equiv \exists s(M \land \varphi)$.
    Then, a formula of FO$[t(n)]$ is of the form
    \[
        \left([QB]^{t(n)}M_{0}\right)(\overline{c}\setminus \overline{s})
    \]
    where $M_0$ is a quantifier-free formula, $\overline{c} = c_1, \dots, c_n$ is a tuple of constants and $\overline{s} = s_1, \dots, s_n$ are the variables occurring in the quantifier block.
    Also, $(\overline{c} \setminus \overline{s})$ means that in the beginning, we set $s_1 \coloneqq c_n, \dots, s_n \coloneqq c_n$ and $[QB]^{t(n)}$ means $QB$ literally repeated $t(n)$ times.
    The truth values of these formulas for a specific structure are defined by iterating the quantifier block $t(|\mathcal{A}|)$ times.
\end{define}
So now we have a formalism for iterative procedures.

If we restrict the number of variables such that all $s_i$ need to be in $\{x_1, \dots, x_k \}$, apart from some boolean variables (variables which can only be $0$ or $1$), we get FO-VAR$[t(n), k]$.
Additionally, we define
\[
    \text{VAR}[k] = \bigcup_{c = 1}^{\infty}\text{FO-VAR}[2^{cn^k}, k]
\]
This is the same as saying that we have unbounded iterations, as after at most $2^{cn^k}$ the truth values of the formula will loop.
The $c$ depends on the number of boolean variables included.

Now, we are ready to look at DSPACE\@.
For DSPACE, we then have
\[
    \text{DSPACE}[n^k] = \text{VAR}[k + 1]
\]
A proof of this can be found in \cite{descriptive-complexity}.
In the main part of the proof, a construction is made to simulate a DSPACE turing machine using VAR$[k + 1]$.
There, the relation $C_{t}(\overline{x}, \overline{b})$

\section{Restricting universal quantification}

\section{Alternating bounds}

\section{Arity Hierarchies}

\subsection{Transitive Closure}

\subsection{Generalised Quantifiers}


\chapter{Results}\label{ch:results}


%! suppress = UnresolvedReference
%! suppress = MissingImport
\chapter{Conclusion and Direction}\label{ch:conclusion-and-direction}

\section{Conclusion}\label{sec:conclusion}
In this work, we investigated multiple logics in relation to Savitch's Theorem.
This has led us to have a better notion of how all these logics are related to NSPACE$[s(n)]$ and thus also to linear-bounded automata and context-sensitive languages.

In the beginning, we introduced the theory of formal languages and the Chomsky Hierarchy.
Afterward, the main tools of Descriptive Complexity, including Complexity Theory and Ehrenfeucht-Fraïssé Games where presented.
Savitch's Theorem is the next proof included in this work which is then present throughout the work.
Then, the equivalence between context-sensitive languages and linear bounded automata is shown, giving a framework for what comes next.

Personally, I investigated multiple ideas I got during the time I read the theory.

The first one was a direct transformation of the results of connections between second-order transitive closure logic and nondeterministic space Turing machines to first-order logic.
This approach does not contain any new insights, but is a good showcase of the close correlation between first- and second-order logic.

After this, I wanted to explicitly prove that some methods could not work.
This branch of the work started with the investigation of the proof that DSPACE is equivalent to VAR$[k + 1]$.
There, I showed that naively applying this method gives wrong results, and that a natural extension by remembering decisions is suboptimal.
A full proof was impossible to make as it is unclear how to generalize this direction of work.
The difficulty of this problem is again illustrated by Savitch's Theorem, which has not been improved for a long time and thus makes it seem unlikely that a simulation of NSPACE machines with subquadratic DSPACE machines is possible.

The next method I tried was mixing transitive closure logic with iterative logic.
This turned out to be not very powerful, as any interesting result in one of the mixed logics would imply one in pure iterative logic.
Nevertheless, some upper and lower bounds were shown using Savitch's Theorem and the space hierarchy theorem.

In an attempt to make FO-VAR less powerful, I then considered restricting universal quantification.
A new formula made this possible, but also for this restriction, I was unable to find a way to simulate it using the transitive closure.
The generalization of this idea by adding more bits to the extended variables and decreasing the iteration count gives an interesting space-time tradeoff, but does not solve the inherent problem of simulating this formula by NSPACE machines.

The most complicated result I was able to get does not really have any connection to context-sensitive languages at all.
It concerns the simulation of alternating Turing machines using FO-VAR\@.
There a quite tight containment could be found which bounds some alternating Turing machine class from above and below by a factor of only $\log(s(n))$.
This result involved the combination of different techniques used in the results described above.
This result could lead onto a path to describe alternating space-time classes exactly using logic.

\subsection{Working process}\label{subsec:working-process}
Generally, this work was a great learning opportunity for me.
I was able to see how scientific research is done and could immerse myself in a domain that interests me.
I knew that I could not plan with any significant results, and indeed I worked on a lot of different branches, but most of the time hit a dead end.

In the beginning of the working process, I spend a lot of time reading multiple books about Descriptive Complexity and logic.
Then, I researched more papers on the topic and tried to understand lower and upper bounds that were described there.
This took a bit more time than expected, but I still was able to start investigating myself early enough.
As this is my first scientific research work, I did not know exactly how to approach it, so I wrote to Dr. Královi\v{c} from the ETH Zürich.
His tips game me good conditions to start.
During the months that followed, I tried multiple directions and read a lot of papers concerning new things I wanted to prove.
Quite soon, Ms. Sprunger and I discussed that starting to write the theoretical part of the project would be good so that I would not have too much stress in the end.
I did this during the summer break, experiencing some difficulties in the simplification of the proofs enough while still maintaining there correctness.
Afterward, I continued doing research.
Two months before handing in, the International Olympiad in Informatics took place.
My time plan did not account for this very well, so I had to work quite a lot in the last two weeks to write down what I had done.

Overall, I think that there are three main points which could be improved:
\begin{itemize}
    \setlength\itemsep{0.2em}
    \item First, I often feel that my proofs are not explained in a way that is clear and concise.
    This makes it difficult for the reader to follow the arguments and understand the underlying thought process.
    \item Another point is the methods of research.
    I found it difficult to find any resources on this matter for the highly abstract field of mathematics.
    Further, I had no mentor on this project which could help me along the right way.
    \item The time management was suboptimal, as I spend a lot of time on branches that where clearly not working and thus did not have time to investigate everything I wanted.
    Also, I did not start writing down my own research until very late, which generated some stress in the end.
\end{itemize}

\section{Directions}\label{sec:directions}
In the future, multiple topics could be investigated further to get an even deeper understanding on what happens in Savitch's Theorem.

In all proofs concerning equivalences between transitive closure logics and NSPACE$[s(n)]$, we assumed that $s(n)$ is at most polynomial.
This is still a quite profound limitation, and no characterization for superpolynomial bounds is known to me.
Finding a generalization could lead to new insights that would help in the polynomial case, and in the end to the case where $s(n) = n$ and we have exactly the context-sensitive languages.

In the context of Savitch's Theorem, it is interesting to investigate the computation graphs of NSPACE and DSPACE machines.
This has been discussed on the theoretical computer science stack exchange in~\cite{Barak2010}.
A motivation for this is that the number of nodes in both graphs are the same, only the number of edges vary.

Another direction that could be investigated is looking at normal forms.
One that seems to have some potential is the Kuroda Normal Form described in~\cite{Kuroda1964}.
This approach could yield a semantic restriction similar to the one for context-free languages described in \cref{subsec:des-context-free-languages}.

After showing that mixing TC and iterative procedures does not give any strong characterizations in \cref{sec:mixing-iterations-and-transitive-closure} and seeing that all interesting cases happen when the extended variables have less than $s(n)$ bits, one direction that took up most of the time and yielded very little results came up.
A very interesting result would have been to show that any iterative formula simulating a TC computation needs to have at least variables of the same size, as this would have shown that the path of investigating any logic which is a subset of these logics.
To do this I read multiple papers concerning hierarchies over the transitive closure, Generalized quantifiers and alternating space-time.
One of the most interesting of these is ``A double arity hierarchy theorem for transitive closure logic''~\cite{Grohe1996} which combines both and shows strict hierarchies on unordered graphs.
The main problem with these approaches is that they can not be generalized to include string structures, which are inherently ordered.
This makes it impossible to connect them to Turing machines which work with these structures.
Some steps in this direction were made in the chapter about proving lower bounds in~\cite{descriptive-complexity}.
\chapter*{Dank}
Ich darf hier wieder <<ich>> schreiben, und ich möchte Luca Bosin ganz herzlich
dafür danken, dass ich diese Vorlage auf Basis seiner Maturaarbeit anpassen
durfte und dabei gleich noch einige Neuerungen entdecken durfte, insbesondere
die verbesserte Literaturverzeichniserstellung mit biber.



% --------------------------


% === ABBILDUNGSVERZEICHNIS ===
\cleardoublepage %             \
\phantomsection %              \
\addcontentsline{toc}{chapter}{\listfigurename}
\listoffigures %               \
% -----------------------------

% === CODEBLOCKVERZEICHNIS ===
\cleardoublepage %            \
\phantomsection %             \
\addcontentsline{toc}{chapter}{\lstlistlistingname}
\lstlistoflistings %          \
% ----------------------------

% === LITERATURVERZEICHNIS ===
\cleardoublepage %            \
\phantomsection %             \
\addcontentsline{toc}{chapter}{\bibname}
\printbibliography %          \
% ----------------------------


% =========== START VON ANHÄNGE ===========
\appendix
%! suppress = UnresolvedReference
\chapter{Mathematical Background}\label{ch:mathematical-background}

The definitions are taken from the lectures Discrete Mathematics in Computer Science~\cite{discrete-maths} and Theory of Computer Science~\cite{theory-cs}, as well as from the book Descriptive Complexity~\cite{descriptive-complexity}.


\section{Set Theory}\label{sec:set-theory}
\begin{description}
    \item[Set] An unordered collection of distinct elements, written with curly braces $\{\}$
    \item[Tuple] An ordered collection of elements, written with pointed braces $\langle  \rangle$.
    \item[Set operations] There are multiple ways to form new sets from already existing sets:
    \begin{description}
        \item[Union] denoted as $\cup$.
        An element is in $A \cup B$ if and only if it is in $A$ or $B$
        \item[Intersection] denoted as $\cap$.
        An element is in $A \cap B$ if and only if it is in $A$ and $B$
        \item[Cartesian product] denoted as $\times$. $A \times B$ is the set of 2-tuples $\langle a, b\rangle$ with an element $a$ of $A$ and an element $b$ of $B$
        \item[Cartesian power] $A^k$ denotes the Cartesian product of $A$ with itself repeated $k$ times
        \item[Power set] denoted as $\mathcal{P}(A)$.
        Contains all subsets of $A$
    \end{description}
\end{description}


\section{First-Order Logic}\label{sec:first-order-logic}
\begin{description}
    \item[Variable] A variable is an element that can have a value from a set.
    We denote tuples of variables by $\overline{x} = \langle x_1, \dots, x_n \rangle$
    \item[Universe] The set over which variables and constants can range
    \item[Relation] A relation of arity $k$, $R(x_1, \dots, x_k)$ can be either true or false for any $k$-tuple of variables.
    In this document, we always consider an equality relation $=$, an ordering relation $\leq$, and \acs{BIT}$(x, 1y)$, which means that the $y^{th}$ bit of $x$ is $1$ in binary notation, to be present
    \item[Vocabulary] A tuple $\tau = \langle R_1^{a_1}, \dots, R_r^{a_r}, c_1, \dots, c_s \rangle$ of relations $R_i$ with arity $a_i$ and constants $c_j$\footnote{We omit functions, which are included in most textbook definitions, as they can be simulated by a relation in our case}
    \item[Structure] A tuple $\mathcal{A} = \langle |\mathcal{A}|, R_1^{\mathcal{A}}, \dots, R_r^{\mathcal{A}}, c_1^{\mathcal{A}}, \dots, c_s^{\mathcal{A}} \rangle$ where $|\mathcal{A}|$ is the universe, the constants are assigned a value from $|\mathcal{A}|$, and each relation is assigned a truth value for each $a_i$-tuple in $|\mathcal{A}|^{a_i}$.
    The set of all structures for a given universe is denoted as \acs{STRUCT}$[\tau]$
    \item[First-order formula] A first-order formula is inductively defined as follows:
    \begin{description}
        \item[Atoms] Any formula of the form $R(x_1, \dots, x_k)$ for some relation of arity $k$ is called an atomic formula
        \item[Conjunction] If $\varphi$ and $\psi$ are formulas, $(\varphi \land \psi)$ is a formula
        \item[Disjunction] If $\varphi$ and $\psi$ are formulas, $(\varphi \lor \psi)$ is a formula
        \item[Negation] If $\varphi$ is a formula, $\lnot \varphi$ is a formula
        \item[Existential quantification] If $\varphi$ is a formula, $\exists x \varphi$ is a formula
        \item[Universal quantification] If $\varphi$ is a formula, $\forall x \varphi$ is a formula
    \end{description}
    \item[Free variables] A variable in a formula which occurs at least once without being bound by a quantifier whose scope surrounds it
    \item[Semantics] For any structure, we can assign a truth value to any formula over the corresponding vocabulary.
    If the formula contains free variables, these need to be assigned a value from the universe first.
    We say $\mathcal{A}$ satisfies $\phi$, denoted as $\mathcal{A} \models \phi$, if and only if $\phi$ is true under the interpretation of the constants and relations in $\mathcal{A}$.
    This truth value is inductively assigned to all formulas as follows:
    \begin{description}
        \item[Atoms] For a formula $\phi$ of the form $R(x_1, \dots, x_k)$, we have $\mathcal{A} \models \phi$ if and only if the interpretation of the relation $R^{\mathcal{A}}$ maps $\langle x_1, \dots, x_k \rangle$ to true
        \item[Conjunction] We have $\mathcal{A} \models (\varphi \land \psi)$ if and only if $\mathcal{A} \models \varphi$ and $\mathcal{A} \models \psi$
        \item[Disjunction]  We have $\mathcal{A} \models (\varphi \lor \psi)$ if and only if $\mathcal{A} \models \varphi$ or $\mathcal{A} \models \psi$
        \item[Negation] We have $\mathcal{A} \models \lnot \varphi$ if and only if $\mathcal{A} \not\models \varphi$
        \item[Existential quantification] We have $\mathcal{A} \models \exists x\varphi$ if and only if there exists a $y \in |\mathcal{A}|$ such that $\mathcal{A} \models \varphi(y / x)$, where $\varphi(y / x)$ denotes $\varphi$ with any occurrence of $x$ replaced by the element $y$
        \item[Universal quantification] We have $\mathcal{A} \models \forall x\varphi$ if and only if for all $y \in |\mathcal{A}|$ we have $\mathcal{A} \models \varphi(y / x)$
    \end{description}
    \item[Logical operator] A logical operator can create a new formula from one or more existing formulas.
    One example is the conjunction, which combines two existing formulas.
    \item[First-order queries] A first-order query is a map from structures over one vocabulary $\sigma$ to structures over another vocabulary $\tau$.
    The mapping is done in such a way that first-order formulas define the universe, which is a subset of $|\mathcal{A}|^k$ for some $k$, the relation symbols, and all the constants.
    For a more formally thorough definition see~\cite{descriptive-complexity}
    \item[Isomorphism] An isomorphism is a map $I: |\mathcal{A}| \to |\mathcal{B}|$ with $\mathcal{A}, \mathcal{B}$ over the same vocabulary which satisfies the following properties:
    \begin{itemize}
        \setlength\itemsep{0.15em}
        \item $I$ is bijective
        \item for every available relation $R_i$ of arity $a_i$ and every $a_i$-tuple $\overline{e} = \langle e_1, \dots, e_{a_i} \rangle$ in $|\mathcal{A}|^{a_i}$, we have \[R_i^{\mathcal{A}}(e_1, \dots, e_{a_i}) \Leftrightarrow R_i^{\mathcal{B}}(I(e_1), \dots, I(e_{a_i}))\]
        \item for every constant symbol $c_j$, we have $I(c_j^{\mathcal{A}}) = c_j^{\mathcal{B}}$
    \end{itemize}
    If such an $I$ exists for two structures $\mathcal{A}$ and $\mathcal{B}$, we write $\mathcal{A} \cong \mathcal{B}$
\end{description}


\section{Second-Order Logic}\label{sec:second-order-logic}
In second-order logic, we extend the capabilities of first-order logic with the ability to quantify over relations.
We thus also need to extend our definitions.
\begin{description}
    \item[Second-order variables] A relation that is not given in the vocabulary and can be substituted with a specific relation
    \item[Second-order formula] In addition to the inductive rules of the first-order formulas, we can quantify over second-order formulas
    \begin{description}
        \item[Second-order existential quantification] If $\varphi$ is a formula, then $\exists V\varphi$ is a formula
        \item[Second-order universal quantification] If $\varphi$ is a formula, then $\forall V\varphi$ is a formula
    \end{description}
    \item[Second-order semantics] We also need to extend the first-order semantics
    \begin{description}
        \item[Second-order existential quantification]  We have $\mathcal{A} \models \exists V\varphi$ if and only if there exists a relation $U$ over $|\mathcal{A}|$ such that $\mathcal{A} \models \varphi(U / V)$, where $\varphi(U / V)$ denotes $\varphi$ with any occurrence of $V$ replaced by $U$
        \item[Second-order universal quantification] We have $\mathcal{A} \models \forall V\varphi$ if and only if for all relations $U$ over $|\mathcal{A}|$ we have $\mathcal{A} \models \varphi(U / V)$
    \end{description}
\end{description}


\section{Turing Machines}\label{sec:turing-machines}
Turing machines are the most common model of computation.
\begin{description}
    \item[Informal definition] A Turing machine is an automaton with a finite number of states and an infinite tape.
    Using a read/write head, which can read one symbol on the tape, modify one symbol on the tape and move left or right, a Turing Machine can compute functions
    \item[Formal definition] Formally, a Turing machine is a 7-tuple $M = \langle  Q, \Sigma, \Gamma, \delta, q_0, q_{accept}, q_{reject}\rangle$, where
    \begin{description}
        \item[$Q$] is the set of states
        \item[$\Sigma$] is the alphabet of the input word
        \item[$\Gamma$] is the set of symbols which can be written or read on the tape, which we call the tape alphabet
        \item[$\delta$] is the transition function, with $\delta : \Gamma \times Q \to \Gamma \times Q \times \{L, R\}$.
        This means that when a Turing machine is in state $n$ and reads $a$ on the tape, $\delta$ tells us to which state we should transition, which symbol we should write and which direction we should move the read/write head
        \item[$q_0$] the start state
        \item[$q_{accept}$] the accepting state
        \item[$q_{reject}$] the rejecting state
    \end{description}
    \item[Turing computation] In the beginning, the Turing machine is in the start state, the input word is written on a consecutive part of the tape and the read/write head is on the first character of the input word.
    In the following steps, the Turing machine state changes according to the transition function.
    If at some point the Turing machine enters the accepting or the rejecting state, the computation halts, and the Turing machine is said to have accepted / rejected the input.
    It can happen that the Turing machine continues indefinitely or loops.
    In that case, we also say that it has rejected its input.
    In this document, we ignore the tape content after the computation and focus on decision problems.
    \item[Decidability] If a Turing machine halts on all inputs, we say that it decides a problem, as we can always be sure that the machine will accept or reject an input in finite time.
    \item[Nondeterministic Turing machine] We can extend the transition function $\delta$ to allow multiple transitions from a given state.
    Formally, we then have $\delta : \Gamma \times Q \to \mathcal{P}(\Gamma \times Q \times \{L, R\})$.
    If at some point multiple transitions are possible from the current state, we can take any of them.
    If there exists any computational path which leads to an accepting state, the nondeterministic Turing machine accepts.
    This is not analogous to how real sequential computers work, but allows interesting results, and is not more powerful than a normal deterministic Turing machine.
    \item[Space/Time-constructible functions] A function $f(n)$ is time constructible if there exists a Turing machine which on input $1^{n}$ writes $f(n)$ in binary on its tape in time $f(n)$.
    Space-constructible functions are defined analogously.
    \item[Church-Turing thesis] The Church-Turing thesis states that anything that can be done on a real-world computer can be done using a Turing machine.
\end{description}


% === EIGENSTÄNDIGKEITSERKLÄRUNG ===
\chapter{Independence declaration (German)}\label{ch:appendix_independencedeclaration}
Ich, Yaël Arn, 4A \\ \\
bestätige mit meiner Unterschrift, dass die eingereichte Arbeit
selbstständig und ohne unerlaubte Hilfe Dritter verfasst wurde.
Die Auseinandersetzung mit dem Thema erfolgte ausschliesslich
durch meine persönliche Arbeit und Recherche. Es wurden
keine unerlaubten Hilfsmittel benutzt.
Ich bestätige, dass ich sämtliche verwendeten Quellen sowie
Informanten/-innen im Quellenverzeichnis bzw. an anderer dafür
vorgesehener Stelle vollständig aufgeführt habe. Alle Zitate
und Paraphrasen (indirekte Zitate) wurden gekennzeichnet und
belegt. Sofern ich Informationen von einem KI-System wie
bspw. ChatGPT verwendet habe, habe ich diese in meiner Maturaarbeit
gemäss den Vorgaben im Leitfaden zur Maturaarbeit
korrekt als solche gekennzeichnet, einschliesslich der Art und
Weise, wie und mit welchen Fragen die KI verwendet wurde.
Ich bestätige, dass das ausgedruckte Exemplar der Maturaarbeit
identisch mit der digitalen Version ist.
Ich bin mir bewusst, dass die ganze Arbeit oder Teile davon
mittels geeigneter Software zur Erkennung von Plagiaten oder
KI-Textstellen einer Kontrolle unterzogen werden können.

\vspace{3cm}

\noindent
\begin{tabular}{p{0.47\linewidth}p{0.47\linewidth}}
  Ort \&\ Datum & Unterschrift \\
  & \\[1cm]
  \hline
\end{tabular}

% ----------------------------------

% =========== ENDE VON ANHÄNGE ============

% =============================================================
\end{document} % ENDE VOM DOKUMENT ============================
% =============================================================
