\documentclass[a4paper,11pt]{report}
% !BIB TS-program = biber
% !BIB program = biber
% Die beiden obigen Zeilen helfen TeXShop auf Mac mit biber
% (Verarbeitung der Bibliographiedatenbank in literatur.bib)


% ======== START VON PAKETE LADEN =========
\usepackage[british]{babel}        % Deutschsprachige Beschriftungen
\usepackage[utf8]{inputenc}       % Utf8 Zeichensatz
\usepackage[T1]{fontenc}          % Schriftenkodierung
\usepackage[lighttt]{lmodern}     % Schriftart
\usepackage{amsmath}
\usepackage{amsthm}     % Mathematische Formeln
\usepackage{amssymb}
\usepackage{mathtools}
\usepackage[normalem]{ulem}       % Durchgestrichener Text
\usepackage{xcolor}               % Farbiger Text
\usepackage{verbatim}             % Text ohne Formatierung
\usepackage{listings}             % Code mit Formatierung
\usepackage{csquotes}             % Kontextsensitive Zitatanlage
\usepackage{caption}              % Erweiterte Beschriftungen
\usepackage{subcaption}           % Unterbeschriftungen
\usepackage{geometry}             % Seitenränder
\usepackage{setspace}             % Zeilenabstand
\usepackage{fancyhdr}             % Header und Footer
\usepackage{graphicx}             % Grafiken
\usepackage{wrapfig}
\usepackage{svg}                  % SVG Grafiken
\usepackage{booktabs}             % Tabellen
\usepackage{tabularx}             % Breite von Boxen
\usepackage[style=alphabetic]{biblatex} % Referenzen
\usepackage{hyperref}             % Hyperlinks im PDF-Dokument
\usepackage{hyperxmp}             % Metadaten im PDF-Dokument
\usepackage{makeidx}              % Optional, für Glossar (Index)
\usepackage[version=4]{mhchem}    % Für chemische Formeln
\usepackage{siunitx}
\usepackage{stackengine}
\usepackage{cleveref}
% ========= ENDE VON PAKETE LADEN =========


% ==== START VON DOKUMENTEIGENSCHAFTEN ==== (Dokument)
\title % Titel festlegen
{Theoretical Informatics: Formal languages and finite model thoery}
\date{\today}   % Datum festlegen
\author{Yaël Arn}  % Autor festlegen
\makeatletter            % Erlaubt das Auslesen der obigen Eigenschaften
\let\papertitle\@title% Titel in \papertitle speichern
\let\paperdate\@date     % Datum in \paperdate speichern
\let\paperauthor\@author % Autor in \paperauthor speichern
\makeatother
\def\papersubtitle
{A study of the connection of first order logic and context-sensitive languages}
\def\paperinstitution    % Bildungseinrichtung in \paperinstitution speichern
{Gymnasium Bäumlihof}
\def\papertype           % Art dieser Arbeit in \papertype speichern
{Maturaarbeit}
\def\papersupervisor     % Betreuungsperson in \papersupervisor speichern
{Aline Sprunger}
\def\papercoreferent{Bernhard Pfammatter}
\def\paperauthorclass
{4A}
\def\paperplace{4058 Basel}
% Achtung: Schlüsselwörter sind weiter unten definiert!
% ==== ENDE VON DOKUMENTEIGENSCHAFTEN =====


% =========================================


% === START VON SCHRIFTPROBLEME BEHEBEN === (Schriftart)
% Die Schreibmaschinen-Schrift lädt nur den normalen Stil, nicht den fetten oder den kursiven
\ttfamily
\DeclareFontShape{T1}{lmtt}{m}{it}{<->sub*lmtt/m/sl}{}
% === ENDE VON SCHRIFTPROBLEME BEHEBEN ====

% Unterstützung von Sonderzeichen
\lstset{literate=
{á}{{\'a}}1 {é}{{\'e}}1 {í}{{\'i}}1 {ó}{{\'o}}1 {ú}{{\'u}}1 {Á}{{\'A}}1 {É}{{\'E}}1
{Í}{{\'I}}1 {Ó}{{\'O}}1 {Ú}{{\'U}}1 {à}{{\`a}}1 {è}{{\`e}}1 {ì}{{\`i}}1 {ò}{{\`o}}1
{ù}{{\`u}}1 {À}{{\`A}}1 {È}{{\'E}}1 {Ì}{{\`I}}1 {Ò}{{\`O}}1 {Ù}{{\`U}}1 {ä}{{\"a}}1
{ë}{{\"e}}1 {ï}{{\"i}}1 {ö}{{\"o}}1 {ü}{{\"u}}1 {Ä}{{\"A}}1 {Ë}{{\"E}}1 {Ï}{{\"I}}1
{Ö}{{\"O}}1 {Ü}{{\"U}}1 {â}{{\^a}}1 {ê}{{\^e}}1 {î}{{\^i}}1 {ô}{{\^o}}1 {û}{{\^u}}1
{Â}{{\^A}}1 {Ê}{{\^E}}1 {Î}{{\^I}}1 {Ô}{{\^O}}1 {Û}{{\^U}}1 {œ}{{\oe}}1 {Œ}{{\OE}}1
{æ}{{\ae}}1 {Æ}{{\AE}}1 {ß}{{\ss}}1 {ű}{{\H{u}}}1 {Ű}{{\H{U}}}1 {ő}{{\H{o}}}1 {Ő}{{\H{O}}}1
{ç}{{\c c}}1 {Ç}{{\c C}}1 {ø}{{\o}}1 {å}{{\r a}}1 {Å}{{\r A}}1 {€}{{\euro}}1 {£}{{\pounds}}1 {«}{{\guillemotleft}}1 {»}{{\guillemotright}}1 {ñ}{{\~n}}1 {Ñ}{{\~N}}1 {¿}{{?`}}1
}


% ======== START VON SEITENRÄNDER ========= (Seitenränder)
\geometry{
a4paper,
total={150mm,237mm},
left=20mm,
top=30mm,
right=20mm,
bottom=30mm
}
\setlength{\headheight}{14pt}
% ========= ENDE VON SEITENRÄNDER =========


% ====== START VON QUELLBIBLIOTHEKEN ====== (Referenzen)
\addbibresource{literatur.bib}
% ====== ENDE VON QUELLBIBLIOTHEKEN =======


% ========= START VON METADATEN ===========
\hypersetup{
pdftoolbar=true,           % Toolbar anzeigen?
pdfmenubar=true,           % Menüleiste anzeigen?
pdffitwindow=false,        % Fenstergrösse anpassen?
pdfstartview={FitH},       % Breite dem Fenster anpassen
pdftitle={\papertitle},    % Titel
pdfauthor={\paperauthor},  % Autor
pdfsubject={\papertype},   % Thema
pdfcreator={pdfLaTeX},     % PDF-Ersteller
pdfproducer={\paperauthor},% Dokument-Ersteller
pdfkeywords={\paperinstitution} {\papersupervisor} % Schlüsselwörter
{LaTex} {Maturaarbeit} {Finite Model Theory} {Formal Languages},
pdfnewwindow=true,         % Links in neuem Fenster?
colorlinks=false,          % Farbige Links: true, Link-Boxen: false
linkcolor=red,             % Farbe interner Links
citecolor=green,           % Farbe der Zitat-Links
filecolor=magenta,         % Farbe der Datei-Links
urlcolor=cyan              % Farbe externer Links
bookmarks=true             % Lesezeichen erstellen?
bookmarksdepth=section     % Lesezeichen bis zu Abschnitten
bookmarksopen=false        % Lesezeichen öffnen?
bookmarksnumbered=true     % Kapitelnummern in Lesezeichen?
}
\pdfinfo{
/Title  (\papertitle)
/Author (\paperauthor)
/Subject (\papertype)
/Keywords (\paperinstitution;\papersupervisor;LaTex;Maturaarbeit;Finite Model Theory;Formal Languages)
}
\makeatletter
\def\@makechapterhead#1{%
    \vspace*{20\p@}%
    {\parindent \z@ \raggedright \normalfont
    \ifnum \c@secnumdepth >\m@ne
        \Huge\bfseries \thechapter.\space%
    \fi
    \interlinepenalty\@M
    \Huge \bfseries #1\par\nobreak
    \vskip 30\p@
}}
\makeatother
\newtheorem{theorem}{Theorem}[chapter]
\newtheorem{corollary}{Corollary}[theorem]
\newtheorem{lemma}[theorem]{Lemma}
\theoremstyle{definition}
\newtheorem{define}{Definition}[chapter]
\theoremstyle{definition}
\newtheorem{exmp}{Example}[chapter]
\addto\captionsbritish{
\renewcommand{\contentsname}{Table of Contents}
}
% ========== ENDE VON METADATEN ===========


% ========== START VON BILDPFADE ==========
\graphicspath{{img/pdf/}{img/png/}}
\svgpath{{img/svg/}}
% =========== ENDE VON BILDPFADE ==========


% =============================================================
\begin{document} % START VOM DOKUMENT =========================
% =============================================================

% === TITELSEITE ===
% Titelseite (https://de.wikibooks.org/wiki/LaTeX/_Eine_Titelseite_erstellen)
\begin{titlepage}
  \centering
  {\scshape\paperinstitution\par}
  \vspace{1cm}
  {\scshape\Large\papertype\par}
  \vspace{1.5cm}
  {\huge\bfseries\papertitle\par}
  \vspace{.5cm}
  {\LARGE\bfseries\papersubtitle\par}
  \vspace{2cm}
  {\Large Written by:\par\paperauthor, \paperauthorclass}
  \vfill
  \includegraphics[width=.7\textwidth]{title_image.pdf}
  \vfill
  Supervisor:\par
  {\sc\papersupervisor\par}
  \vspace{.5cm}
  Second Examiner:\par
  {\sc\papercoreferent\par}
  \vspace{1cm}
  {\large\paperdate, \paperplace\par}
\end{titlepage}
 %     \
% ------------------

\onehalfspacing % Zeilenabstand 1.5

\pagenumbering{gobble}
%! suppress = UnresolvedReference
\chapter*{Foreword}

Some years ago, the Lego Mindstorms \cite{lego} sparked my interest in informatics and programming.
By attending some courses at the Phænovum \cite{phaenovum} in Lörrach, I learned how to program using Java, my first text-based programming language.
At that time, I was planning to work at Boston Dynamics \cite{boston}, as I loved being able to physically see what I had achieved.
But then came the RoboCup robot~\cite{roboCup}.
Me and my colleagues from the Phænovum wanted to program a rolling robot for playing football on a miniature playing field.
After two years, we were still unable to follow the ball because the Corona pandemic prevented us from working on-site together.
Also, and to a greater extent, it didn't work because the hardware never did what it was meant to, and we spent interminable hours just trying to make it roll forward.
As may have become apparent, I got tired of it and decided to move on to something that didn't include too much hardware and was more abstract.

So I chose to participate in the Swiss Olympiad in Informatics \cite{soi}, which organizes national programming contests and selects various teams for international Olympiads.
There, I got many interesting problems which I loved to solve, found like-minded friends, and was able to participate in several international competitions.
Over the years, I began to notice that I enjoy solving tasks theoretically way more than implementing them.
This is also something that was reflected in my competition scores, where I often came out knowing I solved a lot in theory but failed to get the points.
Some internships at informatics firms confirmed that actual programming was still too concrete for me.

Thanks to the ``Schülerstudium'' at the University of Basel, I attended a course on the mathematical background of computer science~\cite{discrete-maths} and one on the theory of computer science~\cite{theory-cs}.
There, I learned more about abstract concepts such as Complexity theory, computability, logic, and the Chomsky hierarchy.
The way in which proofs could be made to hold for every problem with certain properties has fascinated me ever since.
Further, logic is a tool that captures mathematical reasoning, and can thus be seen as a formalization of every ``logical'' thought we have.
Also, I don't have to bother with implementation any more.

While using books to learn more, I found out about the domain of Descriptive Complexity.
This domain relates different kinds of logic to different classes of problems in computer science.
I knew I wanted to do my Matura project in this domain, but had difficulties finding an open question which seemed approachable.
I asked many people if they had a more exact idea of what I could do.
In the end, a friend at the informatics Olympiad asked \emph{ChatGPT} \cite{chatgpt}, which told me I could study the connection between the Chomsky hierarchy and Descriptive Complexity.
Refining this proposition, I came up with the subject of this project: relating first-order logic and context-sensitive languages.

% === INHALTSVERZEICHNIS ===
\newpage
\pagenumbering{roman}
\tableofcontents %          \
\newpage %                  \
% --------------------------

% =========== START VON STYLING ===========
\renewcommand{\chaptermark}[1]{\markboth{\MakeUppercase{\thechapter.\ #1}}{}} % Kapitel
\renewcommand{\sectionmark}[1]{\markright{\thesection.\ #1}{}} % Abschnitt
\fancyhead[R]{\rightmark}
\fancyhead[L]{\leftmark}
\fancyhead[L]{\leftmark}
\fancyhead[R]{\rightmark}
\pagestyle{fancy}
% =========== ENDE VON STYLING ============
\pagenumbering{arabic}
% === KAPITEL DER ARBEIT ===
%! suppress = UnresolvedReference
%! suppress = MissingImport
\chapter{Introduction}\label{ch:intro}

In our daily lives, we are in contact with various kinds of algorithms at all times.
Searching on Google, sending texts, asking questions to AI chatbots, searching for the fastest route with a navy; all of these consume resources in energy, storage space and time.
According to a report from 2021, 3.7 \% of global carbon emissions come from the IT domain, with an upward tendency.
This is similar to that of the airline industry~\cite{webFootprint}.
At this scale, it is thus vitally important to understand and find out if the resource consumption can be reduced.

To find out how we can improve, we first need to understand how computers work, what we can compute, and why solving some problems is more difficult than solving others.
The field of study that investigates this is called Complexity Theory.
This is done by abstraction of computational models and of the problems themselves.
It is then often possible to find classes of similar problems which allow for generalizations.
However, it has proven to be very hard to find proofs of either optimality of algorithms or the separations of complexity classes.
One of the emblematic open problems is the P versus NP question, of which we will speak more in~\cref{subsubsec:pnp}.
Nevertheless, some significant results concerning the complexity of problems on average in the real world and most importantly in cryptography have been made.

One of the subfields of Complexity Theory is Descriptive Complexity.
It relates mathematical logic to different complexity classes.
This allows us to get new insights into the underlying structure of problems of certain classes.
The other main field present in this work, formal languages, is concerned with the abstraction of computational problems as sets, which is often helpful for proofs.
This then allows us to define multiple formalisms which describe these sets.
Our formal language class of focus, context-sensitive languages, is part of the ``Chomsky Hierarchy'' introduced by the famous linguist Noam Chomsky.

Towards the end of the 20$^{\text{th}}$ century, many equivalences were proven between fragments and extensions of various logics to complexity classes, including all classes in the Chomsky hierarchy.
A great summary of all the results is found in Neil Immermans paper ``Languages that capture complexity classes''~\cite{Immerman1987}.

In this work, we will first introduce the relevant theory.
After this, a personal study of connections between first-order logic and context-sensitive languages is presented.

Both \cref{ch:formal-languages} and \cref{ch:descriptive-complexity} present the most important theory of formal languages and Descriptive Complexity, respectively.
Additionally, in \cref{sec:results-concerning-the-chomsky-hierarchy} full proofs concerning equivalence of context-sensitive languages with logics and automata are presented.
These proofs are written in a way trying to show the motivation behind certain crucial steps.
Further, some details have been omitted for the sake of simplicity and length of this work.
For the same purpose, many proofs, explanation, and examples for other languages in the Chomsky hierarchy were moved to \cref{ch:mathematical-context-and-further-proofs}.

After this introduction to the material, we go on in \cref{ch:personal-contribution} with my personal studies about connections between first order logic and context-sensitive language.
In this process, different paths that I tried are presented and investigated.
Towards the end of the chapter, the direct connection to context-sensitive languages fades and the objects of study becomes more related to Savitch's Theorem, which plays a great role in the theory of space-bounded computation.
The most complicated proof is given in \cref{sec:alternating-bounds} and concerns a simulation of alternating Turing machines using iterative logic.
This divergence from the original working question was due to the hope that researching other, more loosely connected problems would help in the end for the core problem.
Also, I tried to pursue ideas I had instead of trying paths where I did not make any progress.

Finally, in \cref{ch:conclusion-and-direction} we reflect on what was accomplished in this work.
A review about the working process is presented, and finally we discuss further possibilities of research.

For the relevant mathematical background, \cref{ch:mathematical-background} contains the basic definitions of Set Theory, first and second order logic, and Turing machines.

A full collection of description, proofs, techniques, and context information about Descriptive Complexity and languages in the Chomsky Hierarchy is given in \cref{ch:mathematical-context-and-further-proofs}.

The entire work is written in English because all mathematical literature is in that language.
\chapter{Formal Languages}\label{ch:formal-languages}

\section{Definition}\label{sec:definition}


\section{Chomsky Hierarchy}\label{sec:chromsky-hierarchy}

\subsection{Regular Languages}\label{subsec:regular-languages}

\subsection{Context-Free Languages}\label{subsec:context-free-languages}

\subsection{Context-Sensitive Languages}\label{subsec:context-sensitive-languages}

\subsection{Recursive Languages}\label{subsec:recursive-languages}
\chapter{Descriptive Complexity}\label{ch:descriptive-complexity}


\section{Aims}\label{sec:aims}


\section{Important Results}\label{sec:important-results}

\subsection{NSPACE[$s(n)$] $\subseteq$ DSPACE[$s(n)^2$]}\label{subsec:nspacesubsetdspacesquared}

\subsection{FO(LFP) $=$ P}\label{subsec:fo(lfp)=p}


\section{Open questions}\label{sec:open-questions}

\subsection{P$\overset{?}{=}$NP}\label{subsec:pnp}

\subsection{NSPACE[$O(n)$]$\overset{?}{=}$DSPACE[$O(n)$]}\label{subsec:nspacedspace}
\chapter{Personal Contribution}\label{ch:personal-contribution}
\input{results}
%! suppress = UnresolvedReference
%! suppress = MissingImport
\chapter{Conclusion and Direction}\label{ch:conclusion-and-direction}

\section{Conclusion}\label{sec:conclusion}
In this work, we investigated multiple logics in relation to Savitch's Theorem.
This has led us to have a better notion of how all these logics are related to NSPACE$[s(n)]$ and thus also to linear-bounded automata and context-sensitive languages.

In the beginning, we introduced the theory of formal languages and the Chomsky Hierarchy.
Afterward, the main tools of Descriptive Complexity, including Complexity Theory and Ehrenfeucht-Fraïssé Games where presented.
Savitch's Theorem is the next proof included in this work which is then present throughout the work.
Then, the equivalence between context-sensitive languages and linear bounded automata is shown, giving a framework for what comes next.

Personally, I investigated multiple ideas I got during the time I read the theory.

The first one was a direct transformation of the results of connections between second order transitive closure logic and nondeterministic space Turing machines to first order logic.
This approach does not contain any new insights, but is a good showcase of the close correlation between first- and second-order logic.

After this, I wanted to explicitly prove that some methods could not work.
This branch of the work started with the investigation of the proof that DSPACE is equivalent to VAR$[k + 1]$.
There, I showed that naively applying this method gives wrong results, and that a natural extension by remembering decisions is suboptimal.
A full proof was impossible to make as it is unclear how to generalize this direction of work.
The difficulty of this problem is again illustrated by Savitch's Theorem, which has not been improved for a long time and thus makes it seem unlikely that a simulation of NSPACE machines with subquadratic DSPACE machines is possible.

The next method I tried was mixing transitive closure logic with iterative logic.
This turned out to be not very powerful, as any interesting result in one of the mixed logics would imply one in pure iterative logic.
Nevertheless, some upper and lower bounds were shown using Savitch's Theorem and the Space Hierarchy Theorem.

In an attempt to make FO-VAR less powerful, I then considered restricting universal quantification.
A new formula made this possible, but also for this restriction, I was unable to find a way to simulate it using the transitive closure.
The generalization of this idea by adding more bits to the extended variables and decreasing the iteration count gives an interesting space-time tradeoff, but does not solve the inherent problem of simulating this formula by NSPACE machines.

The most complicated result I was able to get does not really have any connection to context-sensitive languages at all.
It concerns the simulation of alternating Turing machines using FO-VAR\@.
There a quite tight containment could be found which bounds some alternating Turing machine class from above and below by a factor of only $\log(s(n))$.
This result involved the combination of different techniques used in the results described above.
This result could lead onto a path to describe alternating space-time classes exactly using logic.

\subsection{Working process}\label{subsec:working-process}
Generally, this work was a great learning opportunity for me.
I was able to see how scientific research is done and could immerse myself in a domain that interests me.
I knew that I could not plan with any significant results, and indeed I worked on a lot of different branches, but most of the time hit a dead end.

In the beginning of the working process, I spend a lot of time reading multiple books about Descriptive Complexity and logic.
Then, I researched more papers on the topic and tried to understand lower and upper bounds that were described there.
This took a bit more time than expected, but I still was able to start investigating myself early enough.
As this is my first scientific research work, I did not know exactly how to approach it, so I wrote to Dr. Královič from the ETH Zürich.
His tips game me good conditions to start.
During the months that followed, I tried multiple directions and read a lot of papers concerning new things I wanted to prove.
Quite soon, Ms. Sprunger and I discussed that starting to write the theoretical part of the project would be good so that I would not have too much stress in the end.
I did this during the summer break, experiencing some difficulties in the simplification of the proofs enough while still maintaining there correctness.
Afterward, I continued doing research.
Two months before handing in, the International Olympiad in Informatics took place.
My time plan did not account for this very well, so I had to work quite a lot in the last two weeks to write down what I had done.

Overall, I think that there are three main points which could be improved:
\begin{itemize}
    \setlength\itemsep{0.2em}
    \item First, I often feel that my proofs are not explained in a way that is clear and concise.
    This makes it difficult for the reader to follow the arguments and understand the underlying thought process.
    \item Another point is the methods of research.
    I found it difficult to find any resources on this matter for the highly abstract field of mathematics.
    Further, I had no mentor on this project which could help me along the right way.
    \item The time management was suboptimal, as I spend a lot of time on branches that where clearly not working and thus did not have time to investigate everything I wanted.
    Also, I did not start writing down my own research until very late, which generated some stress in the end.
\end{itemize}

\section{Directions}\label{sec:directions}
In the future, multiple topics could be investigated further to get an even deeper understanding on what happens in Savitch's Theorem.

In all proofs concerning equivalences between transitive closure logics and NSPACE$[s(n)]$, we assumed that $s(n)$ is at most polynomial.
This is still a quite profound limitation, and no characterization for superpolynomial bounds is known to me.
Finding a generalization could lead to new insights that would help in the polynomial case, and in the end to the case where $s(n) = n$ and we have exactly the context-sensitive languages.

In the context of Savitch's Theorem, it is interesting to investigate the computation graphs of NSPACE and DSPACE machines.
This has been discussed on the theoretical computer science stack exchange in~\cite{Barak2010}.
A motivation for this is that the number of nodes in both graphs are the same, only the number of edges vary.

Another direction that could be investigated is looking at normal forms.
One that seems to have some potential is the Kuroda Normal Form described in~\cite{Kuroda1964}.
This approach could yield a semantic restriction similar to the one for context-free languages described in \cref{subsec:des-context-free-languages}.

After showing that mixing TC and iterative procedures does not give any strong characterizations in \cref{sec:mixing-iterations-and-transitive-closure} and seeing that all interesting cases happen when the extended variables have less than $s(n)$ bits, one direction that took up most of the time and yielded very little results came up.
A very interesting result would have been to show that any iterative formula simulating a TC computation needs to have at least variables of the same size, as this would have shown that the path of investigating any logic which is a subset of these logics.
To do this I read multiple papers concerning hierarchies over the transitive closure, Generalized quantifiers and alternating space-time.
One of the most interesting of these is ``A double arity hierarchy theorem for transitive closure logic''~\cite{Grohe1996} which combines both and shows strict hierarchies on unordered graphs.
The main problem with these approaches is that they can not be generalized to include string structures, which are inherently ordered.
This makes it impossible to connect them to Turing machines which work with these structures.
Some steps in this direction were made in the chapter about proving lower bounds in~\cite{descriptive-complexity}.
\chapter*{Acknowledgements}

I would like to express my gratitude to all the people who helped me on this journey.
First of all, I want to thank my supervisor Aline Sprunger for her guidance on mathematical research and valuable feedback.
Further, I would like to thank Dr. Gabriele Röger of the university of Basel for teaching me the basics of the theory of computer science and sparking my interest in this domain.
I am also deeply grateful to my family for their support during stressful moments.
Another thanks goes to all the people who accepted to reread this work and who gave me useful feedback.
% --------------------------


% === ABBILDUNGSVERZEICHNIS ===
\cleardoublepage %             \
\phantomsection %              \
\addcontentsline{toc}{chapter}{\listfigurename}
\listoffigures %               \
% -----------------------------

% === CODEBLOCKVERZEICHNIS ===
\cleardoublepage %            \
\phantomsection %             \
\addcontentsline{toc}{chapter}{\lstlistlistingname}
\lstlistoflistings %          \
% ----------------------------

% === LITERATURVERZEICHNIS ===
\cleardoublepage %            \
\phantomsection %             \
\addcontentsline{toc}{chapter}{\bibname}
\printbibliography %          \
% ----------------------------


% =========== START VON ANHÄNGE ===========
\appendix
%! suppress = UnresolvedReference
\chapter{Mathematical Background}\label{ch:mathematical-background}

The definitions are taken from the lectures Discrete Mathematics in Computer Science~\cite{discrete-maths} and Theory of Computer Science~\cite{theory-cs}, as well as from the book Descriptive Complexity~\cite{descriptive-complexity}.


\section{Set Theory}\label{sec:set-theory}
\begin{description}
    \item[Set] An unordered collection of distinct elements, written with curly braces $\{\}$
    \item[Tuple] An ordered collection of elements, written with pointed braces $\langle  \rangle$.
    \item[Set operations] There are multiple ways to form new sets from already existing sets:
    \begin{description}
        \item[Union] denoted as $\cup$.
        An element is in $A \cup B$ if and only if it is in $A$ or $B$
        \item[Intersection] denoted as $\cap$.
        An element is in $A \cap B$ if and only if it is in $A$ and $B$
        \item[Cartesian product] denoted as $\times$. $A \times B$ is the set of 2-tuples $\langle a, b\rangle$ with an element $a$ of $A$ and an element $b$ of $B$
        \item[Cartesian power] $A^k$ denotes the Cartesian product of $A$ with itself repeated $k$ times
        \item[Power set] denoted as $\mathcal{P}(A)$.
        Contains all subsets of $A$
    \end{description}
\end{description}


\section{First-Order Logic}\label{sec:first-order-logic}
We abbreviate first-order logic as FO\@. % TODO check acronyms
\begin{description}
    \item[Variable] A variable is an element that can have a value from a set.
    Tuples of variables are denoted by $\overline{x} = \langle x_1, \dots, x_n \rangle$
    \item[Universe] The set over which variables and constants can range
    \item[Relation] A relation of arity $k$, $R(x_1, \dots, x_k)$ can be either true or false for any $k$-tuple of variables.
    In this document, we always consider an equality relation $=$, an ordering relation $\leq$, and BIT$(x, 1y)$, which means that the $y^{th}$ bit of $x$ is $1$ in binary notation, to be present
    \item[Vocabulary] A tuple $\tau = \langle R_1^{a_1}, \dots, R_r^{a_r}, c_1, \dots, c_s \rangle$ of relations $R_i$ with arity $a_i$ and constants $c_j$\footnote{We omit functions, which are included in most textbook definitions, as they can be simulated by a relation in our case}
    \item[Structure] A tuple $\mathcal{A} = \langle |\mathcal{A}|, R_1^{\mathcal{A}}, \dots, R_r^{\mathcal{A}}, c_1^{\mathcal{A}}, \dots, c_s^{\mathcal{A}} \rangle$ where $|\mathcal{A}|$ is the universe, the constants are assigned a value from $|\mathcal{A}|$, and each relation is assigned a truth value for each $a_i$-tuple in $|\mathcal{A}|^{a_i}$.
    The set of all structures for a given universe is denoted as STRUCT$[\tau]$
    \item[First-order formula] A first-order formula is inductively defined as follows:
    \begin{description}
        \item[Atoms] Any formula of the form $R(x_1, \dots, x_k)$ for some relation of arity $k$ is called an atomic formula
        \item[Conjunction] If $\varphi$ and $\psi$ are formulas, $(\varphi \land \psi)$ is a formula
        \item[Disjunction] If $\varphi$ and $\psi$ are formulas, $(\varphi \lor \psi)$ is a formula
        \item[Negation] If $\varphi$ is a formula, $\lnot \varphi$ is a formula
        \item[Existential quantification] If $\varphi$ is a formula, $\exists x \varphi$ is a formula
        \item[Universal quantification] If $\varphi$ is a formula, $\forall x \varphi$ is a formula
    \end{description}
    \item[Free variables] A variable in a formula which occurs at least once without being bound by a quantifier whose scope surrounds it
    \item[Semantics] For any structure, we can assign a truth value to any formula over the corresponding vocabulary.
    If the formula contains free variables, these need to be assigned a value from the universe first.
    We say $\mathcal{A}$ satisfies $\phi$, denoted as $\mathcal{A} \models \phi$, if and only if $\phi$ is true under the interpretation of the constant and relations in $\mathcal{A}$.
    This truth value is inductively assigned to all formulas as follows:
    \begin{description}
        \item[Atoms] For a formula $\phi$ of the form $R(x_1, \dots, x_k)$, we have $\mathcal{A} \models \phi$ if and only if the interpretation of the relation $R^{\mathcal{A}}$ maps $\langle x_1, \dots, x_k \rangle$ to true
        \item[Conjunction] We have $\mathcal{A} \models (\varphi \land \psi)$ if and only if $\mathcal{A} \models \varphi$ and $\mathcal{A} \models \psi$
        \item[Disjunction]  We have $\mathcal{A} \models (\varphi \lor \psi)$ if and only if $\mathcal{A} \models \varphi$ or $\mathcal{A} \models \psi$
        \item[Negation] We have $\mathcal{A} \models \lnot \varphi$ if and only if $\mathcal{A} \not\models \varphi$
        \item[Existential quantification] We have $\mathcal{A} \models \exists x\varphi$ if and only if there exists a $y \in |\mathcal{A}|$ such that $\mathcal{A} \models \varphi(y / x)$, where $\varphi(y / x)$ denotes $\varphi$ with any occurrence of $x$ replaced by the element $y$
        \item[Universal quantification] We have $\mathcal{A} \models \forall x\varphi$ if and only if for all $y \in |\mathcal{A}|$ we have $\mathcal{A} \models \varphi(y / x)$
    \end{description}
    \item[Logical operator] A logical operator can create a new formula from one or more existing formulas.
    One example is the conjunction, which combines two existing formulas.
    \item[First-order queries] A first-order query is a map from structures over one vocabulary $\sigma$ to structures over another vocabulary $\tau$.
    The mapping is done in such a way that first-order formulas define the universe, which is a subset of $|\mathcal{A}|^k$ for some $k$, the relation symbols, and all the constants.
    For a more formally thorough definition see~\cite{descriptive-complexity}
    \item[Isomorphism] An isomorphism is a map $I: |\mathcal{A}| \to |\mathcal{B}|$ with $\mathcal{A}, \mathcal{B}$ over the same vocabulary which satisfies the following properties:
    \begin{itemize}
        \setlength\itemsep{0.15em}
        \item $I$ is bijective
        \item for every available relation $R_i$ of arity $a_i$ and every $a_i$-tuple $\overline{e} = \langle e_1, \dots, e_{a_i} \rangle$ in $|\mathcal{A}|^{a_i}$, we have \[R_i^{\mathcal{A}}(e_1, \cdot, e_{a_i}) \Leftrightarrow R_i^{\mathcal{B}}(I(e_1), \cdot, I(e_{a_i}))\]
        \item for every constant symbol $c_j$, we have $I(c_j^{\mathcal{A}}) = c_j^{\mathcal{B}}$
    \end{itemize}
    If such an $I$ exists for two structures $\mathcal{A}$ and $\mathcal{B}$, we write $\mathcal{A} \cong \mathcal{B}$
\end{description}


\section{Second-Order Logic}\label{sec:second-order-logic}
In second-order logic, we extend the capabilities of first-order logic with the ability to quantify over relations.
We thus also need to extend our definitions.
We abbreviate second-order logic as SO\@.
\begin{description}
    \item[Second-order variables] A relation that is not given in the vocabulary and can be substituted with a specific interpretation
    \item[Second-order formula] In addition to the inductive rules of the FO formulas, we can quantify over second-order formulas
    \begin{description}
        \item[Second-order Existential quantification] If $\varphi$ is a formula, then $\exists V\varphi$ is a formula
        \item[Second-order Universal quantification] If $\varphi$ is a formula, then $\forall V\varphi$ is a formula
    \end{description}
    \item[Second-order Semantics] We also need to extend the first-order semantics
    \begin{description}
        \item[Second-order Existential quantification]  We have $\mathcal{A} \models \exists V\varphi$ if and only if there exists a relation $U$ over $|\mathcal{A}|$ such that $\mathcal{A} \models \varphi(U / V)$, where $\varphi(U / V)$ denotes $\varphi$ with any occurrence of $V$ replaced by $U$
        \item[Second-order Universal quantification] We have $\mathcal{A} \models \forall V\varphi$ if and only if for all relations $U$ over $|\mathcal{A}|$ we have $\mathcal{A} \models \varphi(U / V)$
    \end{description}
\end{description}


\section{Turing Machines}\label{sec:turing-machines}
Turing machines are the most common model of computation.
\begin{description}
    \item[Informal definition] A Turing machine is an automaton with a finite number of states and an infinite tape.
    Using a read/write head, which can read one symbol on the tape, modify one symbol on the tape and move left and right, a Turing Machine can compute functions
    \item[Formal definition] Formally, a Turing machine is a 7-tuple $M = \langle  Q, \Sigma, \Gamma, \delta, q_0, q_{accept}, q_{reject}\rangle$, where
    \begin{description}
        \item[$Q$] is the set of states
        \item[$\Sigma$] is the alphabet of the input word
        \item[$\Gamma$] is the set of symbols which can be written or read on the tape, which we call the tape alphabet
        \item[$\delta$] is the transition function, with $\delta : \Gamma \times Q \to \Gamma \times Q \times \{L, R\}$.
        This means that when a Turing machine is in state $n$ and reads $a$ on the tape, $\delta$ tells us to which state we should transition, which symbol we should write and which direction we should move the read/write head
        \item[$q_0$] the start state
        \item[$q_{accept}$] the accepting state
        \item[$q_{reject}$] the rejecting state
    \end{description}
    \item[Turing computation] In the beginning, the Turing machine is in the start state, the input word is written on a consecutive part of the tape and the read/write head is on the first character of the input word.
    In the following steps, the Turing machine state changes according to the transition function.
    If at some point the Turing machine enters the accepting or the rejecting state, the computation halts, and the Turing machine is said to have accepted / rejected the input.
    It can happen that the Turing machine continues indefinitely or loops.
    In that case, we also say that it has rejected its input.
    In this document, we will ignore the tape content after the computation and focus on decision problems.
    \item[Decidability] If a Turing machine halts on all inputs, we say that it decides a problem, as we can always be sure that the machine will accept or reject an input in finite time.
    \item[Nondeterministic Turing machine] We can extend the transition function $\delta$ to allow multiple transitions from a given state.
    Formally, we then have $\delta : \Gamma \times Q \to \mathcal{P}(\Gamma \times Q \times \{L, R\})$.
    If at some point there are multiple transitions that are possible from the current state, we can take any of them.
    If there exists any computational path which leads to an accepting state, the nondeterministic Turing machine accepts.
    This is not analogous to how real sequential computers work, but allows interesting results, and is as powerful as a normal deterministic Turing machine.
    \item[Space/Time-Constructible functions] A function $f(n)$ is time constructible if there exists a Turing machine which on input $1^{n}$ writes $f(n)$ in binary on its tape in time $f(n)$.
    Space-constructible functions are defined analogously.
    \item[Church-Turing Thesis] The Church-Turing Thesis states that anything that can be done on a real-world computer can be done using a Turing machine.
\end{description}



% === EIGENSTÄNDIGKEITSERKLÄRUNG ===
\chapter{Independence declaration (German)}\label{ch:appendix_independencedeclaration}
Ich, Yaël Arn, 4A \\ \\
bestätige mit meiner Unterschrift, dass die eingereichte Arbeit
selbstständig und ohne unerlaubte Hilfe Dritter verfasst wurde.
Die Auseinandersetzung mit dem Thema erfolgte ausschliesslich
durch meine persönliche Arbeit und Recherche. Es wurden
keine unerlaubten Hilfsmittel benutzt.
Ich bestätige, dass ich sämtliche verwendeten Quellen sowie
Informanten/-innen im Quellenverzeichnis bzw. an anderer dafür
vorgesehener Stelle vollständig aufgeführt habe. Alle Zitate
und Paraphrasen (indirekte Zitate) wurden gekennzeichnet und
belegt. Sofern ich Informationen von einem KI-System wie
bspw. ChatGPT verwendet habe, habe ich diese in meiner Maturaarbeit
gemäss den Vorgaben im Leitfaden zur Maturaarbeit
korrekt als solche gekennzeichnet, einschliesslich der Art und
Weise, wie und mit welchen Fragen die KI verwendet wurde.
Ich bestätige, dass das ausgedruckte Exemplar der Maturaarbeit
identisch mit der digitalen Version ist.
Ich bin mir bewusst, dass die ganze Arbeit oder Teile davon
mittels geeigneter Software zur Erkennung von Plagiaten oder
KI-Textstellen einer Kontrolle unterzogen werden können.

\vspace{3cm}

\noindent
\begin{tabular}{p{0.47\linewidth}p{0.47\linewidth}}
  Ort \&\ Datum & Unterschrift \\
  & \\[1cm]
  \hline
\end{tabular}

% ----------------------------------

% =========== ENDE VON ANHÄNGE ============

% =============================================================
\end{document} % ENDE VOM DOKUMENT ============================
% =============================================================
