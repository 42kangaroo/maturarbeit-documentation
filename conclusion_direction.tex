%! suppress = UnresolvedReference
%! suppress = MissingImport

\chapter{Conclusion and Direction}\label{ch:conclusion-and-direction}

\section{Conclusion}\label{sec:conclusion}
In this document, we investigated multiple logics which have a relation to Savitch's theorem.  % TODO we / I in conclusion?
This has led us to a better notion of how these logics are related to NSPACE$[s(n)]$ and thus also to linear bounded automata and context-sensitive languages.

In the beginning, we introduced the theory of formal languages and the Chomsky Hierarchy.
Afterwards, the main tool of Descriptive Complexity used in this document, Complexity theory, was presented.
Next, Savitch's theorem and its proof were explained.
From then on, this theorem was used multiple times.
Then, the equivalence between context-sensitive languages and monadic second-order logic was shown.

Personally, I investigated multiple ideas I got when I was reading the theory.

The first one was a direct transformation of the results of the equivalence between second-order transitive closure logic and nondeterministic Turing machines with bounded space to first-order logic.
This approach did not contain any new insights, but it was able to showcase the close relationship between first- and second-order logic.

After this, I wanted to explicitly prove that some methods could not work.
This branch of my work started with the investigation of the proof that DSPACE is equivalent to VAR$[k + 1]$.
There, I showed that naively applying the proof to nondeterministic Turing machines gives wrong results and that a natural extension by remembering decisions is suboptimal.
A full proof was impossible to make as it is unclear how to define a generalization of the proof for DSPACE\@.
The difficulty of this problem is illustrated by Savitch's theorem, which has not been improved for a long time and thus makes it seem unlikely that a simulation of NSPACE Turing machines with subquadratic DSPACE Turing machines is possible.

The next method I tried was mixing transitive closure logic with iterative logic.
This turned out not to be very powerful, as any interesting result in one of the mixed logics would imply one in pure iterative logic.
Nevertheless, some upper and lower bounds were shown using Savitch's theorem and the space hierarchy theorem.

In an attempt to make FO-VAR less powerful, I also tried restricting universal quantification.
A new formula made this possible.
As for all the other restrictions, I was unable to find a way to simulate formulas in my restriction of FO-VAR using NSPACE Turing machines.
The generalization of this idea by adding more bits to the extended variables and decreasing the iteration count gave a space-time tradeoff but did not solve the inherent problem of simulating this formula by NSPACE Turing machines.

The most complicated result I was able to get does not have any tight connection to context-sensitive languages.
It concerns the simulation of alternating Turing machines using FO-VAR\@.
There, a quite tight containment was found, which bounds some alternating Turing machine classes from above and below by a factor of $\log(s(n))$.
A combination of different techniques used in previous results was used to get the logical formula.
This result could lead to a path for describing alternating space-time classes exactly using an equivalent logic.

\subsection{Working process}\label{subsec:working-process}
Generally, I learned a lot during this project.
It also provided me insight into how scientific research is done, and I was able to immerse myself in a domain that fascinates me.
I knew that I could not expect any significant results, and indeed I worked on many different ideas, but most of the time hit a dead end.

At the beginning of the working process, I spend a lot of time reading multiple books about Descriptive Complexity and logic.
I then researched more papers on the topic and tried to understand the lower and upper bounds described there.
This took more time than expected, but I still was able to start investigating early enough.
As this was my first scientific research project, I did not know how to approach it, so I wrote to Dr. Královi\v{c} from the ETH Zürich.
His tips were helpful and gave me a starting point for the research.
During the months that followed, I tried multiple ideas and read a lot of papers concerning things I wanted to prove or needed for my proofs.
Quite soon, Ms Sprunger and I discussed that in order to help reduce the stress before handing in, I should start writing the theoretical part of this document.
I did this during the summer break, experiencing some difficulties in the simplification of proofs while still maintaining their correctness.
Afterwards, I continued doing research.
During this time, I found multiple results which are presented in \cref{ch:personal-contribution}, but also spent a lot of time working on hierarchies without notable results.
Two months before handing in, the International Olympiad in Informatics took place.
My time plan did not account for this very well, so I had to work quite a lot in the last two weeks to write down my personal contribution.

Overall, I think there are three main points which could be improved:
\begin{itemize}
    \setlength\itemsep{0.2em}
    \item First, I often felt that my proofs were not explained in a way that was clear and concise.
    This made it difficult for the reader to follow the arguments and understand the underlying thought process.
    In this document I focused on this point and was able to improve the quality of the proofs.
    \item Another point is the methods of research.
    I found it difficult to find any resources on this matter in the highly abstract field of mathematics.
    I reached out to multiple professors at the University of Basel and the ETH Zürich, but unfortunately none of the people I contacted hat time to help me along the right path.
    \item The time management could also be improved, as I spent a lot of time on ideas that were clearly not working and thus did not have time to investigate everything I wanted.
    Also, I only started writing down my own research very late, which generated some stress in the end.
\end{itemize}

\section{Directions}\label{sec:directions}
Multiple topics could be investigated further and would provide an even deeper understanding of the inner workings of Savitch's theorem.

In all proofs concerning equivalences between transitive closure logics and NSPACE$[s(n)]$, we assumed that $s(n)$ is at most polynomial.
This is still a profound limitation, and finding a generalization could lead to new insights that could help in the polynomial case and in the end also in the case where $s(n) = n$, capturing exactly the context-sensitive languages.

In the context of Savitch's theorem, it would also be interesting to investigate the computation graphs of NSPACE and DSPACE Turing machines.
This has been discussed on the theoretical computer science stack exchange in~\cite{Barak2010}.
A motivation for this is that the number of nodes in both graphs is the same, only the number of edges varies.

Another approach that could be investigated is looking at normal forms.
One that seems to have some potential is the Kuroda Normal Form described in~\cite{Kuroda1964}.
This approach could yield a semantic restriction similar to the one for context-free languages described in \cref{subsec:des-context-free-languages}.

Further, one could investigate hierarchies of transitive closure logics, generalized quantifiers and alternating time-space complexity classes.
``A double arity hierarchy theorem for transitive closure logic''\cite{Grohe1996} by Grohe shows a strict hierarchy for logics which combine the transitive closure operator with generalized quantifiers but is limited to unordered structures.
As string structures are inherently ordered, this does not generalize to cases where we use Turing machines.
Some steps in the direction of hierarchies for ordered structures were made in the chapter about proving lower bounds in~\cite{descriptive-complexity}.